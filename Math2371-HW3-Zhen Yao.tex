\documentclass[12pt,leqno]{amsart}
\pagestyle{plain}
\usepackage{latexsym,amsmath,amssymb}
%\usepackage[notref,notcite]{showkeys}
\usepackage{amsfonts}
\usepackage{geometry}
\usepackage{graphicx}
\graphicspath{ {images/} }

\setlength{\oddsidemargin}{1pt}
\setlength{\evensidemargin}{1pt}
\setlength{\marginparwidth}{30pt} % these gain 53pt width
\setlength{\topmargin}{1pt}       % gains 26pt height
\setlength{\headheight}{1pt}      % gains 11pt height
\setlength{\headsep}{1pt}         % gains 24pt height
%\setlength{\footheight}{12 pt} 	  % cannot be changed as number must fit
\setlength{\footskip}{24pt}       % gains 6pt height
\setlength{\textheight}{650pt}    % 528 + 26 + 11 + 24 + 6 + 55 for luck
\setlength{\textwidth}{460pt}     % 360 + 53 + 47 for luck

\newtheorem{theorem}{Theorem}



\def\dsp{\def\baselinestretch{1.35}\large
\normalsize}
%%%%This makes a double spacing. Use this with 11pt style. If you
%%%%want to use this just insert \dsp after the \begin{document}
%%%%The correct baselinestretch for double spacing is 1.37. However
%%%%you can use different parameter.


\def\U{{\mathcal U}}

\begin{document}

\centerline{\bf Homework 3 for Math 2371}
\centerline{Zhen Yao}

\medskip

\noindent{\bf Problem 1.} Let 
\begin{align*}
    A = \begin{pmatrix}
    1 & 2 & 0 \\
    0 & 1 & 2
    \end{pmatrix}.
\end{align*}
Find all of its positive singular values.
\begin{proof}
Singular values of $A$ are eigenvalues of $AA^* = \begin{pmatrix}
    5 & 2  \\
    2 & 5 
    \end{pmatrix}$, which are $\lambda_1 = 7$ and $\lambda_2 = 3$.
\end{proof}

\medskip

\noindent{\bf Problem 2.} Let $m \geq n \geq 2$ and $A$ be an $m\times n$ matrix. Suppose $\sigma_1 \geq \sigma_2 \geq \cdots \geq \sigma_n \geq 0$ are singular values of $A$. Suppose $u_1\in \mathbf{R}^m, v_1 \in \mathbf{R}^n$  are unit vectors such that 
\begin{align*}
    (u_1, Av_1) = \sigma_1.
\end{align*}
Show that 
\begin{align*}
    \sigma_2 = \max_{ \substack{\|u\| = \|v\| = 1\\
                     (u,u_1) = (v, v_1) = 0}}
    (u, Av).
\end{align*}
\begin{proof}
Suppose $\sigma_2 = (u_2, Av_2)$ and $\|u_2\| = \|v_2\| = 1$. Define for any vector $g,h$:
\begin{align*}
    f(s,t) = \frac{(u_2+sg, A(v_2+th))}{\|u_2+sg\|\cdot \|v_2+th\|},
\end{align*}
and we have
\begin{align*}
    \frac{\partial f}{\partial s}\bigg|_{(0,0)} & = \frac{(g, Av_2) - (u_2, Av_2)(u_2, g)}{\|u_2+sg\|^2\cdot \|v_2+th\|^2} \\
    & = \frac{(g, Av_2) - \sigma_2(u_2, g)}{\|u_2+sg\|^2\cdot \|v_2+th\|^2}.
\end{align*}
Since $\sigma_2 = (u_2, Av_2)$, then we have $(u_2,\sigma_2 u_2) = (u_2, Av_2)$, which implies $\sigma_2 u_2 = Av_2$. Then, $\frac{\partial f}{\partial s}\bigg|_{(0,0)} = 0$. Similarly, we have $\frac{\partial f}{\partial t}\bigg|_{(0,0)} = 0$. Thus, $f(x,t)$ obtain its maximum at $(0,0)$ if $\sigma_2 = (u_2, Av_2)$. And with $(u_1, Av_1) = \sigma_1$, we have $Av_1 = \sigma_1 u_1$ and if $(u_2,u_1) = (v_2, v_1) = 0$, we can have 
\begin{align*}
    (Av_1, v_2) = (\sigma_1 u_1, v_2) = 0,
\end{align*}
which implies $(u_1, v_2) = 0$. Thus, we have $\sigma_2 = \max_{ \substack{\|u\| = \|v\| = 1\\
                     (u,u_1) = (v, v_1) = 0}}
    (u, Av)$.
\end{proof}









\end{document}
