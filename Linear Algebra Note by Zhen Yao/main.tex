\documentclass[12pt]{book}
\pagestyle{plain}
%\documentclass{article}
\usepackage[utf8]{inputenc}
\usepackage[english]{babel}

\usepackage{latexsym,amsmath,amssymb}
\usepackage{amsthm}
%\usepackage[notref,notcite]{showkeys}
\usepackage{amsfonts}
\usepackage{geometry}
\usepackage{graphicx}
\usepackage{lmodern}
\usepackage{pifont}
\usepackage{tikz}
\usepackage{pgfplots}
\usepackage{thmtools}
\usepackage{wrapfig}
\usepackage{extarrows}
\usepackage{breqn}
\usepackage{physics}
\usepackage{afterpage}
\usepackage{enumitem}
\usepackage[utf8]{inputenc}
\usepackage{mathrsfs}
\usepackage{scalerel}
\usepackage{stackengine,wasysym}
\usepackage{aligned-overset}
\usepackage{stackengine}
\usepackage{mathtools}
\usepackage{nccmath}
\graphicspath{ {images/} }


\newcommand*\circled[1]{\tikz[baseline=(char.base)]{
            \node[shape=circle,draw,inner sep=2pt] (char) {#1};}}

\setlength{\oddsidemargin}{1pt}
\setlength{\evensidemargin}{1pt}
\setlength{\marginparwidth}{30pt} % these gain 53pt width
\setlength{\topmargin}{1pt}       % gains 26pt height
\setlength{\headheight}{1pt}      % gains 11pt height
\setlength{\headsep}{1pt}         % gains 24pt height
%\setlength{\footheight}{12 pt} 	  % cannot be changed as number must fit
\setlength{\footskip}{24pt}       % gains 6pt height
\setlength{\textheight}{650pt}    % 528 + 26 + 11 + 24 + 6 + 55 for luck
\setlength{\textwidth}{460pt}     % 360 + 53 + 47 for luck

\title{Sections and Chapters}

\newtheorem{definition}{Definition}[section]
\newtheorem{theorem}{Theorem}[section]
\newtheorem{corollary}{Corollary}[section]
\newtheorem{lemma}[theorem]{Lemma}
\newtheorem{proposition}{Proposition}[section]
\newtheorem{exercise}{Exercise}[section]
\newtheorem{remark}{Remark}[section]
\theoremstyle{definition}
\newtheorem{example}{Example}[section]
\numberwithin{equation}{subsection}

\def\dsp{\def\baselinestretch{1.35}\large
\normalsize}
%%%%This makes a double spacing. Use this with 11pt style. If you
%%%%want to use this just insert \dsp after the \begin{document}
%%%%The correct baselinestretch for double spacing is 1.37. However
%%%%you can use different parameter.

\newcommand\blankpage{%
    \null
    \thispagestyle{empty}%
    \addtocounter{page}{-1}%
    \newpage}
    
\def\U{{\mathcal U}}

\begin{document}
\frontmatter

\begin{titlepage}
	\begin{center}
	\textbf{\LARGE{}} \\
	\vspace{40mm}
    \textbf{\LARGE{Linear Algebra Note}} \\
    \vspace{2mm} %5mm vertical space
    \large{\textsc{Zhen Yao, Department of Mathematics}}\\
    \large{\textsc{University of Pittsburgh}}
    \end{center}
\end{titlepage}

\tableofcontents{}
\mainmatter

\newpage

\chapter{Fundamentals of Linear Spaces}
\section{Linear Spaces, Isomrphism}

A \emph{field} $K$ is a nonempty set in which two operations are defined, usually called addition and multiplication, denoted by $+$ and $\cdot$ respectively such that it satisfies the following axioms:
\begin{enumerate}[label=(\arabic*)]
    \item $K$ is closed under addition and multiplication, i.e., if $a,b\in K$, then $a+b, a\cdot b\in K$.
    \item Associativity of addition and multiplication, i.e., for any $a,b,c\in K$, $a+(b+c) = (a+b)+c, a\cdot(b\cdot c) = (a\cdot b)\cdot c$.
    \item Existence of additive and multiplicative identity elements, i.e., there exists an element of $K$ called additive identity, denoted by $0$, such that for any $a
    \in K, a+0=a$. Similarly, there exists an element of $K$ called multiplicative identity, denoted by $1$, such that for any $a\in K, a\cdot 1=a$.
    \item Existence of additive inverse and multiplicative inverse, i.e., for any $a\in K$, there exists an element $-a\in K$, such that $a+(-a)=0$. Similarly, for any $a\in K\setminus \{0\}$, there exists an element $a^{-1}\in K$, such that $a\cdot a^{-1}=1$.
    \item Distributivity of mulitiplication over addition, i.e., for any $a,b,c\in K, a\cdot(b+c)=a\cdot b+a\cdot c$.
\end{enumerate}

\medskip

\begin{example}
Examples of field: $\mathbb{R}$, $\mathbb{Q}$, $\mathbb{C}$. When $K$ is $\mathbb{R}$ or $\mathbb{C}$, the elements of $K$ are called scalars.
\end{example}

\medskip

\begin{example}
Some important structures are “very nearly”fields. For example, let $\mathbb{R}_\infty = \mathbb{R} \cup \{\infty\}$, and define operations $\boxplus$ and $\boxdot$ on $\mathbb{R}_\infty$ as
\begin{align*}
    a \boxplus b = \begin{cases} 
    \min\{a,b\} & {\rm if}\, a,b\in \mathbb{R}, \\
    b & {\rm if}\, a = \infty, \\
    a & {\rm if}\, b = \infty.
    \end{cases}\,\, {\rm and}\,\,\,
    a \boxdot b = \begin{cases} 
    a+b & {\rm if}\, a,b\in \mathbb{R}, \\
    \infty & {\rm otherwise}.
    \end{cases}
\end{align*}
This structure, called the \emph{optimization algebra}, satisfies all of the conditions of a field \emph{except} for the existence of additive inverse (such structures are known as \emph{semifields}\cite{2}).
\end{example}

\medskip

\begin{example}
Fields do not have to be infinite. Let $p$ be a positive integer and $\mathbb{Z}/(p) = \{1,2,\cdots,p-1\}$. For each nonnegative integer $n$, denote the remainder after dividing $n$ by $p$ as $[n]_p$. Then it is easy to see that $[n]_p\in\mathbb{Z}/(p)$ for each nonnegative integer $n$ and $[i]_p = i$ for all $i\in\mathbb{Z}/(p)$. 

We now define operations on $\mathbb{Z}/(p)$ by setting $[n]_p + [k]_p = [n+k]_p$ and $[n]_p\cdot [k]_p = [n\cdot k]_p$. It is easy to check that if the integer $p$ is prime, then $\mathbb{Z}/(p)$ with two operations is a field, known as the \emph{Galois field} of order $p$, usually denoted by $GF(p)$. 
\end{example}

\medskip

\begin{proposition}
Let $a$ be an nonzero element of a finite field $K$ which contains $q$ elements. Then $a^{-1} = a^{q-2}$.
\end{proposition}
\begin{proof}
If $q = 2$, then $K = {\rm GF}(2)$ and $a = 1$. Then the result is obvious. 

If $q > 2$, let $B = \{a_1, \cdots, a_{q-1}\}$ be nonzero elements of $K$. Then $aa_i\neq aa_k$ for $i\neq k$. If not, we would have $a_i = a^{-1}(aa_k) = a_k$. Therefore, $B = \{a a_1, \cdots, a a_{q-1}\}$ and we have 
\begin{align*}
    \prod^{q-1}_{i=1}a_i = \prod^{q-1}_{i=1}aa_i = a^{q-1}\prod^{q-1}_{i=1}a_i 
\end{align*}
Then we have $a^{q-1} = 1 = a a^{-1}$, which implies $a^{-1} = a^{q-2}$.
\end{proof}

\medskip
\begin{definition}
Now we define the characteristic of a field $K$ to be equal to the smallest positive integer $p$ such that $1+\cdots+1$($p$ summands) equals $0$—if such an integer $p$ exists—and to be equal to $0$ otherwise.
\end{definition}

\medskip

\begin{proposition}
If $K$ is a field having characteristic $p > 0$, then $p$ is prime.
\end{proposition}
\begin{proof}
Suppose by contrary that $p = xy, 0<x,y<p$. Therefore $a = x$ nd $b = y$ are nonzero elements of $K$ and we have $ab = xy = 0$, which implies $a=b=0$. Then there is a contradiction. 
\end{proof}

\medskip

\begin{theorem}[Loo-Keng Hua's Identity]
If $a$ and $b$ are nonzero elements of a field $K$ satisfying $a\neq b^{-1}$, then
\begin{align*}
    a - aba = \left(a^{-} + (b^{-1} - a)^{-1}\right)^{-1}
\end{align*}
\end{theorem}
\begin{proof}
We have
\begin{align*}
    a^{-} + (b^{-1} - a)^{-1} & = a^{-1}\left((b^{-1} - a) + a \right)(b^{-1} - a)^{-1}\\
    & = a^{-1}b^{-1}(b^{-1} - a)^{-1} \\
    \Rightarrow \left(a^{-} + (b^{-1} - a)^{-1}\right)^{-} & = (b^{-1} - a)ba = a - aba
\end{align*}
\end{proof}


\medskip

Now we introduce the term of linear space. 
\begin{definition}
A linear space $X$ over a field $K$ is a set in which two operations are defined: 
\begin{enumerate}[label=(\arabic*)]
    \item Addition, denoted by $+$, such that for any $x,y
    \in X, x+y\in X$.
    \item Scalar multiplication, denoted by $\cdot$, such that for $a\in K$ and $x\in X$, $aX\in X$.
\end{enumerate}
And these two operations satisfy the following axioms:
\begin{enumerate}[label=(\arabic*)]
    \item Associativity of addition, i.e., for $x,y,z\in X, x+(y+z)=(x+y)+z$.
    \item Commutativity of addition, i.e., for $x,y\in X, x+y=y+x$.
    \item Identity element of addiiton, i.e., for all $x\in X$, there exists an element $0\in X$, called the zero vector, such that $x+0=x$.
    \item Inverse element of addition, i.e., for all $x\in X$, there exists an element $-x\in X$, called the additive inverse of $x$, such that $x+(-x)=0$.
    \item Compatibility(Associativity) of scalar multiplication with field multiplication, i.e., for any $a,b\in K$ and $x\in X$, $a\cdot(b\cdot x)=(a\cdot b)\cdot x$.
    \item Identity element of scalar multiplication, i.e., for all $x\in X$, there exists an element $1\in X$, such that $1\cdot x=x$.
    \item Distributivity of scalar multiplicaytion with respect to vector addition, i.e., for $a\in K, x,y\in X, a\cdot(x+y)=a\cdot x+a\cdot y$.
    \item Distributivity of scalar multiplicaytion with respect to field addition, i.e., for $a,b\in K, x\in X, (a+b)\cdot x=a\cdot x+b\cdot y$.
\end{enumerate}
\end{definition}

\begin{remark}
Zero vector is unique.
\end{remark}
\begin{proof}
If there exist two zeros $0_1$ and $0_2$ in $X$, then for all $x\in X$, we have $x+(-x)=0_1$, $x+(-x)=0_2$. Then $0_1 = 0_2$.
\end{proof}

\begin{remark}
$0x = x, (-1)\cdot x = -x$.
\end{remark}

\medskip

\begin{example}[Examples of Linear Spaces]\label{example_1}
~\begin{enumerate}[label=(\roman*)]
    \item $\mathbb{R}^n, \mathbb{C}^n$.
    \item Set of all row vectors: $(a_1,\cdots, a_n)$ in $K$, this space is denoted as $K^n$.
    \item Set of all real-valued functions $f(x)$ defined on the real line, $K=\mathbb{R}$.
    \item Set of all functions with values in $K$, defined on an arbitrary set $S$.
    \item Set of all polynomials with real coefficients of order at most $n$.
\end{enumerate}
\end{example}

\medskip

\begin{definition}
A one-to-one corresponding between two linear spaces over the same field that maps sum into sum and scalar multiples into scalar multiples is called isomorphism.
\end{definition}

\begin{example}
The linear space of real valued functions on $\{1,2,\cdots,n\}$ is isomorphic to $\mathbb{R}^n$.
\end{example}

\medskip

\begin{example}
The set $(ii)$ and $(v)$ in example (\ref{example_1}) are isomorphic.
\end{example}
\begin{proof}
Polynomials can be written as $a_1+a_2x+a_3x^2+\cdots+a_nx^{n-1}$, where we can represent this as $p(x) = (a_1,a_2,\cdots,a_n)x$. Then we can define a map $p(a_1,a_2,\cdots,a_n) = a_1+a_2x+a_3x^2+\cdots+a_nx^{n-1}$, which is an isomorphism.
\end{proof}

\medskip

\begin{example}
If $S$ in $(iv)$ has $n$ elements, then $(ii)$ and $(iv)$ in example (\ref{example_1}) are isomorphic.
\end{example}
\begin{proof}
Assume $S = \{x_1, x_2,\cdots, x_n\}$, then we can define $T(f) = (f(x_1), f(x_2), \cdots, f(x_n))\in K^S$, which is indeed an isomorphism.
\end{proof}

\medskip

\begin{example}
If $K=\mathbb{R}$ in $(v)$, then $(v)$ and $(iv)$ in example (\ref{example_1}) are isomorphic when S
consists of $n$ distinct points of $\mathbb{R}$.
\end{example}

\medskip

\section{Subspace}
\begin{definition}
A subset $Y$ of a linear space $X$ is said to be subspace if sums and scalar multiples of elements of $Y$ belong to $Y$. The set $\{0\}$ consisting of the zero element of a linear space $X$ is a subspace of $X$, called the trivial subspace.
\end{definition}

\medskip

\begin{definition}
The sum of two subsets $Y$ and $Z$ of a linear space $X$, is the set defined by 
\begin{align*}
    Y+Z = \{y+z\in X: y\in Y, z\in Z\}
\end{align*}
The intersection of two subsets $Y$ and $Z$ of a linear space $X$, is the set defined by
\begin{align*}
    Y\cap Z = \{x\in X: x\in Y, x\in Z\}
\end{align*}
\end{definition}

\medskip

\begin{proposition}
If $Y$ and $Z$ are two linear subspaces of $X$, then both $Y+Z$ and $Y\cap Z$ are linear subspaces of $X$.
\end{proposition}
\begin{remark}
The union of two subspaces many not be a subspace. For exapmle, two lines that intersect into one point in $\mathbb{R}^2$, then the union of these two lines is not a subspace.
\end{remark}

\medskip

\section{Algebra Over a Field}
A vector space $X$ over a field $K$ is an \emph{K-algebra} if and only if there exists a function $X\times X\ni (x,y)\mapsto x\cdot y\in X$ such that
\begin{enumerate}[label=(\arabic*)]
    \item $x\cdot (y+z) = x\cdot y + x\cdot z$,
    \item $(x+y)\cdot z = x\cdot z + y\cdot z$,
    \item $a(x\cdot y) = (ax)\cdot y = x\cdot (ay)$.
\end{enumerate}
for all $x,y,z\in X$ and $a\in K$. And these conditions suffice to show that $0\cdot x = x\cdot 0 = 0$ for all $x\in X$. Indeed, $0\cdot x = (-x + x)\cdot x = (-x)\cdot x + x\cdot x = - (x\cdot x) + (x\cdot x) = 0$.
\begin{remark}
The operation $\cdot$ need not be associative, nor need there exist an identify element for this operation.
\end{remark}


If this operation is associative, i.e., it satisfies $x\cdot (y\cdot z) = (x\cdot y)\cdot z$ for all $x,y,z\in X$, then the algebra is called an \emph{associative K-algebra}. If an identify element for operation $\cdot$ exists, i.e., there exists an element $e\neq 0\in X$ satisfying $e\cdot x = x\cdot e = x$ for all $x\in X$, then we call this $K$-algebra $(X,\cdot)$ is \emph{unital}. 

If $x$ is an element of an associative $K$-algebra $(K,\cdot)$ and if $n$ is a positive integer, we write $x^n$ instead of $x\cdot x \cdots x \cdot x$ ($n$  factors). If $X$ is also unital and has a multiplicative identity $e$, we set $x^0 = e$ for all $x \in X, x\neq 0$. The element $0^0$ is not defined. 

If $x\cdot y = y\cdot x$ for all $x,y\in X$ in some $K$-algebra $(X,\cdot)$, then the algebra is \emph{commutative}. An $F$-algebra $(X,\cdot)$ satisfying $x\cdot y = - y\cdot x$ is called \emph{anticommutative}. If the characteristic of $K$ is other than $2$, then this condition is equivalent to the condition that $x\cdot x = 0$ for all $x\in X$.

If $(X,\cdot)$ is an associative and $K$-algebra having a multiplicaiton identity $e$, and if $x\in X$ satisfies the condition that there exists an element $y\in X$ such that $x\cdot y = y\cdot x = e$, then we say that $x$ is a \emph{unit} of $X$. If such element $y$ exists, then it is unique and denoted by $x^{-1}$. Also, if $x,y$ are units of $X$, then so is $x\cdot y$. Indeed,
\begin{align*}
    (x\cdot y)\cdot \left(y^{-1}\cdot x^{-1}\right) & = \left(x\cdot \left(y\cdot y^{-1}\right)\right)\cdot x^{-1} \\
    & = (x\cdot e)\cdot x^{-1} = e
\end{align*}
similarly, $\left(y^{-1}\cdot x^{-1}\right)\cdot(x\cdot y) = e$. 
\begin{remark}
Loo-Keng Hua’s identity holds in any associative unital $F$-algebra in which the inverses exist, since the proof relies only on associativity of addition and multiplication and distributivity of multiplication over addition\cite{2}.
\end{remark}

\medskip

\begin{example}[Examples of $X$-algebra]
~\begin{enumerate}[label=(\arabic*)]
    \item Any vector space $V$ over a field $K$ can be turned into an associative and commutative $K$-algebra which is not unital by setting $x\cdot y = 0$ for all $x,y\in V$.
    \item If $F$ is a subfield of $K$, then $K$ has the structure of an associative $F$-algebra, with multiplication being the multiplication in $K$. Thus, $\mathbb{C}$ is an $\mathbb{R}$-algebra and $\mathbb{Q}(\sqrt{p})$ is a $\mathbb{Q}$-algebra for prime number $p$.
\end{enumerate}
\end{example}

\medskip

\begin{definition}
Let $K$ be a field. An anticommutative $K$-algebra $(X,\cdot)$ is a Lie algebra over $K$ if and only if it satisfies Jacobi identity:
\begin{align*}
    x\cdot (y\cdot z) + y\cdot (z\cdot x) + z\cdot (x\cdot y) = 0
\end{align*}
for all $x,y,z\in X$. This algebra is not associative unless $x\cdot y = 0$ for all $x,y\in X$.
\end{definition}

One particular Lie algebra on $\mathbb{R}^3$ is defined with multiplication $\times$, called \emph{cross product}, as below
\begin{align*}
    \begin{bmatrix}
    a_1 \\
    a_2 \\
    a_3
    \end{bmatrix} \times 
    \begin{bmatrix}
    b_1 \\
    b_2 \\
    b_3
    \end{bmatrix} = 
    \begin{bmatrix}
    a_2 b_3 - a_3 b_2 \\
    a_3 b_1 - a_1 b_3 \\
    a_1 b_2 - a_2 b_1
    \end{bmatrix}.
\end{align*}
and $(\mathbb{R}^3,\times)$ is a Lie algebra over $\mathbb{R}$. Moreover, the cross product is the only possible anticommutative product which can be defined on $\mathbb{R}^3$. Indeed, if $\cdot$ is any such product defined on $\mathbb{R}^3$, then 
\begin{align*}
    \begin{bmatrix}
    a_1 \\
    a_2 \\
    a_3
    \end{bmatrix}\cdot 
    \begin{bmatrix}
    b_1 \\
    b_2 \\
    b_3
    \end{bmatrix} & = \left(\sum^3_{i=1} a_i x_i\right)\cdot \left(\sum^3_{i=1} b_i x_i\right) = \sum^3_{i=1}\sum^3_{j=1} a_i b_j (x_i\cdot x_j)\\
    & = \begin{bmatrix}
    a_2 b_3 - a_3 b_2 \\
    a_3 b_1 - a_1 b_3 \\
    a_1 b_2 - a_2 b_1
    \end{bmatrix} = \begin{bmatrix}
    a_1 \\
    a_2 \\
    a_3
    \end{bmatrix} \times 
    \begin{bmatrix}
    b_1 \\
    b_2 \\
    b_3
    \end{bmatrix}.
\end{align*}


\medskip

\section{Linear Dependence}

\begin{definition}
A linear combination of $m$ vectors $x_1,\cdots, x_m$ of a linear space is a vector of the form
\begin{align*}
    \sum^m_{j=1} c_j x_j, {\rm where}\, c_j \in K
\end{align*}
Given $m$ vectors $x_1,\cdots, x_m$ of a linear space $X$, the set of all linear combinations of $x_1,\cdots, x_m$ is a subspace of $X$, and it is the smallest subspace of $X$
containing $x_1,\cdots, x_m$. This is called the subspace spanned by $x_1,\cdots, x_m$.
\end{definition}

\begin{definition}
A set of vectors $x_1,\cdots, x_m$ in $X$ spans the whole space $X$ if every $x$ in $X$ can be expressed as a linear combination of $x_1,\cdots, x_m$.
\end{definition}

\begin{definition}
The vectors $x_1,\cdots, x_m$ are called linearly dependent if there exist scalars $c_1,\cdots, c_m$, not all of them are zero, such that
\begin{align*}
    \sum^m_{j=1} c_j x_j = 0
\end{align*}
The vectors $x_1,\cdots, x_m$ are called linearly independent if they are not dependent. 
\end{definition}

\begin{definition}
A finite set of vectors which span $X$ and are linearly independent is called a basis for $X$.
\end{definition}

\medskip

\begin{proposition}
A linear space which is spanned by a finite set of vectors has a basis.
\end{proposition}
\begin{proof}
Let $m$ be the smallest number such that there exist $x_1,\cdots,x_m \in X$, and $X = {\rm span}\{x_1,\cdots,x_m\}$. If $x_1,\cdots,x_m$ are linearly dependent, then there exist $c_1,\cdots,c_m\in K$, not all of them are zero, such that
\begin{align*}
    \sum^m_{j=1}c_j x_j = 0
\end{align*}
Suppose without losing generality that $c_1 \neq 0$, then 
\begin{align*}
    c_1 x_1 + c_2 x_2 + \cdots + c_m x_m = 0\\
    \Rightarrow x_1 = -\frac{c_2}{c_1}x_2 - \cdots - \frac{c_m}{c_1}x_m
\end{align*}
then $\{x_2,\cdots,x_m\}$ is also a span of $X$, which is a contradiction.
\end{proof}

\medskip

\begin{theorem}
All bases for a finite-dimensional linear space $X$ contain the
same number of vectors. This number is called the dimension of $X$ and is denoted as ${\rm dim} X$.
\end{theorem}
\begin{proof}
The theorem follows from the lemma below.
\end{proof}

\begin{lemma}
Suppose that the vectors $\{x_1,\cdots,x_n\}$ span a linear space $X$ and that
the vectors $\{y_1,\cdots,y_m\}$ in $X$ are linear independent. Then $m \leq n$.
\end{lemma}
\begin{proof}
Since ${\rm span}\{x_1,\cdots,x_m\} = X$, then for $y_1$, we have $y_1 = \sum^n_{j=1}c_j x_j \neq 0$. Then, for some $k$ such that $c_k\neq 0$, we have
\begin{align*}
    c_k x_k & = y_1 - \sum^n_{j=1, j\neq k} c_j x_j \\
    x_k & = \frac{y_1}{c_k} - \sum^n_{j=1, j\neq k} \frac{c_j}{c_k} x_j
\end{align*}
Then we have $\{y_1\}\cup\{x_1,\cdots,x_{j-1},x_{j+1},\cdots, x_n\}$ can span $X$. Then, $y_2$ can be written as a linear combination of $y_1$ and $\{x_j\}_{j\neq k}$. Some coefficients for $x_j, j\neq k$ must be nonzero, since $y_1$ and $y_2$ are linearly independent. Then we can replace $x_j$ by $y_2$, and continue this process until it spans $X$. If $m \geq n$, then $n$ steps total yields that $y_1,\cdots,y_m$ span $X$. If $m > n$, this contradicts the linear independence of the vectors $y_1,\cdots,y_m$.
\end{proof}

\begin{remark}
The dimension of the trivial space consisting of the single element $0$ is zero.
\end{remark}

\medskip

\begin{theorem}
Every linearly independent set of vectors $y_1,\cdots,y_n$ in a finite dimensional linear space $X$ can be completed to a basis of $X$.
\end{theorem}
\begin{proof}
If ${\rm span}\{y_1,\cdots,y_n\} \neq X$, then there exists $y_{n+1}\in X \setminus \{y_1,\cdots,y_n\}$. We can continue this process if ${\rm span}\{y_1,\cdots,y_n, y_{n+1}\}\neq X$. Since ${\rm dim}X \subset \infty$, then the process will stop after finitely many step, which  constructs a basis of $X$.
\end{proof}

\medskip

\begin{theorem}
Let $X$ be a finite dimensional space over $K$ with ${\rm dim}X = n$, then $X$ is isomorphic to $K^n$.
\end{theorem}
\begin{proof}
Let $x_1, \cdots, x_n$ be a basis of $X$. For any $x\in X$, we have $x = \sum^n_{k=1}c_k x_k$. We can define $\varphi: X\to K^n$ as $\varphi(x) = (c_1, \cdots, c_n) \in K^n$. Then $\varphi$ is an isomorphism.
\end{proof}

\medskip

\begin{theorem}
~\begin{enumerate}[label=(\alph*)]
    \item Every subspace $Y$ of a finite dimensional linear space $X$ is of finite dimensional.
    \item Every subspace $Y$ has a complement in $X$, that is, another subspace $Z$ such that every vector $x$ in $X$ can be decomposed uniquely as
    \begin{align*}
        x = y + z, y\in Y, z\in Z.
    \end{align*}
    Furthermore ${\rm dim}X = {\rm dim}Y + {\rm dim}Z$.
\end{enumerate}
\end{theorem}
\begin{proof}
~\begin{enumerate}[label=(\alph*)]
    \item Construct a finite basis for $X$ and pick $y_1\in Y, y_1\neq 0$. If ${\rm span}\{y_1\} = Y$, then we are done. Otherwise, we can pick $y_2\in Y\setminus\{y_1\}$ and $y_1, y_2$ are linearly independent. And we can continue this process and this process will stop in finite steps, since we cannot find more than ${\rm dim}X$ linearly independent vectors. 
    \item Let $\{y_1, \cdots, y_m\}$ be a basis of $Y$ and ${\rm dim}Y = m$. Then we can complete it into a basis of $X$, saying ${\rm span}\{y_1, \cdots, y_m, y_{m+1},\cdots, y_n\} = X$. 
    
    We define $Z = {\rm span}\{y_{m+1},\cdots, y_n\}$, and then ${\rm dim}Z = n - m$. For any $x\in X$, we have 
    $$x = \sum^n_{k=1}c_k x_k, y = \sum^m_{k=1}c_k x_k, z = \sum^m_{k=m+1}c_k x_k$$ 
    then we have $x = y+z$. If $x = \Tilde{y} + \Tilde{z}, \Tilde{y}\in Y, \Tilde{z}\in Z$, then we have $\Tilde{y} + \Tilde{z} = y + z$, which implies $\Tilde{y} - y = z - \Tilde{z}$. Since $Y,Z$ are subspaces of $X$, then we can have 
    \begin{align*}
        \Tilde{y} - y = \sum^m_{k=1}a_ky_k & = \sum^m_{k=m+1}b_ky_k = z - \Tilde{z} \\
        \Rightarrow \sum^m_{k=1}a_ky_k & - \sum^m_{k=m+1}b_ky_k = 0
    \end{align*}
    Since $y_1,\cdots, y_n$ are linearly independent, then $a_k = b_k = 0$, which implies that $\Tilde{y}=y, \Tilde{z}=z$.
\end{enumerate}
\end{proof}
\begin{remark}
~\begin{enumerate}[label=(\alph*)]
    \item $Y\cap Z = \{0\}$.
    \item $X$ is said to be direct sum of $Y$ and $Z$, if $X = Y+Z$ and $Y\cap Z = \{0\}$. Then we write $X = Y \oplus Z$.
\end{enumerate}
\end{remark}

\medskip

\begin{definition}
$X$ is said to be a direct sum of its subspaces $Y_1, \cdots, Y_m$ if every $x\in X$ can be uniquely expressed as
\begin{align*}
    x = \sum^m_{j=1}y_j, y_j\in Y_j
\end{align*}
We write $X = Y_1 \oplus \cdots \oplus Y_m$. Furthermore, ${\rm dim}X = {\rm dim}Y_1 +\cdots + {\rm dim}Y_m$.
\end{definition}

\medskip

\begin{exercise}
Prove that if $X = Y_1 \oplus \cdots \oplus Y_m$, then ${\rm dim}X = {\rm dim}Y_1 +\cdots + {\rm dim}Y_m$.
\end{exercise}
\begin{proof}
Suppose $y_{i1}, \cdots, y_{in_i}$ form a basis for $Y_i, 1\leq i\leq m$. Then for any $x\in X$, we have $x = x_1 +\cdots +x_m$, where $x_i\in Y_i$. Also, we can express $x_i$ as $x_i = \sum^{n_i}_{k=1}c_{i k} y_{i k}$. Then we have 
\begin{align*}
    x = \sum^m_{i=1}\sum^{n_i}_{k=1}c_{i k} y_{i k}
\end{align*}
If $\sum^m_{i=1}\sum^{n_i}_{k=1}c_{i k} y_{i k} = 0$ for some $c_{ik}\neq 0$, then it contradicts with the definition of then direct sum.
\end{proof}

\medskip

\begin{exercise}
Prove that every finite dimensional space $X$ over field $K$ is isomorphic to $K^n$, where $n = {\rm dim}X$. And this isomorphism is not unique if $n > 1$.
\end{exercise}
\begin{proof}
Suppose $x_1,\cdots, x_n$ form a basis for $X$. Then for any $x\in X$, it can be expressed as $x = \sum^n_{k=1}c_k x_k$. We define $T(x) = (c_1, \cdots, c_n)\in K$, then this is an isomorphism. However, different choice of basis will give different isomorphism.
\end{proof}

\medskip

\section{Quotient Space}
\begin{definition}
For $X$ being a linear space, and $Y$ being a subspace of $X$, we say that two vectors $x_1, x_2\in X$ are congruent modulo $Y$, denoted by
\begin{align*}
    x_1 \equiv x_2 \bmod Y
\end{align*}
if $x_1-x_2\in Y$.
\end{definition}

Congruent $\bmod Y$ is an equivalence relation, i.e., it satisfies
\begin{enumerate}[label=(\arabic*)]
    \item Symmetric, i.e., if $x_1\equiv x_2$, then $x_2\equiv x_1$.
    \item Reflexive, i.e., $x\equiv x$ for all $x\in X$.
    \item Transitive, i.e., if $x_1\equiv x_2$ and $x_2\equiv x_3$, then $x_1\equiv x_3$.
\end{enumerate}
Thus, we can divide elements of $X$ into congruence classes $\bmod Y$ . The congruence class containing the vector $x$ is the set of all vectors congruent with $X$, denoted by $\{x\}$.

The set of congruence classes can be made into a linear space by defining addition and multiplication by scalars in $K$, as follows:
\begin{align*}
    \{x\} + \{y\}& = \{x+y\}\\
    a\{x\} &= \{ax\}
\end{align*}
That is, the sum of the congruence class containing $x$ and the congruence class containing $y$ is the class containing $x + y$. Similarly for multiplication by scalars.

The linear space of congruence classes defined above is called the quotient space of $X \bmod Y$ and is denoted as $X/Y$. 

\medskip

\begin{example}
Taking $X$ to be the linear space of all row vectors $(x_1,\cdots, x_n)$ with n components, and take $Y$ to be all vectors $y = (0, 0, x_3,\cdots, x_n) $ whose first two components are zero. Then two vectors are congruent $\bmod Y$ if and only if their first two components are equal. Each equivalence class can be represented by a vector with two components, the common components of all vectors in the equivalence class. 
\end{example}

\medskip

\begin{exercise}
Prove that two congruence classes are either identical or disjoint. 
\end{exercise}
\begin{proof}
For $\{x\}$ and $\{y\}$ are congruence classes $\bmod Y$, if there exists $z\in \{x\}\cap \{y\}$, then $x-z \in Y$ and $y-z\in Y$. Then we have $x-y = x-z-(y-z) \in Y$. So if $\{x\}\cap \{y\}\neq \varnothing$, then $x\equiv y \bmod Y$, which means $\{x\} = \{y\}$. Otherwise, $\{x\}$ and $\{y\}$ are disjoint.
\end{proof}

\medskip

\begin{theorem}\label{theorem_quotient_dim}
If $Y$ is a subspace of a finite-dimensional linear space $X$, then
\begin{align*}
    {\rm dim}X = {\rm dim}Y + {\rm dim}X/Y.
\end{align*}
\end{theorem}
\begin{proof}
Let $\{x_1, \cdots, x_m\}$ be a basis of $Y$, where $m = {\rm dim}Y$. This set can be completed into a basis for $X$ by adding $x_{m+1}, \cdots, x_n, n = {\rm dim}X$. We claim that $\{x_{m+1}\}, \cdots, \{x_n\}$ form a basis for $X/Y$ by  verifying that they span the whole space $X/Y$ and they are linearly independent as below
\begin{enumerate}[label=(\arabic*)]
    \item For any $x\in X$, we can write it as 
    \begin{align*}
        x = \sum^m_{k=1}a_k x_k + \sum^{n}_{k=m+1}a_k x_k
    \end{align*}
    Then we have 
    \begin{align*}
        \{x\} = \sum^{n}_{k=m+1}a_k \{x_k\}.
    \end{align*}
    \item Suppose that $\sum^{n}_{k=m+1}a_k \{x_k\} = 0$, then we have 
    \begin{align*}
        \sum^{n}_{k=m+1}a_k x_k = y, y\in Y
    \end{align*}
    And $y$ can be expressed as $\sum^{m}_{k=1}a_k x_k$, then we have 
    \begin{align*}
        \sum^{n}_{k=m+1}a_k x_k - \sum^{m}_{k=1}a_k x_k = 0
    \end{align*}
    which implies $a_k = 0$ for all $k$, since $x_1, \cdots, x_n$ form a basis for $X$.
\end{enumerate}
\end{proof}

\begin{corollary}
A subspace $Y$ of a finite-dimensional linear space $X$ whose dimension is the same as the dimension of $X$ is all of $X$.
\end{corollary}
\begin{proof}
Suppose ${\rm dim}X = n$, and a subspace $Y$ of $X$ with dimension $n$. Suppose $y_1,\cdots,y_n$ form a basis for $Y$, then we can complete it into a basis of $X$. If we can find another $x\in X$ that is linearly independent with $y_1,\cdots,y_n$, then we have $\{y_1,\cdots,y_n, x\}$ is the basis of $X$, which is a contradiction. 

Also, we can prove it with ${\rm dim}X/Y = 0$, which implies $X/Y = \{\{0\}\}$.
\end{proof}

\medskip

\begin{theorem}
Suppose $X$ is a finite-dimensional linear space, $U$ and $V$ two subspaces of $X$. Then we have
\begin{align*}
    {\rm dim}(U+V) = {\rm dim}U + {\rm dim}V - {\rm dim}(U\cap V).
\end{align*}
\end{theorem}
\begin{proof}
If $U\cap V = \{0\}$, then $U+V$ is a direct sum and hence 
$${\rm dim}(U+V) = {\rm dim}U + {\rm dim}V $$

In general, let $W = U\cap V$, we claim that $U/W+V/W = (U+V)/W$, which is a direct sum. It suffices to prove that $U/W\cap V/W = \{0\}$. Let $\{x\}\subset U/W\cap V/W$, then $x = u+w_1$ for some $u\in U$ and $w_1\in W$, also, $x = v+w_2$ for some $v\in V$ and $w_2\in W$. Then we have $u+w_1 = v+w_2$, and hence $u+w_1 = v+w_2\in U\cap V = W$. Thus, we have $x\in W$, which gives $\{x\} = \{0\}$.

Now we proved $U/W+V/W = (U+V)/W$, then we have 
\begin{align*}
    {\rm dim}U/W + {\rm dim}V/W & = {\rm dim}(U+V)/W \\
    \Rightarrow {\rm dim}U - {\rm dim} W + {\rm dim} V - {\rm dim}W & = {\rm dim}(U+V) - {\rm dim}W\\ 
    \Rightarrow {\rm dim}U + {\rm dim} V - {\rm dim}(U\cap V) & = {\rm dim}(U+V)
\end{align*}
The proof is complete.
\end{proof}

\medskip

\begin{definition}
The Cartesian sum $X_1\oplus X_2$ of two linear spaces $X_1, X_2$ over the same field is the set of pair $(x_1,x_2)$ where $x_i\in X_i, i = 1,2$. $X_1\oplus X_2$ is a linear space with addition and multiplication by scalars defined componentwisely. 
\end{definition}

\medskip

\begin{theorem}
$${\rm dim}X_1\oplus X_2 = {\rm dim}X_1 + {\rm dim}X_2$$
\end{theorem}
\begin{proof}
Let $x_1, \cdots, x_n$ be a basis of $X_1$ and $y_1, \cdots, y_m$ be a basis of $X_2$. We claim that $(x_1,0),\cdots, (x_n,0), (0,y_1), \cdots, (0,y_n)$ form a basis for $X_1\oplus X_2$ by verifying this is indeed a basis.

Also, we can prove it in another way by defining
\begin{align*}
    Y_1 = \{(x,0): x\in X_1, 0\in X_2\} \\
    Y_2 = \{(0,x): 0\in X_1, x\in X_2\}
\end{align*}
and it is easy to see that $Y_1$ is isomorphic to $X_1$ and $Y_2$ isomorphic to $X_2$. Also, we have $Y_1\cap Y_2 = \{0\}$, then we have 
\begin{align*}
    {\rm dim}X_1\oplus X_2 = {\rm dim}Y_1 + {\rm dim}Y_2 - {\rm dim}X_1(Y_1\cap Y_2) = {\rm dim}X_1 + {\rm dim}X_2
\end{align*}
\end{proof}

Moreover, we can define the Cartesian sum $\oplus^m_{k=1}X_k$ of $m$ linear spaces and we have 
\begin{align*}
    {\rm dim}\oplus^m_{k=1}X_k = \sum^m_{k=1}{\rm dim}X_k.
\end{align*}

\medskip

Next we present an important theorem.

\begin{theorem}
Let $K$ be a field such that it has infinite number of elements and let $X$ be a finite dimensional linear space over $K$. Prove that $X$ cannot be written as a finite union of its proper subspaces.
\end{theorem}
\begin{proof}
Suppose by contrary that there exist $W_1, W_2,\cdots, W_n$, which are proper subspaces of $X$ such that $X = \bigcup^n_{i=1}W_i$. 

If for any $1\leq j\leq n$, $W_j\subset \bigcup^n_{i\neq j}W_i$, then we can remove such $W_j$. Thus, without losing generality, we can assume that no $W_j$ is contained in the union of other $W_i$'s. Note that since $W_i$'s are proper subspaces of $X$, then $X$ must have ${\rm dim}X = n \geq 2$. Since $W_1\not\subset \bigcup^n_{i\neq 1}W_i$, then there exists $u\in W_1$ such that $u\notin W_i, i\geq 2$. Also, $W_1$ is a proper subspace, then there exists $v\notin W_1$. 

Now consider $v + \lambda u$ for $\lambda\in K$. We claim that $v + \lambda u\in W_j$ for at most one $\lambda\in K$. Now we prove this:
\begin{enumerate}[label=(\arabic*)]
    \item Consider the case $j=1$. If $v + \lambda u\in W_1$ for some $\lambda\in K$, then $(v + \lambda u) - \lambda u\in W_1$, since $u\in W_1$ and $W_1$ is a subspace. Thus, we have $v\in W_1$, which is a contradiction.
    \item Now consider the case $j\geq 2$. If there exist $\lambda_1, \lambda_2\in K, \lambda_1\neq\lambda_2$ such that $v+\lambda_1 u\in W_j$ and $v+\lambda_2 u\in W_j$, then $(v+\lambda_1 u) - (v+\lambda_2 u) = (\lambda_1 - \lambda_2)u \in W_j$. Then, since $\lambda_1\neq\lambda_2$, we have $u\in W_j$, which is a contradiction.
\end{enumerate}

This claim implies that there are only finitely many $\lambda\in K$, saying $\lambda_1, \cdots, \lambda_s$ such that 
\begin{align*}
    v+\lambda_i u \in \bigcup^n_{i=1}W_i = X
\end{align*}
Since $K$ has infinitely many elements, then we can choose $\lambda_0\in K$ such that $\lambda_0\notin \{\lambda_1, \cdots, \lambda_s\}$, then $v+\lambda_0 u\notin \bigcup^n_{i=1}W_i = X$, which is a contradiction.
\end{proof}

\medskip

\section{Exercises}
\begin{exercise}\label{ex_1}
Consider a polynomial $X(t):\mathbb{C}\to\mathbb{C}$. Let $V$ be vector space for all complex valued polynomials and let $M = \{X(t): X \,\text{is even}\}$ and $N = \{X(t): X \,\text{is odd}\}$. Prove that 
\begin{enumerate}[label=(\alph*)]
    \item $M,N$ are subspaces of $X$.
    \item $M, N$ are each other's complement in $V$, i.e., $V = M\oplus N$.
\end{enumerate}
\end{exercise}
\begin{proof}
~\begin{enumerate}[label=(\alph*)]
    \item Let $f(t), g(t)\in M$ and $\lambda\in\mathbb{C}$, then we have $f(-t) = f(t)$ and $g(t) = -g(-t)$. Thus, we have $(f+\lambda g)(-t) = f(-t) + \lambda g(-t) = f(t) + \lambda g(t) = (f+\lambda g)(t)$, which implies that $f+\lambda g\in M$. Same argument is similar for $N$.
    \item Let $f(t)\in V$, then we have 
    \begin{align*}
        f(t) & = \frac{f(t) + f(-t)}{2} + \frac{f(t) - f(-t)}{2} \\
        & = f_1(t) + f_2(t)
    \end{align*}
    and it is easy to see that $f_1\in M$ and $f_2\in N$. Also, if $f(t)\in M\cap N$, then we have $f(t) = f(-t)$ and $f(t) = -f(-t)$. Thus, $f(t) = 0$, which implies $M\cap N = \{0\}$. Thus, $V = M\oplus N$.
\end{enumerate}
\end{proof}

\medskip

\begin{exercise}\label{ex_2}
Let $U, V$ and $W$ be subspaces of a finite-dimensional vector space $X$. Is the statement
\begin{align*}
    {\rm dim}(U + V + W) = & {\rm dim} U + {\rm dim} V + {\rm dim} W - {\rm dim}(U \cap V) - {\rm dim}(U \cap W) \\
    & - {\rm dim}(V \cap W) + {\rm dim}(U \cap V \cap W)
\end{align*}
true or false? If true, prove it. If false, provide a counterexample
\end{exercise}
\begin{proof}
The statement is not true. Consider three lines $U,V,W$ in $\mathbb{R}^2$ such that they intersect in one point. So we have
$$\text{dim} (U+V+W) = 2$$
and
\begin{align*}
   \text{dim}(U) + \text{dim}(V) & + \text{dim}(W) - \text{dim}(U \cap V) - \text{dim}(U \cap W) - \\
   & \text{dim}(V \cap W) + \text{dim}(U \cap V \cap W) = 3
\end{align*}
the left and right sides are not the same. 
\end{proof}

\medskip

\begin{exercise}\label{ex_3}
Let $U, V$, and $W$ be subspaces of a finite dimensional linear space $X$. Show that if $W\subset U$, then 
$$U\cup (V + W) = U\cup V + W$$
\end{exercise}
\begin{proof}
~\begin{enumerate}[label=(\arabic*)]
    \item We set $u_1 \in U \cap (V+W)$, then there exist some $v_1 \in V$ and $w_1 \in W$ such that $u_1 = v_1 + w_1$ since $u_1 \in U$ and also $u_1 \in (V+W)$. Also, $u_1 \in U$ and $W \subset U$ which means $w_1 \in U$, then we have $v_1 \in U$ since $U$ is a subspace which is closed under addition. Then from $v_1 \in U$, we have $v_1 \in (U \cap V)$. Based on the fact that $u_1 = v_1 + w_1$ and $w_1 \in W$, we have $u_1 \in (U \cap V + W)$. This implies that $U \cap (V+W) \subset (U \cap V + W)$.
    \item Now we set $u_2 \in (U \cap V + W)$, then there exist some $\lambda \in U \cap V$ and $w_2 \in W$ such that $u_2 = \lambda + w_2$. And we have $\lambda + w_2 \in V+W$, since $\lambda \in U \cap V$ and $w_2 \in W$. Also, we know that $W \subset U$, then we have $\lambda + w_2 \in U$. Thus we can have $\lambda + w_2 \in U \cap (V+W)$. Hence, $u_2 \in U \cap (V+W)$, which implies $(U \cap V + W) \subset U \cap (V+W)$.
\end{enumerate}
\end{proof}

\medskip

\begin{exercise}\label{ex_4}
Denote by $X$ the linear space of all polynomials $p(t)$ of degree less than $n$, and denote by $Y$ the subset of $X$ containing polynomials that are zero at distinct $t_1,t_2,\cdots,t_m\in K$, where $m < n$.
\begin{enumerate}[label=(\roman*)]
    \item Show that $Y$ is a subspace of $X$.
    \item Determine ${\rm dim}Y$ and find a basis of $Y$.
    \item Determine ${\rm dim}X = Y$ and find a basis of $X/Y$.
\end{enumerate}
\end{exercise}
\begin{proof}
~\begin{enumerate}[label=(\roman*)]
    \item Set $P_1, P_2 \in Y$ of form $P_i = (t-t_1)(t-t_2) \cdots (t-t_m)q_i(t)$ that are zero at distinct $t_1, t_2, \cdots, t_m\in K$. Then we have 
    \begin{align*}
        P_1 + P_2 &= \sum^2_{i=1}(t-t_1)(t-t_2) \cdots (t-t_m)q_i(t) \\
        a P_1 &= a (t-t_1)(t-t_2) \cdots (t-t_m)q_1(t)
    \end{align*}
    where $a \in K$. It is easy to see that both $P_1 + P_2$ and $a P_1$ are zero at points $t_1, t_2, \cdots, t_m\in K$. So $Y$ is closed under addition and multiplication. Hence, $Y$ is a subspace of $X$.
    \item In order to being zero at distinct $t_1, t_2, \cdots, t_m\in K$ where $m < n$, the polynomial $P_Y(t) \in Y$ has the form $P_Y(t) = (t-t_1)(t-t_2) \cdots (t-t_m)q(t)$, where $q(t)$ is not determined. Also, we know that the space of all polynomials is degree less than $n$, which means that $q(t)$ is degree less than $n-m$.
    
    Since the polynomials $P(t)$ in the space $X$ are degree less that $n$, it can be presented by the form
    \begin{align*}
        P(t) = \sum^{n-1}_{k=0} c_k t^k
    \end{align*}
    So the basis of $X$ can be written as ${1, t, t^2,\cdots, t^{n-1}}$, and we have dim $X = n$. Now we can present $q(t)$ by utilizing this basis as 
    \begin{align*}
        q(t) = \sum^{n-m-1}_{k=0} c_k t^k
    \end{align*}
    then the basis for subspace $Y$ can be presented as
    \begin{align*}
        \left\{ \prod^m_{i=1}(t-t_i), t\prod^m_{i=1}(t-t_i),\cdots, t^{n-m-1}\prod^m_{i=1}(t-t_i) \right\}
    \end{align*}
    and we can check that the linear combination of this basis is equal to zero if and only if all coefficients are all zero. So we have dim $Y = n-m$.
    \item We have dim $X/Y =$ dim $X -$ dim $Y = m$. Now we set a basis that spans the subspace $X/Y$.
    
    We firstly set $P_1(t) = (t-t_2)\cdots(t-t_m)q(t)$ and the class $\{P_1\}$ of $P_1$ is the space $\left\{P(t)\in X: P(t) - P_1(t) \in Y \right\}$, and then set $P_2(t) = (t-t_1)(t-t_3)\cdots(t-t_m)q(t)$ and the class $\{P_2 \}$ in the same way. And we continue this process where we get rid of $(t-t_i)$ in class $P_i$ until we finally have $P_m(t) = (t-t_1)\cdots(t-t_{m-1})q(t)$ and the class $\{P_m \}$. Then we can check that $( \{P_1\}, \{P_2\}, \cdots, \{P_m\})$ is the span of $X/Y$.
    
    Better solution for (iii): From (ii) we know the basis of $Y$, and we can know that every $p(t)$ in $X$ can be replaced by a polynomial of degree less than $m$ in $x/Y$, so $1, t, \cdots, t^{m-1}$ form a basis of $X/Y$.
\end{enumerate}
\end{proof}

\medskip

\begin{exercise}\label{ex_5}
Let $U_1,U_2,\cdots,U_k$ be subspaces of a finite-dimensional linear space $X$ such that
\begin{align*}
    {\rm dim}\,U_1 = {\rm dim}\,U_2 = \cdots = {\rm dim}\,U_k
\end{align*}
Then there is a subspace $V$ of $X$ for which
\begin{align*}
    X = U_1\oplus V = U_2\oplus V = \cdots = U_k\oplus V
\end{align*}
\end{exercise}
\begin{proof}
If ${\rm dim}\, U_1 = \cdots = {\rm dim}\, U_k = n$, then we can simply take $V = \varnothing$.

If ${\rm dim}\, U_1 = \cdots = {\rm dim}\, U_k = n-1$, based on the fact that space $X$ cannot be a finite union of its proper subspace, there exists a $v \notin U_k$ for every $k$, and we can define the complement of $U_1, U_2, \cdots, U_k$ as $U^c = \text{span}(v)$.

Then we consider the case when ${\rm dim}\, U_1 = \cdots = {\rm dim}\, U_k = n-2$, also there exists a $v \notin U_k$ for every $k$. Then we can define a new subspace $\title{\Tilde{U}_i} = \text{span}(u_i,v)$ for $1 \leq i \leq k$, and we can immediately know that ${\rm dim}\, \Tilde{U}_i = n - 1$. Then we can get a complement of $\Tilde{U}_i$, denoted by $U_i^c$ and we have ${\rm dim}\, U_i^c = 1$. In particular, $v \notin U_i^c$, so we can take $\text{span} (U_i^c, v)$, which is dimension $2$ and a complement of $U_1, U_2, \cdots, U_k$.

Now we can continue this induction and consider ${\rm dim}\, U_1 = \cdots = {\rm dim}\, U_k = m, m < n$, and we can find $v \notin U_k$ for every $k$. Then we can define $\title{\Tilde{U}_i} = \text{span}(u_i,v)$ for $1 \leq i \leq k$, and we can immediately know that ${\rm dim}\, \Tilde{U}_i = m + 1$. Then we can get a complement of $\Tilde{U}_i$, denoted by $U_i^c$ and we have ${\rm dim}\, U_i^c = n - m - 1$. In particular, $v \notin U_i^c$, so we can take $\text{span} (U_i^c, v)$, which is dimension $n - m$ and a complement of $U_1, U_2, \cdots, U_k$.
\end{proof}
Exercises (\ref{ex_1}) to (\ref{ex_5}) are Homework 1 for MATH2370. Next we present some exercises from the book Challenging Problems for Students by Fuzhen Zhang\cite{1} and other books.

\medskip

\begin{exercise}\label{ex_6}
Let $\mathbb{C},\mathbb{R}$, and $\mathbb{Q}$ be the fields of complex, real, and rational numbers, respectively. Determine whether each of the following is a vector space. Find the dimension and a basis for each that is a vector space.
\begin{enumerate}[label=(\alph*)]
    \item $\mathbb{C}$ over $\mathbb{C}$. Yes, the dimension is $1$, with a basis $\{1\}$.
    \item $\mathbb{C}$ over $\mathbb{R}$. Yes, the dimension is $2$, with a basis $\{1, i\}$.
    \item $\mathbb{R}$ over $\mathbb{C}$. No, since $i\in\mathbb{C}$, and $1\cdot i = i\notin\mathbb{R}$.
    \item $\mathbb{R}$ over $\mathbb{Q}$. Yes, the dimension is infinite, since $1,\pi,\pi^2,\cdots$ are linearly independent over $\mathbb{Q}$.
    \item $\mathbb{Q}$ over $\mathbb{R}$. No, since $\pi\in\mathbb{R}$, and $\pi\cdot 1 = \pi\notin\mathbb{Q}$.
    \item $\mathbb{Q}$ over $\mathbb{Z}$. No, since $\mathbb{Z}$ is not a field.
    \item $\mathbb{S} = \{a+\sqrt{2}b+\sqrt{5}c|\, a,b,c\in\mathbb{Q}\}$ over $\mathbb{Q},\mathbb{R}$ or $\mathbb{C}$. 
    
    Yes over $\mathbb{Q}$, and the dimension is $3$, with a basis $\{1,\sqrt{2},\sqrt{5}\}$. 
    
    No over $\mathbb{R}$, since $1+\sqrt{2}+\sqrt{5}\in\mathbb{S}, \sqrt{10}\in\mathbb{R}$, and $(1+\sqrt{2}+\sqrt{5})\cdot \sqrt{10}\notin\mathbb{S}$.
    
    No over $\mathbb{C}$, with the similar argument.
\end{enumerate}
\end{exercise}




\chapter{Duality}
\section{Linear Functions and Dual Space}

Let $X$ be a linear space over a field $K$. A scalar valued function $l:X\to K$ is called \emph{linear} if 
\begin{align*}
    l(x+y) &= l(x) + l(y) \\
    l(kx) &= k l(x)
\end{align*}
for all $x,y\in X$, and for all $k\in K$.

The set of linear functions on a linear space $X$ forms a linear space $X'$, the \emph{dual space} of $X$, if we define 
\begin{align*}
    (l+m)(x) &= l(x) + m(x) \\
    (kl)(x) &= k(l(x))
\end{align*}

\medskip

\begin{theorem}\label{theorem_dual}
Let $X$ be a linear space of dimension $n$. Under a chosen basis $x_1,\cdots,x_n$, the elements of $X$ can be represented as arrays of $n$ scalars:
$$x = (c_1,\cdots,c_n) = \sum^n_{k=1}c_k x_k$$
Let $a_1,\cdots,a_n$ be any array of $n$ scalars, the function $l$ defined by
\begin{align*}
    l(x) = \sum^n_{k=1} a_k c_k
\end{align*}
is a linear function of $X$. Conversely, every linear function $l$ of $X$ can be so represented.
\end{theorem}
\begin{proof}
For any $l\in X'$, define $a_k = l(x_k)$, then we have 
\begin{align*}
    l(x) = l\left(\sum^n_{k=1}c_k x_k \right) = \sum^n_{k=1}c_k l(x_k) = \sum^n_{k=1}c_k a_k.
\end{align*}
\end{proof}

\medskip

\begin{theorem}
${\rm dim}X = {\rm dim}X'$.
\end{theorem}
\begin{proof}
Suppose ${\rm dim}X = n$. Define $l_j(x) = c_j$, for $x\in X$. We claim $l_j, 1\leq j\leq n$ form a basis of $X'$. Indeed, we have
\begin{enumerate}[label=(\arabic*)]
    \item For any $l\in X'$, we have $l(x) = \sum^n_{k=1}c_k a_k = \sum^n_{k=1}a_k l_k(x)$. Thus, $l = \sum^n_{k=1}a_k l_k$, which implies that $\{l_j, 1\leq j\leq n\}$ span the space $X'$.
    \item We claim $l_j, 1\leq j\leq n$ are linearly independent. If $\sum^n_{k=1}b_k l_k = 0$, then we have 
    \begin{align*}
        \sum^n_{k=1}b_k l_k(x_k) = \sum^n_{k=1}b_k c_k = 0
    \end{align*}
    for all $x_k\in X$. Then we have $b_k = 0, 1\leq k \leq n$.
\end{enumerate}
\end{proof}

\medskip

We defined $l(x) = \sum^n_{k=1} a_k c_k$ in theorem (\ref{theorem_dual}), the right-hand side depends symmetrically on $l$ and $x$, then we can write left-hand side also symmetrically, we introduce the \emph{scalar product} notation
\begin{align*}
    (l,x) \equiv l(x)
\end{align*}
which is a bilinear function of $l$ and $x$. 

The dual of $X'$ is $X''$, consisting of all linear functions on $X'$. Also, $(l,x)$ defines an element in $X''$.

\begin{theorem}
$(l,x)$ is a bilinear form, which gives a natural identification of $X$ with $X''$. The map $\varphi:X\ni x \to x^{**}\in X''$ is an isomorphism, where $(x^{**},l) = (l,x)$ or any $l\in X'$.
\end{theorem}
\begin{proof}
$\varphi(x)$ is a subspace of $X''$.
\begin{enumerate}[label=(\arabic*)]
    \item If $\varphi(x_1) = \varphi(x_2)$, then $(l,x_1) = (l,x_2)$ for any $l\in X'$. Then we have $(l,x_1 - x_2) = 0$. Now we can set $x_1 - x_2 = (c_1,\cdots,c_n)$, and pick $l = (\overline{c_1},\cdots,\overline{c_n})$, then we have $\sum^n_{k=1}|c_k|^2 = 0$. Then, $c_k = 0$, which implies $x_1 = x_2$. Thus, $\varphi$ is noe-to-one.
    \item We claim that $\varphi(x) = X''$. It suffices to prove that ${\rm dim}\varphi(x) = {\rm dim}X''$.
    
    Let $x_1,\cdots,x_n$ be a basis of $X$, then $x_1^{**},\cdots,x_n^{**}$ is a basis of $X''$. Thus, $\varphi$ is onto.
\end{enumerate}
\end{proof}

\medskip

\section{Annihilator and Codimension}
\begin{definition}
Let $Y$ be a subspace of $X$. The set of linear functions that vanish on $Y$, that is, satisfying $l(y) = 0$ for all $y\in Y$ is called the annihilator of the subspace $Y$, denoted by $Y^\bot$.
\end{definition}
\begin{remark}
$Y^\bot$ is a subspace of $X'$.
\end{remark}

\medskip

\begin{theorem}
${\rm dim}Y^\bot + {\rm dim}Y = {\rm dim}X$.
\end{theorem}
\begin{proof}
We can establish a natural isomorphism $T:Y^\bot\to (X/Y)'$ as follows $$T(l)(\{x\}) = l(x)$$
for any $l\in Y^\bot$ and $\{x\}\in X/Y$. It suffices to prove that $T$ is well-defined.
\begin{enumerate}[label=(\arabic*)]
    \item If $\{x_1\} = \{x_2\}$, then $x_1 = x_2 + y$ for some $y\in Y$. Then $l(x_1) = l(x_2) + l(y) = l(x_2)$. Then $T(l)$ is well defined.
    \item Also, $T(l)$ is linear. Indeed, for $l_1, l_2\in Y^\bot$ and $a,b \in K$, we have $T(al_1 + bl_2)(\{x\}) = (al_1 + bl_2)(x) = a l_1(x) + bl_2(x) = a T(l_1)(\{x\}) + b T(l_2)(\{x\})$.
    \item $T(l)$ is an isomorphism. 
    \begin{enumerate}
        \item $T$ is one-to-one. Indeed, if $T(l) = 0$, then $T(l)(\{x\}) = l(x) = 0$, for all $x\in X$. Then we have $l = 0$.
        \item $T$ is onto. For $\forall \Tilde{l}\in (X/Y)'$, define $l\in X'$ such that $l(x) = \Tilde{l}(\{x\})$. If $x\in Y$, then $l(x) = \Tilde{l}(\{0\}) = 0$, it follows $l\in Y^\bot$. Thus, $T(l) = \Tilde{l}$ is onto.
    \end{enumerate}
\end{enumerate}

Thus, we have ${\rm dim}Y^\bot = {\rm dim}(X/Y)' = {\rm dim}(X/Y)$ and hence 
\begin{align*}
    {\rm dim}Y^\bot = {\rm dim}(X/Y) = {\rm dim}X - {\rm dim}Y.
\end{align*}
\end{proof}

\medskip

The dimension of $Y^\bot$ is called the \emph{codimension} of $Y$ as subspace of $X$. And since $Y^\bot$ is a subspace of $X'$, its annihilator denoted by $Y^{\bot \bot}$ is a subspace of $X''$.

\medskip

\begin{theorem}
Under the natural identification of $X''$ and $X$, for every subspace $Y$ of a finite-dimensional space $X$, $Y^{\bot\bot} = Y$.
\end{theorem}
\begin{proof}
For any $y\in Y$ and $l\in Y^\bot$, $y^{**}(l) = l(y) = 0$, where $y^{**} \in Y^{\bot\bot}$. Thus we have $Y\subset Y^{\bot\bot}$. Also, ${\rm dim}Y^{\bot\bot} = {\rm dim}X' - {\rm dim}Y^\bot = {\rm dim}X - {\rm dim}Y^\bot = {\rm dim}Y$. Thus, $Y = Y^{\bot\bot}$.
\end{proof}

\medskip

\begin{definition}
Let $X$ be a finite-dimensional linear space, and let $S$ be a subset of $X$. The annihilator $S^\bot$ of $S$ is the set of linear functions $l$ that are zero at all vectors $s\in S$, that is, $l(s) = 0$. 
\end{definition}

\medskip

\begin{theorem}
Denote by $Y$ the smallest subspace containing $S$, then $S^\bot = Y^\bot$.
\end{theorem}
\begin{proof}
~\begin{enumerate}[label=(\arabic*)]
    \item Since $S\subset Y$, then $Y^\bot\subset S^\bot$. Indeed, if $l\in Y^\bot$, then $l(y) = 0$ for all $y\in Y$. Since $S\subset Y$, then for $\forall s\in S$, we have $l(s) = 0$. Thus, $Y^\bot\subset S^\bot$. 
    \item Now we prove $S^\bot\subset Y^\bot$. Suppose $x_1,\cdots,x_j$ be the basis of $S$, then ${\rm span}\{x_1,\cdots,x_j\} = Y$. Then for any $y\in Y$, it can be written as $y = \sum^j_{k=1}c_k x_k$. For $l\in S^\bot$, we have $l(x_k) = 0, 1\leq k \leq j$, then $l(y) = \sum^j_{k=1}c_k l(x_k) = 0$. Thus, $l\in Y^\bot$, which implies $S^\bot\subset Y^\bot$. 
\end{enumerate}
In another words, $S^\bot = ({\rm span}S)^\bot$.
\end{proof}

\medskip

\section{Quadrature Formula}
\begin{theorem}
Let $I$ be an interval on the real axis, $t_1, \cdots, t_n$ are $n$ distinct points. Then there exist $n$ numbers $m_1,\cdots,m_n$ such that the quadrature formula
\begin{align*}
    \int_I p(t) {\rm d}t = m_1 p(t) + \cdots + m_n p(t_n)
\end{align*}
holds for all polynomials $p$ of degree less than $n$.
\end{theorem}
\begin{proof}
Denote by $X$ the space of all polynomials $P(t) = a_0 + a_1 t +\cdots+ a_{n-1}t^{n-1}$ of degree less than $n$. Since $X$ is isomorphic to the space $\mathbb{R}^n = (a_0,a_1,\cdots,a_{n-1})$, then ${\rm dim}X = n$. We define $l_j\in X'$ as the linear function 
\begin{align*}
    l_j(p) = p(t_j)
\end{align*}

We claim that $l_j, 1\leq j \leq n$ are linearly independent. Indeed, assume $\sum^n_{k=1}c_k l_k = 0$, then we have 
$$\sum^n_{k=1}c_k l_k(p) = \sum^n_{k=1}c_k p(t_k) = 0$$
and for $k$, pick $p(t) = \prod_{j\neq k}(t - t_j)$, then we have $c_k = 0, 1\leq k\leq n$. Then $\{l_j\}^n_{j=1}$ form a basis of $X'$, since ${\rm dim}X' = {\rm dim}X = n$. Then any linear function $l$ on $X$ can be represented as below
$$l = m_1 l_1 + \cdots + m_n l_n.$$

The integral of $p$ over $I$ is a linear function, therefore it can be represented as above.
\end{proof}

\medskip

\section{Exercises}
\begin{exercise}
Suppose $\{x_1, x_2,\cdots, x_n\}$ is a basis for the vector space $X$. Show that there exists linear functions $\{e_1, e_2,\cdots, e_n\}$ in the dual space $X'$ satisfying 
\begin{align*}
    e_i(x_j) = \delta_{ij}
\end{align*}
Show that $\{e_1, e_2,\cdots, e_n\}$ is a basis of $X'$, called the dual basis.
\end{exercise}
\begin{proof}
~\begin{enumerate}[label=(\arabic*)]
    \item First, we check that $\{e_1, e_2, \cdots, e_n\}$ are linearly independent. Suppose that there exist  $a_1, a_2, \cdots, a_n \in K$ such that 
    \begin{align*}
        a_1 e_1 + a_2 e_2 + \cdots + a_n e_n = 0
    \end{align*}
    then for $\forall x_i \in X$, we have
    \begin{align*}
        (a_1 e_1 + a_2 e_2 + \cdots + a_n e_n)(x_i) = a_1 e_1(x_i) + \cdots + a_n e_n(x_i) = a_i = 0
    \end{align*}
    Thus, $a_i = 0$, for $\forall a_i$, which means $\{e_1, e_2, \cdots, e_n\}$ are linearly independent.
    \item Then, we need to show that span$(e_1, e_2, \cdots, e_n) = X'$. For any $f \in X'$, let $b_i  = f(x_i)$, and $f = b_1 e_1 + b_2 e_2 + \cdots + b_n e_n$. Then, for $\forall x_i$, we have
    \begin{align*}
        f(x_i) = (b_1 e_1 + b_2 e_2 + \cdots + b_n e_n)(x_i) = b_i
    \end{align*}
    Thus, $f$ can be presented by $\{e_1, e_2, \cdots, e_n\}$. The proof is complete.
\end{enumerate}
\end{proof}

\medskip

\begin{exercise}
Let $X$ be a finite dimensional linear space. Show that two nonzero linear functionals $T,S\in X'$ have the same null space if and only if there is a nonzero scalar $\lambda$ such that $S = \lambda T$.
\end{exercise}
\begin{proof}
~\begin{enumerate}[label=(\arabic*)]
    \item Since $T: X \rightarrow R$, then there $\exist x_1 \in X$ such that $T x \neq 0$. And we let $x_2  = \frac{x_1}{T(x_1)}$, then we have $T(x_2) = 1$. And for $\forall x \in X$we can know
    \begin{align*}
        T(x - T(x)\cdot x_2) = T(x) - T(x) = 0
    \end{align*}
    Since $T$ and $S$ have the same null space, then we have $S(x - T(x)\cdot x_2) = 0$, then 
    \begin{align*}
        S(x) & = S(x - T(x)\cdot x_2 + T(x)\cdot x_2) \\
        & = S(x - T(x)\cdot x_2) + S(T(x)\cdot x_2) \\
        & = 0 + S(x_2) T(x)
    \end{align*}
    \item Let $\lambda = S(x_2)$, then we proved that $S = \lambda T$.
    If $S = \lambda T$, then for $\forall x \in N_P$, we have $T(x) = \frac{1}{\lambda} S(x) = 0$, which means $N_P \subset N_T$. And for $\forall x \in N_T$, $S(x) = 0$, which means $N_T \subset N_P$. So $N_T = N_P$. The proof is complete.
\end{enumerate}
\end{proof}

\medskip

\chapter{Linear Mappings}
\section{Null-space and Range}

Let $X,U$ be linear spaces over the same field $K$. A mapping $T:X\to Y$ is called linear if it is additive and homogeneous, i.e.,
\begin{align*}
    T(x+y) &= T(x)+T(y), \forall x,y \in X\\
    T(k x) &= k T(x), \forall x\in X, \forall k\in K
\end{align*}
For simplicity, we write $T(x) = T x$.

\medskip

\begin{example}[Examples of Linear Mappings]
~\begin{enumerate}[label=(\arabic*)]
    \item Any isomorphism.
    \item Differentiation from polynomial $P_n(t)$ to $P_{n-1}(t)$.
    \item Linear functionals.
    \item $X=\mathbb{R}^n, U=\mathbb{R}^m, u = TX$ defined by
    \begin{align*}
        u_i = \sum^n_{j=1}t_{ij}x_j, i = 1,2,\cdots,m
    \end{align*}
    Hence, $u = (u_1,\cdots,u_m), x = (x_1,\cdots,x_n)$.
\end{enumerate}
\end{example}

\medskip

\begin{theorem}
~\begin{enumerate}[label=(\arabic*)]
    \item The image of a subspace of $X$ under a linear map $T$ is a subspace of $U$.
    \item The inverse image of a subspace of $U$, that is the set of all vectors in $X$ mapped by $T$ into the subspace, is the subspace of $X$.
\end{enumerate}
\end{theorem}
\begin{proof}
It follows from the definition of subspace.
\end{proof}

\medskip

\begin{definition}
The range of $T$ is the image of $X$ under $T$, denoted by $R_T$. The null-space of $T$ is the inverse image of $\{0\}$, denoted by $N_T$.
\end{definition}
\begin{remark}
If $T:X\to U$, then $R_T\subset U, N_T\subset U$ are subspaces of $U$.
\end{remark}

\begin{definition}
${\rm dim}\,R_T$ is called the rank of the mapping $T$ and ${\rm dim}\,N_T$ is called the nullity of the mapping $T$.
\end{definition}

\medskip

\section{Rank-Nullity Theorem}
\begin{theorem}[Rank-Nullity Theorem]
Let $T:X\to U$ be a linear map. Then $${\rm dim}\,R_T + {\rm dim}\,N_T = {\rm dim}\, X.$$
\end{theorem}
\begin{proof}
We can define $\Tilde{T}:X/N_T\to R_T$ as $\Tilde{T}(\{x\}) = Tx \in R_T$, for $\forall x\in X$. We claim that $\Tilde{T}$ is an isomorphism. Indeed, if $\{x\} = \{y\}$, then $x-y\in N_T$, then we have $T(x-y) = 0$, which implies $Tx = Ty$. Thus, $\Tilde{T}(\{x\}) = \Tilde{T}(\{y\})$. Also, $\Tilde{T}$ is linear, since $\Tilde{T}(a\{x\} + b\{y\}) = a\Tilde{T}(\{x\}) + b\Tilde{T}(\{y\})$. 

Thus, we have ${\rm dim}\, X/N_T = {\rm dim}\, R_T$. With theorem (\ref{theorem_quotient_dim}), we have ${\rm dim}\,X - {\rm dim}\,N_T = {\rm dim}\, R_T$.
\end{proof}

\medskip

\begin{theorem}
Let $T:X\to U$ be a linear map, then
\begin{enumerate}[label=(\alph*)]
    \item Suppose ${\rm dim}\,U < {\rm dim}\, X$, then there exists $x\neq 0$, such that $T x = 0$.
    \item Suppose ${\rm dim}\,U = {\rm dim}\, X$, the only vector satisfying $T x = 0$ is $x= 0$. Then $R_T = U$ and $T$ is an isomorphism.
\end{enumerate}
\end{theorem}
\begin{proof}
~\begin{enumerate}[label=(\alph*)]
    \item Since ${\rm dim}\,R_T \leq {\rm dim}\, U < {\rm dim}\, X$, then we have ${\rm dim}\, N_T = {\rm dim}\,X - {\rm dim}\,R_T > 0$. Then there exists $x \neq 0, x\in N_T$ such that $T x = 0$.
    \item Since ${\rm dim}\,U = {\rm dim}\, X$, we have ${\rm dim}\,N_T = 0$. Then we have ${\rm dim}\,R_T = {\rm dim}\,U$. Thus $R_T = U$ and $T$ is an isomorphism.
\end{enumerate}
\end{proof}

\medskip

\section{Injectivity and Surjectivity}
\begin{definition}
A linear mapping $T:X\to U$ is called injective (or one-to-one) if $Tu = Tv$ implies $u = v$.
\end{definition}

\medskip

\begin{theorem}
Injectivity is equivalent to null space equals $\{0\}$, i.e., if $T:X\to U$, then $T$ is injective if and only if $N_T = \{0\}$.
\end{theorem}
\begin{proof}
~\begin{enumerate}[label=(\arabic*)]
    \item ($\Rightarrow$) Suppose $T$ is injective, and we need to prove that $N_T = \{0\}$. We already know that $\{0\}\subset N_T$. 
    
    Let $v\in N_T$, then we have $Tv = 0 = T(0)$. Since $T$ is injective, then we have $v = 0$. Thus, $N_T = \{0\}$.
    \item ($\Leftarrow$) Suppose $N_T = \{0\}$. Let $u,v\in X$ such that $Tu = Tv$. Then we have $Tu - Tv = T(u - v) = 0$, which implies $u = v$. Thus, $T$ is injective.
\end{enumerate}
\end{proof}

\medskip

\begin{definition}
A linear mapping $T:X\to U$ is called surjective (or onto) if its range equals $U$, i.e., $R_T = U$.
\end{definition}

\medskip

\begin{theorem}
Suppose $X$ and $U$ are finite-dimensional vector spaces such that $\dim X > \dim U$, then no linear map from $X$ to $U$ is injective.
\end{theorem}
\begin{proof}
Let $T:\mathcal{L}(X,U)$, then with Rank-Nullity theorem, we have
\begin{align*}
    \dim N_T & = \dim X - \dim R_T \\
    & \geq \dim X - \dim U \\
    & > 0
\end{align*}
Thus, $T$ is not injective.
\end{proof}

\medskip

\begin{theorem}
Suppose $X$ and $U$ are finite-dimensional vector spaces such that $\dim X < \dim U$, then no linear map from $X$ to $U$ is surjective.
\end{theorem}
\begin{proof}
Let $T:\mathcal{L}(X,U)$, then with Rank-Nullity theorem, we have
\begin{align*}
    \dim R_T & = \dim X - \dim R_T \\
    & \leq \dim X \\
    & < \dim U
\end{align*}
Thus, $T$ is not surjective.
\end{proof}

\medskip

\section{Underdetermined Linear Systems}
\begin{theorem}
Suppose $m < n$, then for any real numbers $t_{ij}, 1\leq i\leq m, 1\leq i\leq n$, the system of linear equations
\begin{align*}
    \sum^n_{j=1}t_{ij} x_j = 0, 1\leq i \leq m
\end{align*}
has a nontrivial solution.
\end{theorem}
\begin{proof}
Define $T:\mathbb{R}^n\to\mathbb{R}^m$ as
\begin{align*}
    T(x_1,\cdots,x_n) = \left(\sum^n_{j=1}t_{1j} x_j, \cdots, \sum^n_{j=1}t_{nj} x_j\right)
\end{align*}
Then $T$ is linear, and with previous theorem, there exists $x\in\mathbb{R}^n, x\neq 0$ such that $T x = 0$. Thus, $x = (x_1,\cdots,x_n)$ is an nontrivial solution.
\end{proof}

\medskip

\begin{theorem}
Given $n^2$ real numbers $t_{ij}, 1\leq i,j\leq n$, the inhomogeneous system of linear equations
\begin{align*}
    \sum^n_{j=1}t_{ij} x_j = u_i, 1\leq i \leq n
\end{align*}
has a unique solution for any $u_i, 1\leq i \leq n$ if and only if the homogeneous system
\begin{align*}
    \sum^n_{j=1}t_{ij} x_j = 0, 1\leq i \leq n
\end{align*}
has only the trivial solution.
\end{theorem}
\begin{proof}
~\begin{enumerate}[label=(\arabic*)]
    \item ($\Rightarrow$) Set $u_i = 0$ and it is trivial.
    \item ($\Leftarrow$) Define $T:\mathbb{R}^n\to\mathbb{R}^n$ as 
    $$T(x_1,\cdots,x_n) = \left(\sum^n_{j=1}t_{1j} x_j, \cdots, \sum^n_{j=1}t_{nj} x_j\right)$$
    If homogeneous system has only trivial solution, then $N_T = \{0\}$, which implies $R_T = \mathbb{R}^n$. Thus, $T$ is an isomoorphism. 
\end{enumerate}
\end{proof}

\medskip

\section{Algebra of Linear Mappings}
Let $X,U$ be linear spaces and let $\mathscr{L}(X,U)$ be the collection of all linear maps from $X$ to $U$. $\mathscr{L}(X,U)$ is a linear space if we define
\begin{align*}
    (T+S)(x) & = Tx + Sx \\
    (kT)(x) & = k T x
\end{align*}
for $\forall x\in X, \forall k\in K, \forall T,S\in \mathscr{L}(X,U)$.

\medskip

\begin{definition}
Let $T\in \mathscr{L}(X,U)$ and $S\in \mathscr{L}(U,V)$, where $X,U$ and $V$ are linear spaces. The composition of $S$ and $T$ is defined by 
\begin{align*}
    S \circ T(x) = S\left(T(x)\right)
\end{align*}
denoted by $ST$, called the multipilication of $S$ and $T$. In general, $ST\neq TS$.
\end{definition}
\begin{remark}
~\begin{enumerate}[label=(\arabic*)]
    \item $S\circ T \in \mathscr{L}(X,V)$.
    \item The composition is associative, i.e., if $R\in \mathscr{L}(V,Z)$, then $R\circ (S\circ T) = (R\circ S)\circ T$.
    \item The composition is distributive, i.e., if $T\in \mathscr{L}(X,U)$ and $R,S \in \mathscr{L}(U,V)$, then $(R+S)\circ T = R\circ T + S\circ T$.
\end{enumerate}
\end{remark}

\medskip

\begin{definition}
A linear map is called invertible if it is one-to-one and onto, that is, if it is isomorphism, denoted by $T^{-1}$.
\end{definition}

\medskip

\begin{theorem}
~\begin{enumerate}[label=(\arabic*)]
    \item The inverse of invertible map is linear.
    \item If $S$ and $T$ are both invertible, then $ST$ is also invertible and $(ST)^{-1} = T^{-1} S^{-1}$.
\end{enumerate}
\end{theorem}
\begin{proof}
~\begin{enumerate}[label=(\arabic*)]
    \item Let $T\in \mathscr{L}(X,U)$ be invertible, it suffices to prove that 
    \begin{align*}
        T^{-1}(k_1 u_1+k_2 u_2) = k_1 T^{-1}(u_1) + k_2 T^{-1}(u_2)
    \end{align*}
    for all $k_1,k_2\in K$ and $u_1,u_2\in U$. We have
    \begin{align*}
        T \left(T^{-1}(k_1 u_1+k_2 u_2)\right) & = k_1 TT^{-1}(u_1) + k_2 TT^{-1}(u_2) \\
        & = k_1 u_1 + k_2 u_2
    \end{align*}
    which implies the above indication.
    \item Let $T:U\to V, S:V\to W$ and then $ST:U\to W$, then $ST$ is also an isomorphism, which implies it is invertible. For any $w\in W$, there exists a $u\in U$ such that $(ST)^{-1}(w) = u$. It suffices to prove that $T^{-1} S^{-1}(w) = u$.
    
    If $T^{-1} S^{-1}(w)\neq u$, then there is another $u'\in U$ such that $T^{-1} S^{-1}(w) = u'$. Since $S$ is isomorphism, then there exists only one element in $V$, saying $v$ such that $S^{-1}(w) = v$ and we have $T^{-1}(v) = u$ and also $T^{-1}(v) = u'$, which is a contradiction.
\end{enumerate}
\end{proof}

\medskip

\section{Transposition}
\begin{definition}
Let $T\in L(X,U)$ the transpose $T'\in \mathscr{L}(U',X')$ of $T$ is defined by 
\begin{align*}
    \left(T'(l)\right)(x) = l(T(x))
\end{align*}
for any $l\in U'$ and $x\in X$. We could use the dual notation to represent the identity as $(T' l,x) = (l, Tx)$.
\end{definition}

\medskip

\begin{theorem}
~\begin{enumerate}[label=(\arabic*)]
    \item $(ST)' = T'S'$.
    \item $(T+R)' = T' + R'$.
    \item $(T^{-1})' = (T')^{-1}$.
\end{enumerate}
\end{theorem}
\begin{proof}
~\begin{enumerate}[label=(\arabic*)]
    \item Let $T:X\to U, S:U\to V$. Then we have 
    \begin{align*}
        \left((ST)'l, x\right) = (l, ST x) = (S' l, T x) = (T'S' l, x).
    \end{align*}
    \item It is obvious.
    \item Let $T\in \mathscr{L}(X,U)$ be invertible. And we assume $I_U = T\circ T^{-1}:U\to U$. We claim $(T\circ T^{-1})' = (I_U)'$ is an identity of $U'$. Indeed, $(T^{-1})'\circ T' = I_{U'}$, then we have $(T')^{-1} = (T^{-1})'$. We need to prove that $(I_U)' = I_{U'}$. Indeed, we have $\left((I_U)'l,u\right) = (l, I_U u) = (l, u)$, which implies $(I_U)'l = l$. Thus, we have $(I_U)' = I_{U'}$.
\end{enumerate}
\end{proof}

\medskip

\begin{example}
Let $X=\mathbb{R}^n, U=\mathbb{R}^m$ and $T:X\to U$ is defined by $y = T x$, where 
\begin{align*}
    y_i = \sum^n_{j=1}t_{ij}x_j, 1\leq i \leq m.
\end{align*}
Identifying $(\mathbb{R}^n)'=\mathbb{R}^n, \mathbb{R}^m=\mathbb{R}^m$, then $T':\mathbb{R}^m\to \mathbb{R}^n$ is defined by $v = T' u$, where 
\begin{align*}
    v_j = \sum^m_{i=1} t_{ij} u_i, 1\leq j \leq m.
\end{align*}
\end{example}

\medskip

\begin{theorem}
Let $T\in L(X,U)$. Identifying $X''=X, U''=U$, then $T''=T$.
\end{theorem}
\begin{proof}
We have $T'\in \mathscr{L}(U',X'), T''\in \mathscr{L}(X'',U'') = L(X,U)$. Now we pick $X'' \ni x^{**} = x\in X$, then we have
\begin{align*}
    \left(T''x^{**},l\right) = \left(x^{**},T'l\right) = (T' l, x) = (l, T x) = \left((T x)^{**},l\right)
\end{align*}
Thus, $T'' = T$.
\end{proof}

\section{Dimension of Null-space and Range}
\begin{theorem}
Let $T\in \mathscr{L}(X,U)$, then we have $R_T^\bot = N_{T'}$ and $R_T = N_{T'}^\bot$.
\end{theorem}
\begin{proof}
~\begin{enumerate}[label=(\arabic*)]
    \item For $l\in R_T^\bot$, then for any $u\in R_T$, $(l,u) = 0$. Since $u\in R_T$, then $u = T x$ for some $x\in X$. Thus, we have
    \begin{align*}
        (l, T x) = 0 \Rightarrow (T' l, x) = 0
    \end{align*}
    for $\forall x\in X$. Then $T' l = 0$, which implies $l\in N_{T'}$.
    \item Since $R_T^\bot = N_{T'}$, then we have $R_T^{\bot\bot} = N_{T'}^\bot = R_T$.
\end{enumerate}
\end{proof}

\medskip

\begin{theorem}
Let $T\in \mathscr{L}(X,U)$, then ${\rm dim}R_T = {\rm dim}R_{T'}$.
\end{theorem}
\begin{proof}
First, we have ${\rm dim}R_T + {\rm dim}R_T^\bot = {\rm dim}U$, then we have $${\rm dim}R_T + {\rm dim}N_{T'} = {\rm dim}U$$
With Rank-Nullity theorem, we have $${\rm dim}R_{T'} + {\rm dim}N_{T'} = {\rm dim}U' = {\rm dim}U$$ 
Thus, ${\rm dim}R_{T'} = {\rm dim}R_{T}$.
\end{proof}

\medskip

\begin{corollary}
Suppose $T\in \mathscr{L}(X,U)$ and ${\rm dim} X = {\rm dim} U$. Then, ${\rm dim}N_T = {\rm dim}N_{T'}$.
\end{corollary}
\begin{proof}
From Rank-Nullity theorem, we have 
\begin{align*}
    {\rm dim}R_T + {\rm dim}N_T & = {\rm dim}U \\
    {\rm dim}R_{T'} + {\rm dim}N_{T'} & = {\rm dim}U' = {\rm dim}U = {\rm dim}X
\end{align*}
Then it is easy to see that ${\rm dim}N_T = {\rm dim}N_{T'}$.
\end{proof}

\medskip

\section{Similarity}
\begin{definition}\label{similarity}
Given an invertible element $S\in \mathscr{L}(X, X)$, we assign to each $M\in \mathscr{L}(X,X)$ the element
\begin{align*}
    M_S = SMS^{-1}
\end{align*}
The assignment $M\mapsto M_S$ is called similarity transformation, $M$ is said to be similar to $M_S$.
\end{definition}

\medskip

\begin{theorem}
~\begin{enumerate}[label=(\alph*)]
    \item Every similarity transformation is an automorphism of $\mathscr{L}(X,X)$:
    \begin{align*}
        (kM)_S & = kM_S\\
        (M+K)_S & = M_S + K_S\\
        (MK)_S & = M_S K_S
    \end{align*}
    \item The similarity transformations form a group with $$(M_S)_T = M_{TS}.$$
\end{enumerate}
\end{theorem}
\begin{proof}
~\begin{enumerate}[label=(\alph*)]
    \item We only prove $(MK)_S = M_S K_S$. Indeed, we have 
    \begin{align*}
        (MK)_S = S M K S^{-1} = S M S S^{-1} K S^{-1} = M_S K_S.
    \end{align*}
    \item We have 
    \begin{align*}
        M_{TS} = TS M (TS)^{-1} = TS M S^{-1} T^{-1} & = T(S M S^{-1}) T^{-1} \\
        & = (M_S)_T.
    \end{align*}
\end{enumerate}
\end{proof}

\medskip

\begin{theorem}
Similarity is an equivalence relation, i.e., it is:
\begin{enumerate}[label=(\roman*)]
    \item Reflexive. $M$ is similar to itself.
    \item Symmetric. If $M$ is similar to $K$, then $K$ is similar to $M$.
    \item Transitive. If $M$ is similar to $K$, $K$ is similar to $L$, then $M$ is similar to $L$.
\end{enumerate}
\end{theorem}
\begin{proof}
~\begin{enumerate}[label=(\roman*)]
    \item It is true if we choose $S=I$ in definition (\ref{similarity}).
    \item We have $K = SMS^{-1}$, then we have $S^{-1}KS = S^{-1} S M S^{-1} S = M$. Then $K$ is similar to $M$.
    \item We have $K = SMS^{-1}$ and $L=TKT^{-1}$, then we have $$L = TSMS^{-1}T^{-1} = (TS)M(TS)^{-1}$$ which is similar to $M$.
\end{enumerate}
\end{proof}

\medskip

\begin{theorem}
If either $A$ or $B$ in $\mathscr{L}(X,X)$ is invertible, then $AB$ and $BA$ are similar.
\end{theorem}
\begin{proof}
Assume $A$ is invertible, then we have 
$$AB = ABAA^{-1} = (BA)_A.$$
\end{proof}

\medskip

\section{Projection}
\begin{definition}
A linear mapping $P\in \mathscr{L}(X,X)$ is called a projection if $P^2=P$.
\end{definition}

\medskip

\begin{theorem}
If $P\in\mathscr{L}(X,X)$ is a projection, then $X = N_P\oplus R_P$, and $P|_{R_P} = I$ is identity.
\end{theorem}
\begin{proof}
Assume $x\in N_P\cap R_P$, then we have $P(x) = 0$. And $x = P y$ for some $y\in X$. Then we have $Px = P^2 y = P y = x = 0$, which implies $N_P\cap R_P = \{0\}$. Moreover, with ${\rm dim}\, N_P + {\rm dim}\, R_P = {\rm dim}\, X$, we have $X = N_P\oplus R_P$.

For any $x\in R_P$, we have $x = P y$ for some $y\in X$. Then we have $Px = P^2 y = Py = x$, which implies $P|_{R_P}$ is an identity.
\end{proof}
\begin{remark}
The opposite direction of the theorem above is also true. Indeed, for any $x\in X$, we can write $x = y+z$, where $y\in N_P, z\in R_P$. Then we have
\begin{align*}
    Px & = Py + Pz = Pz\\
    P^2 x & = P^2 y + P^2 z = Pz = Px
\end{align*}
Then we have $P^2=P$.
\end{remark}

\medskip

\begin{definition}
The commutator of two linear mappings $A$ and $B$ of $X$ into $X$ is $AB - BA$. Two mappings of $X$ into $X$ commute if their commutator is zero.
\end{definition}

\medskip

\section{Exercises}
\begin{exercise}
Let $X,U$ be two linear spaces such that ${\rm dim}\, X = {\rm dim}\, U < \infty$. Prove that a linear mapping $T\in\mathscr{L}(X,U)$ is one-to-one if and only if it is onto.
\end{exercise}
\begin{proof}
~\begin{enumerate}[label=(\arabic*)]
    \item ($\Rightarrow$) Assume $(x_1, x_2, \cdots, x_n)$ is a basis of $X$, then we have
    $\text{dim} R_T = \text{dim} X - \text{dim} N_T$. If $T\in \mathscr{L}(X,U)$ is one-to-one, then $\text{dim} N_T = 0$. We can have 
    $\text{dim} R_T = \text{dim} X = \text{dim} U$. Then $R_T = U$, which implies that $T$ is an isomorphism. Then $T$ is onto.
    \item ($\Leftarrow$) If $T$ is onto, and $\text{dim} X = \text{dim} N$, then the only element $x\in X$ satisfying $Tx = 0$ is $x = 0$. So $\text{dim} N_T = 0$. Then $\text{dim} R_T = \text{dim} U$, which means $R_T = U$. Then $T$ is an isomorphism and $T$ is of course one-to-one.
\end{enumerate}
\end{proof}

\medskip

\begin{exercise}
Let $X$ be a finite dimensional linear space and $T\in\mathscr{L}(X,X)$. Suppose $${\rm dim}\, R_{T^2} = {\rm dim}\, R_T.$$
Prove that $R_T\cap N_T = \{0\}$, where $T^2 = T\circ T$.
\end{exercise}
\begin{proof}
We knew that $R_{T^2}\subset R_T$, and since ${\rm dim}\, R_{T^2} = {\rm dim}\, R_T$, we have $R_{T^2} = R_T$, which also implies $N_{T^2}\subset N_T$ by Rank-Nullity theorem.

Assume $y\in N_T\cap R_T$,then there must exists a $x\in X$ such that $y = T x$. Then we have $T(Tx) = Ty = 0$, since $y\in N_T$. Then, $x\in N_{T^2} = N_T$, then $Tx = 0 = y$. Now we concluded that $R_T\cap N_T = \{0\}$.
\end{proof}

\medskip

\begin{exercise}
If $Y$ and $Z$ are subspaces of a finite dimensional linear space, prove that 
\begin{align*}
    (Y + Z)^\bot = Y^\bot \cap Z^\bot\,\,{\rm and}\,\, (Y \cap Z)^\bot = Y^\bot + Z^\bot.
\end{align*}
\end{exercise}
\begin{proof}
~\begin{enumerate}[label=(\arabic*)]
    \item Assume $l \in (Y+Z)^\bot$. Then we have $l(m) = 0$, for all $m \in Y+Z$. Also, we know $Y\subset Y+Z$, so $l(y) = 0, \forall y \in Y$. Similarly, we have $l(z) = 0, \forall z \in Z$. Then we have $l \in Y^\bot \cap Z^\bot$, which implies $(Y+Z)^\bot \subset Y^\bot \cap Z^\bot$.
    
    Now assume $l \in Y^\bot \cap Z^\bot$, then we have $l(y) = 0$ and $l(z) = 0$, for $\forall y \in Y, \forall z \in Z$. Then, for $\forall m \in Y+Z$ we have $l(m) = 0$, since $m = y + z$ for some $y \in Y$ and $z \in Z$. Thus, we have $l \in (Y+Z)^\bot$, which implies $Y^\bot \cap Z^\bot \subset (Y+Z)^\bot$. Now we proved that $(Y+Z)^\bot = Y^\bot \cap Z^\bot$.
    \item It is equivalent to prove that $(Y \cap Z)^{\bot\bot} = Y\cap Z = (Y^\bot + Z^\bot)^\bot$.
    
    
    If $l \in (Y^\bot + Z^\bot)^\bot$, we have $l(l_1 + l_2) = 0$, for $l_1 \in Y^\bot$ and $l_2 \in Z^\bot$. Also, we have $Y^\bot \subset Y^\bot + Z^\bot$, we have $l(l_1) = 0$ for $\forall l_1 \in Y^\bot$. Similarly, we have $l(l_2) = 0$ for $\forall l_2 \in Z^\bot$. Then $l \in Y^{\bot\bot} = Y$ and $l \in Z^{\bot\bot} = Z$. Thus, $l \in Y\cap Z$, which implies $(Y^\bot + Z^\bot)^\bot \subset Y\cap Z$.
    
    
    If $l \in Y\cap Z = (Y\cap Z)^{\bot\bot}$, we have $l \in Y\cap Z$, then $l \in Y = Y^{\bot\bot}$ and $l \in Z = Z^{\bot\bot}$. Then we have $l(l_1) = 0, l_1 \in Y^\bot$ and $l(l_2) = 0, l_2 \in Z^\bot$. Thus, we have $l(l_1 + l_2) = 0, l_1 + l_2 \in Y^\bot + Z^\bot$, which implies $l \in (Y^\bot + Z^\bot)^\bot$. Then we have $Y\cap Z \subset (Y^\bot + Z^\bot)^\bot$.
\end{enumerate}
\end{proof}

\medskip

\begin{exercise}
Let $X,Y$ be finite dimensional linear space and $T\in\mathscr{L}(X,Y)$ be invertible. Prove that $T'$ is also invertible and $(T^{-1})' = (T')^{-1}$.
\end{exercise}
\begin{proof}
~\begin{enumerate}[label=(\arabic*)]
    \item Assume $l_1,l_2 \in Y'$,  and $(T'l_1)(x) = (T'l_2, x)$ for all $x\in X$. Then we have $(l_1 - l_2, Tx) = 0$ for all $x\in X$, then we have $l_1 = l_2$, which imlpies $T$ is one-to-one.
    
    Also, if $(T'l)(x) = 0$ for all $x\in X$, then it implies $l(T(x)) = 0$. Then $l$ can only be zero. Thus, $T'$ is invertible.
    \item Since $T$ is invertible, then $T \circ T^{-1} = I$. For $T'$, we denote 
    \begin{align*}
        & (T \circ T^{-1})' = I' \\
        \Rightarrow & (T^{-1})' \circ T' = I'
    \end{align*}
    We need to show $I = I'$. For $y \in Y$, we have 
    \begin{align*}
        & (I' l,y) = (l,I(y)) = (l, y)
    \end{align*}
    Then $I' = I$, thus we have $(T^{-1})' = (T')^{-1}$.
\end{enumerate}
\end{proof}

\medskip

\begin{exercise}
Let $X$ be an $n$-dimensional linear space and $T\in\mathscr{L}(X,X)$. Prove that there is a non-zero polynomial $p(t)$ of degree no more than $n^2$ such that $p(T) = 0$.
\end{exercise}
\begin{proof}
Since $X$ is $n$-dimensional linear space, and $T\in\mathscr{L}(X,X)$, then $T$ can be presented as an $n\times n$ matrix. Polynomials $p(T)$ can be viewed as an operator acting on the space of matrix $T$, which is $n^2$-dimensional, denoted by $P$. Since $\text{dim} P = n^2$, then for any $T \in P$, $1, T, T^2, \cdots, T^{n^2}$ must be linear dependent, since there are $n^2 + 1$ elements. So there exist $a_0, a_1, a_2, \cdots, a_{n^2}$ such that
\begin{align*}
    p(T) = a_0 \cdot 1 + a_1 T + a_2 T^2 + \cdots + a_{n^2}T^{n^2} = 0
\end{align*}
which is at most degree $n^2$. 
\end{proof}

\medskip

\begin{exercise}
Prove that if $U,V,W$ are finite dimensional vector spaces, and $T\in\mathscr{L}(U,V), S\in\mathscr{L}(V,W)$, then
\begin{align*}
    {\rm dim}\, N_{ST} \leq {\rm dim}\, N_S + {\rm dim}\, N_T.
\end{align*}
\end{exercise}
\begin{proof}
Assume $u\in U$, such that $ST(u) = 0$. Then we have two possibilities, that is $u\in N_T$ or $T(u)\in N_S$.

With Rank-Nullity theorem, we have 
\begin{align*}
    {\rm dim}\, N_{ST} & \leq {\rm dim}\, U = {\rm dim}\, N_T + {\rm dim}\, R_T \\
    {\rm dim}\, R_T & \leq {\rm dim}\, V = {\rm dim}\, N_S + {\rm dim}\, R_S\\
    \Rightarrow {\rm dim}\, N_{ST} & \leq {\rm dim}\, N_T + {\rm dim}\, N_S + {\rm dim}\, R_S
\end{align*}
Now we assume $Z = N_{ST}\subset U$, then we have $ST(z) = 0$  for $z\in Z$, and 
\begin{align*}
    {\rm dim}\, N_{ST} & = {\rm dim}\, Z = {\rm dim}\, N_T + {\rm dim}\, T|_Z \\
    {\rm dim}\, T|_Z & \leq {\rm dim}\, N_S + {\rm dim}\, ST(Z)
\end{align*}
combining these, we have ${\rm dim}\, N_{ST} \leq {\rm dim}\, N_S + {\rm dim}\, N_T$.
\end{proof}

\medskip

\chapter{Matrices}
Let $T:\mathbb{R}^n\to\mathbb{R}^m$ be defined by $y = T x$, where
\begin{align*}
    y_i = \sum^n_{j=1} t_{ij}x_j, 1\leq i\leq m.
\end{align*}
Then $T$ is a linear map. On the other hand, every map $T\in \mathscr{L}(\mathbb{R}^n,\mathbb{R}^m)$ can be represented in this form. Actually, $t_{ij}$ is the $i$th component of $Te_j$, where $e_j\in \mathbb{R}^n$ has $j$th component $1$, others be $0$. We write
\begin{align*}
    T = (t_{ij})_{m\times n} = \left(
        \begin{matrix}
        t_{11} & t_{12} & \cdots & t_{1n}\\
        t_{21} & t_{22} & \cdots & t_{2n}\\
        \vdots & \vdots & \ddots & \vdots\\
        t_{m1} & t_{m2} & \cdots & t_{mn}
        \end{matrix}
    \right)
\end{align*}
which is called an $m$ by $n$ ($m\times n$) matrix, where $t_{ij}$ is called the \emph{entries} of the matrix $T$. A matrix is called a \emph{square matrix} if $m=n$. 

A matrix $T$ can be thought of as a row of column vectors, or a column of row vectors: 
\begin{align*}
    T = (c_1,\cdots,c_n) = \left(
    \begin{matrix}
        r_1 \\
        r_2 \\
        \vdots \\
        r_m 
        \end{matrix}
    \right).
\end{align*}
where $c_j = T e_j$, $e_j\in\mathbb{R}^n$ is defined as above.

\medskip
\section{Matrix Multiplication and Transposition}
Since matrices represent linear mappings, the algebra of linear mappings induces a corresponding algebra of matrices, i.e., if $T,S\in \mathscr{L}(\mathbb{R}^n,\mathbb{R}^m)$, then
\begin{align*}
    T+S &= (t_{ij} + s_{ij})_{m\times n}\\
    k T &= (kt_{ij})_{m\times n}
\end{align*}
If $T\in \mathscr{L}(\mathbb{R}^n,\mathbb{R}^m)$ and $S\in \mathscr{L}(\mathbb{R}^m,\mathbb{R}^l)$, then the product $St = S\circ T \in \mathscr{L}(\mathbb{R}^n,\mathbb{R}^l)$. For $e_j\in\mathbb{R}^n$,
\begin{align*}
    (ST)(e_j^n) & = S(Te_j^n) = S\left(\sum^m_{i=1}t_{ij}e^m_i\right) = \sum^m_{i=1}t_{ij} S(e^m_i) \\
    & = \sum^m_{i=1}t_{ij} \left(\sum^l_{k=1}s_{ki}e^l_k\right) = \sum^l_{k=1}\left(\sum^m_{i=1}t_{ij}s_{ki}\right)e^l_k \\
    & = \sum^l_{k=1}(ST)_{kj} e^l_k
\end{align*}
where $e_j^m\in\mathbb{R}^m,e_j^l\in\mathbb{R}^l$. Hence, we have 
\begin{align*}
    (ST)_{kj} = \sum^m_{i=1}t_{ij}s_{ki}
\end{align*}
which is the product of $k$th row of $S$ and $j$th column of $T$.

We can write any $n\times n$ matrix $A$ in $2\times 2$ block form
\begin{align*}
    A = \left(
        \begin{matrix}
        A_{11} & A_{12} \\
        A_{21} & A_{22} 
        \end{matrix}
    \right)
\end{align*}
where $A_{11}$ is $k\times k$ matrix, and $A_{22}$ is $(n-k)\times(n-k)$ matrix. 

We shall identify the dual of the space $\mathbb{R}^n$ of all column vectors with $n$ components as the space $(\mathbb{R}^n)'$ of all row vectors with $n$ components. For $l\in (\mathbb{R}^n)'$ and $x\in\mathbb{R}^n$, 
$$l x = \sum^n_{i=1}l_ix_i$$

Let $x\in\mathscr{L}(\mathbb{R},\mathbb{R}^n), T \in \mathscr{L}(\mathbb{R}^n,\mathbb{R}^m)$, and $l \in \mathscr{L}(\mathbb{R}^m,\mathbb{R})$ be linear mappings, according to associative law, we have
$$(lT)x = l(Tx)$$
We identity $l$ as an element of $(\mathbb{R}^m)'$, and $lT$ as an element of $(\mathbb{R}^n)'$, and we can rewrite is into form
$$(lT, x) = (l, Tx)$$
and we recall the definition of transpose $T'$ of $T$, defined by $(T'l, x) = (l, Tx)$. Now we can define the transpose $T^T$ of the matrix $T$ as 
$$\left(T^T\right)_{ij} = T_{ji}.$$

\medskip

\section{Rank}
\begin{theorem}
Let $T\in\mathscr{L}(\mathbb{R}^n,\mathbb{R}^m)$, the range of $T$ consists of all linear combinations of the columns of the matrix $T$.
\end{theorem}

\medskip

\begin{definition}
The ${\rm dim}\, R_T$ is called the column rank of $T$, and ${\rm dim}\, R_{T^T}$ is called the row rank of $T$.
\end{definition}

\begin{theorem}
${\rm dim}\,R_T = {\rm dim}\, R_{T^T}$.
\end{theorem}
\begin{proof}
We can apply elementary row operations and elementary column operations to make $A$ into a matrix that is in both row and column reduced form, i.e., there exist invertible matrices $P$ and $Q$ (which are products of elementary matrices) such that
\begin{align*}
    PAQ = E = \left(
        \begin{matrix}
        I_{k} &  \\
         & 0_{(m-k)\times (n-k)}
        \end{matrix}
    \right)
\end{align*}
Since $P$ and $Q$ are invertible, then the maximum number of linearly independent rows in $A$ is equal to the maximum number of linearly independent rows in $E$. Also, it is similar for the column rank. Then it is obvious that ${\rm dim}\,R_T = {\rm dim}\, R_{T^T}$.
\end{proof}

Now we present another different approach to prove this theorem.

\begin{proof}
Let $T$ be $m\times n$ matrix and it has row rank $k$. Therefore, the dimension of the row space of $T$ is $k$. Let $x_1,\cdots,x_k$ be a basis of row space of $T$ and we claim that $Tx_1,\cdots, Tx_k$ are linearly independent. Indeed, we choose coefficients $c_1,\cdots,c_k$ and then 
\begin{align*}
    c_1 Tx_1 + \cdots + c_k Tx_k = T(c_1x_1 +\cdots + c_kx_k) = Tx = 0
\end{align*}
Then $x$ is a linear combination of basis of row space of $T$, which implies that $x$ belongs to row space of $T$. Also, $TX = 0$ implies that $x$ is orthogonal to every vector of row space of $T$, then $x$ is orthogonal to itself, giving us $x^2 = c_1^2 x_1^2 + \cdots + c_k^2 x_k^2 = 0$. Then it is obvious that $c_1 = \cdots = c_k = 0$.

Now, each $Tx_i$ is obviously in the column space of $T$, and then $Tx_1,\cdots, Tx_k$ are $k$ linearly independent vectors in the column space of $T$, implying that ${\rm dim}\, R_{T_T}\leq {\rm dim}\,R_T$. 

Now we can consider $T^T$ in the similar argument, and it will give us ${\rm dim}\, R_{T_T}\geq {\rm dim}\,R_T$. Thus, we have ${\rm dim}\, R_{T_T} = {\rm dim}\,R_T$. 
\end{proof}

\medskip

Next, we discuss some properties of rank of a matrix.
\begin{proposition}
Let $T\in\mathscr{L}(\mathbb{R}^n,\mathbb{R}^m)$, and define the linear map $f$ by $f(x) = Tx$, then
\begin{enumerate}[label=(\arabic*)]
    \item ${\rm rank}\,(T)\leq \min(m,n)$. A matrix that has rank equal to $\min(m, n)$ is called full rank; otherwise, the matrix is rank deficient.
    \item Only a zero matrix has rank zero.
    \item $f$ is injective(or one-to-one) if and only if $T$ has rank $n$, i.e. full column rank.
    \item $f$ is surjective(or onto) if and only if $T$ has rank $m$, i.e. full row rank.
    \item If $T$ is a square matrix, i.e., $m = n$, then $T$ is invertible if and only if $T$ has rank $n$(that is, $T$ has full rank).
    \item If $T$ is a square matrix, then $T$ is invertible if and only if its determinant is non-zero.
    \item If $S\in\mathscr{L}(\mathbb{R}^n,\mathbb{R}^k)$, then $${\rm rank}\,(TS) \leq \min( {\rm rank}\,(T), {\rm rank}\,(S)).$$
    \item If $S$ is an $n\times k$ matrix of rank $n$, then $${\rm rank}\,(TS) = {\rm rank}\,(T).$$
    \item If $K$ is a $l\times m$ matrix of rank $m$, then$${\rm rank}\,(KT) = {\rm rank}\,(T).$$
    \item The rank of $T$ is equal to $k$ if and only if there exists an invertible $m\times m$ matrix $P$ and an invertible $n\times n$ matrix $Q$ such that
    $$PTQ = \left(
        \begin{matrix}
        I_{k} & 0 \\
        0 & 0
        \end{matrix}
    \right)$$
    where $I_{k\times k}$ is $k\times k$ identity matrix.
    \item Sylvester’s rank inequality: if $A$ is an $m\times n$ matrix and $B$ is $n\times k$, then
    $${\rm rank}\,(A) + {\rm rank}\,B - n \leq {\rm rank}\,(AB).$$
    \item Frobenius inequality: if $AB, ABC$ and $BC$ are defined, then
    $${\rm rank}\,(AB) + {\rm rank}\,(BC) \leq {\rm rank}\,(B) + {\rm rank}\,(ABC).$$
    \item Subadditivity:
    $${\rm rank}\, (A+B)\leq {\rm rank}\, (A)+{\rm rank}\, (B).$$
    when $A$ and $B$ are of the same dimension. As a consequence, a rank-$k$ matrix can be written as the sum of $k$ rank-$1$ matrices, but not fewer.
    \item Rank–Nullity theorem: The rank of a matrix plus the nullity of the matrix equals the number of columns of the matrix, i.e., for $T$ being a $m\times n$ matrix, then $${\rm Rank}\, T + {\rm Nullity}\, T = n.$$
\end{enumerate}
\end{proposition}
\begin{proof}
~\begin{enumerate}[label=(\arabic*)]
    \item Since $T:\mathbb{R}^n\to\mathbb{R}^m$, then the image of $T$ is a subspace of $\mathbb{R}^m$, and it is easy to see that ${\rm rank}\,(T) \leq m$. Also, with Rank-Nullity theorem, we have ${\rm rank}\,(T) \leq n$. Thus, ${\rm rank}\,(T)\leq \min(m,n)$.
    \item $T$ is $m\times n$ matrix and has rank $0$, then the nullity of $T$ is $n$, which implies that all columns of $T$ are zero vectors.
    \item ($\Rightarrow$) If $f$ is injective, then there exists only one element $x'$ in $\mathbb{R}^n$ such that $Tx' = 0\in\mathbb{R}^m$. Also, we claim $x' = 0\in\mathbb{R}^n$. Indeed, we have $T0 = T(0 - 0) = T(0) - T(0) = 0$. Thus, we have $N_T = \{0\}$, implying that $T$ is full column rank.
    
    ($\Leftarrow$) Since $T$ is full column rank, and then we have $N_T = \{0\}$. For $x_1,x_2\in\mathbb{R}^n$ such that $Tx_1 = Tx_2$, we have $T(x_1 - x_2) = 0$. Thus, $x_1 = x_2$, implying that $f$ is one-to-one.
    \item ($\Rightarrow$) If $f$ is surjective, then the columns of $T$ span the space $\mathbb{R}^m$, which implies ${\rm rank}\, T = m$.
    
    ($\Leftarrow$) This direction is bovious.
    \item ($\Rightarrow$) If $T$ is invertible, then there exists $T^{-1}$ such that $T T^{-1} = I$. Then we have $\det(T) \det(T^{-1}) = 1$, which implies $\det(T)\neq 0$. Thus, $T$ has full rank.
    
    ($\Leftarrow$) If $T$ is full rank, the its row reduced echelon form is identity matrix. Then there exists a $n\times n$ matrix $H$ such that $TH = I$. Thus, $T$ is invertible.
    \item It is shown in last statement.
    \item We have $TS$ is a $m\times k$ matrix, and then we have $R_{TS}\subset R_T$, which implies ${\rm rank}\,(TS) \leq {\rm rank}\,(T)$. Similarly, we have $R_{(TS)'}\subset R_{S'}$, which implies ${\rm rank}\,(TS) = {\rm rank}\,(TS)' \leq {\rm rank}\,(S') = {\rm rank}\,S$. Then the result follows.
    \item The rank is the dimension of the column space. The column space of $TS$ is the same as the column space of $T$. Indeed, for any $y\in \mathbb{R}^n$, there is a $x\in\mathbb{R}^k$ such that $y = S x$, since $S$ is of rank $n$, implying that $S$ is onto. Then we have $Ty = TSx$. Thus, ${\rm rank}\, (TS) = {\rm rank}\, (T)$.
    \item For any $x\in\mathbb{R}^n$, we denote $T x$ by $y\in\mathbb{R}^m$. And since $K$ is full column rank, then the rank of $Ky$ is equal to the rank of $y$, which implies that ${\rm rank}\,(KT) = {\rm rank}\,(T)$.
    \item With Rank-Nullity theorem and $T:\mathbb{R}^n\to\mathbb{R}^m$, we have $\dim R_T + \dim N_T = n$. Then we can find a basis $(x_1,\cdots,x_{k},x_{k+1},\cdots,x_n)$ for $\mathbb{R}^n$, where $(x_{k+1},\cdots,x_n)$ is a basis for null space $N_T$ of $T$.
    
    Now we define $f_j = T(e_j), 1\leq j\leq k$, then it is easy to see that $(f_1, \cdots, f_k)$ is linearly independent. Then we can complete this basis into a basis $(f_1, \cdots, f_k,f_{k+1},\cdots,f_m)$ of $\mathbb{R}^m$. Relative to this basis, we can choose $$f_1 = (1,0,\cdots, 0)^T, \cdots, f_j = (0,\cdots,\underbrace{1}_{j {\rm th}}, \cdots, 0)^T, \cdots, f_k = (0,\cdots,\underbrace{1}_{k {\rm th}}, \cdots, 0)^T$$
    which gives us $I_K$, along with all zeros below for the first $k$ columns of $T$, which is \cite{3}
    \begin{align*}
        \left(
        \begin{matrix}
        I_{k} \\
        0
        \end{matrix}
        \right).
    \end{align*}
    \item Suppose $A$ is an $m\times n$ matrix and $B$ is an $n\times k$ matrix, then we have $AB$ is an $m\times k$ matrix. With Rank-Nullity theorem, we have 
    \begin{align*}
        \dim R_A + \dim N_A & = n \\
        \dim R_B + \dim N_B & = k \\
        \dim R_{AB} + \dim N_{AB} & = k
    \end{align*}
    Then, we have
    \begin{align*}
        \dim N_A + \dim R_B + \dim N_A + \dim N_B = n + \dim R_{AB} + \dim N_{AB} \\
        \Rightarrow \dim R_{AB} - \dim R_A - \dim R_B + n = \dim N_A + \dim N_B - \dim N_{AB} \geq \dim N_A \geq 0
    \end{align*}
    since $\dim N_B - \dim N_{AB}\leq 0$. Indeed, for any $v\in N_B$, we have $BV = 0$, also, we have $ABv = 0$, which implies $N_B\subset N_{AB}$ \cite{4}. 
    \item Consider $A_{m\times n}$ and $B_{n\times k}$ and $B$ is of rank $r$. Using full-rank factorization of $B$, we have $B = U_{n\times r}V_{r\times k}$, where both $U$ and $V$ are of rank $r$\cite{7}\cite{5}. With Sylvester’s rank inequality, we have
    \begin{align*}
        {\rm rank}\, (ABC) & \geq {\rm rank}\, (AU) + {\rm rank}\, (VC) - r\\
        & = {\rm rank}\, (AB) + {\rm rank}\, (BC) - {\rm rank}\, (B)
    \end{align*}
    Then the inequality follows\cite{6}.
    \item It is easy to see that $C(A+B)\subset C(A)+C(B)$, where $C(A)$ denote the column space of $A$. Indeed, for any $y\in C(A+B)$, we can find $x$ such that $y = (A+B)x = Ax + Bx\in C(A)+C(B)$. Thus, ${\rm rank}\, (A+B)\leq {\rm rank}\, (A)+{\rm rank}\, (B)$.
    \item Rank–Nullity theorem is proved before.
\end{enumerate}
\end{proof}

\medskip

A linear mapping $T\in\mathscr{L}(X,U)$ can be represented by a matrix if the bases for $X$ and $U$ are chosen. A choice of basis for $X$ defineds an isomorphism $B:X\to\mathbb{R}^n$, and similarly, we have isomorphism $C: U\to\mathbb{R}^m$. Clearly, there are as many isomorphisms as there are bases. We can use any of these isomorphisms to represent $T$ as a matrix from $\mathbb{R}^n$ to $\mathbb{R}^m$, and we have a matrix representation $M$:
\begin{align*}
    CTB^{-1} = M.
\end{align*}

If $T\in\mathscr{L}(X,X)$, and $B:X\to\mathbb{R}^n$ is an isomorphism, then we have $M = BTB^{-1}\in\mathscr{L}(\mathbb{R}^n,\mathbb{R}^n)$ is a square matrix. Let $C:X\to\mathbb{R}^n$ be another isomorphism, then $N = CTC^{-1}$ is another square matrix. Also, we have 
\begin{align*}
    N = C B^{-1} M B C^{-1}
\end{align*}
then $M$ and $N$ are similar. Thus, similar matrices represent the same linear mapping under different choices of bases.

\medskip

\begin{definition}
Two $n\times n$ matrices $A$ and $B$ are similar if there exists isomorphism $M\in\mathscr{L}(\mathbb{R}^n,\mathbb{R}^n)$ such that $A = MBM^{-1}$.
\end{definition}

\medskip

\begin{definition}
An $n\times n$ matrix $A$ is said to be invertible if and only if $A$ is an isomorphism. And we say $A$ is singular if it is not invertible.
\end{definition}
\begin{remark}
Invertible, non-singular and full rank are equivalent.
\end{remark}

\medskip

\begin{definition}
Let $I$ be the identity matrix, if $A$ is invertible, then there exists a matrix, called inverse of $A$, denoted by $A^{-1}$ such that $AA^{-1} = A^{-1}A = I$.
\end{definition}

\medskip

\section{Exercises}


\medskip

\chapter{Determinant and Trace}
\section{Ordered Simplices, Signed Volume and Determinant}
A \emph{simplex} in $\mathbb{R}^n$ with $n+1$ vertices. We take one of the vertices to be the origin and denote others by $a_1,\cdots, a_n$. The orders in which the vertices are taken matters, and we say $0, a_1,\cdots, a_n$the vertices of an \emph{ordered simplex}.

An ordered simplex $S$ is called \emph{degenerate} if it lies on an $(n-1)$-dimensional subspace. An ordered nondegenerate simplex $S = (0, a_1,\cdots, a_n)$ is called \emph{positively oriented} if it can be deformed continuously and nondegenerately into the standard simplex $(0, e_1,\cdots, e_n)$, where $e_j$ is the $j$th unit vector in the standard basis of $\mathbb{R}^n$. By such deformation we mean $n$ vector-valued continuous functions $a_j(t)$ of $t,0 < t < 1$, such that (i) $S(t) = (0,a_1(t),\cdots, a_n(t))$ is nondegenerate for all $t$ and (ii) $a_j(0) = 0, a_j(i) = e_j$. 
Otherwise, we say it \emph{negatively oriented}. 

For a nondegenerate simplex $S$, we define $\mathcal{O}(S) = +1(-1)$ if it is positivelt (negatively) oriented. For a degenerate simplex $S$, we set $\mathcal{O}(S) = 0$. The \emph{volume} of a simplex $S$ is given by elementary formula
\begin{align*}
    {\rm Vol}(S) = \frac{1}{n} {\rm Vol}_{n-1}({\rm base}) \times {\rm Altitude}
\end{align*}
by base we mean any of the $(n-1)$-dimensional surfaces of $S$, and by altitude we mean the distance from the opposite vertices to the hyperplane that contains the base. 

And the \emph{signed volume} of an ordered simplex $S$ is defined as 
\begin{align*}
    \sum (S) = \mathcal{O}(S){\rm Vol}(S).
\end{align*}
Since $S$ is described by its vertices, $\sum (S)$ is a function of $a_1,\cdots,a_n$. Obviously, when two vertices are equal, $S$ is degenerate. Thus, we have following properties:
\begin{enumerate}[label=(\roman*)]
    \item $\sum (S) = 0$ if $a_j = a_k, i\neq k$.
    \item $\sum (S)$ is a linear function of $a_j$ when $a_k, k\neq j$ are fixed.
    \item $\sum (0,e_1,\cdots,e_n) = \frac{1}{n!}$.
\end{enumerate}

Now we consider the signed volume as
\begin{align*}
    \sum (S) =\frac{1}{n}{\rm Vol}_{n-1}({\rm base})
\end{align*}
where $k = \mathcal{O}(S) {\rm Altitude}$. The altitude is the \emph{distance} of the vertex $a_j$, also $k$ is called \emph{signed distance} of the vertex from the hyperplane containing the base. 

Determinant are related to the signed volume of ordered simplices by formula
\begin{align*}
    \sum (S) = \frac{1}{n!}D(a_1,\cdots,a_n).
\end{align*}

\medskip

\begin{definition}
Let $A = (a_1,\cdots, a_n)$ be a square matrix, where $a_j\in\mathbb{R}^n, 1\leq j\leq n$ are column vectors. Its determinant is defined by 
\begin{align*}
    \det A = D(a_1,\cdots,a_n) = n! \sum (S) 
\end{align*}
where $S = (0,a_1,\cdots,a_n)$.
\end{definition}

\medskip

\begin{theorem}\label{determinant}
~\begin{enumerate}[label=(\roman*)]
    \item $D(a_1,\cdots,a_n) = 0$ if $a_j = a_k$ for some $j\neq k$.
    \item $D(a_1,\cdots,a_n)$ is a multilinear function of its arguments.
    \item Normalization: $D(e_1,\cdots,e_n) = 1$.
    \item $D$ is an alternating function of its arguments, i.e., if $a_j$ and $a_k$ are interchanged, $j\neq k$, the value of $D$ changes by $-1$.
    \item If $a_1,\cdots,a_n$ are linearly dependent, then $D(a_1,\cdots,a_n) = 0$.
\end{enumerate}
\end{theorem}
\begin{proof}
The first three statements are obvious. We only prove $(iv)$ and $(v)$.
\begin{enumerate}[label=(\roman*)]
    \setcounter{enumi}{3}
    \item Let $D(a,b) = (\cdots,a_i,\cdots,a_j,\cdots)$ and $D(b,a) = (\cdots,a_j,\cdots,a_i,\cdots)$. Then we have
    \begin{align*}
        D(a,b) & = D(a,a) + D(a,b) \\
        & = D(a,a+b) - D(a+b, a+b) \\
        & = - D(b, a+b) \\
        & = - D(b, a+b) + D(b,b)\\
        & = -D(b,a).
    \end{align*}
    \item Suppose $a_1,\cdots,a_n$ are linearly dependent, then there exist $c_1,\cdots,c_n$ not all zero, such that $\sum^n_{j=1}c_j a_k = 0$. Without losing generality, assume $c_1\neq 0$, then we have
    \begin{align*}
        a_1 & = -\sum^n_{j=2}\frac{c_k}{c_1}a_k \\
        \Rightarrow D(a_1,\cdots,a_n) & = D\left(-\sum^n_{j=2}\frac{c_k}{c_1}a_k, a_2,\cdots,a_n\right) \\
        & = -\frac{c_k}{c_1} \sum^n_{j=2} D(a_k, a_2, \cdots, a_n) = 0.
    \end{align*}
\end{enumerate}
\end{proof}

\medskip

\section{Permutation}
\begin{definition}
A permutation is a mapping $p$ of $n$ objects, saying the numbers $1,2,\cdots,n$, onto themselves. Permutations are invertible
and they form a group with compositions. These groups, except for $n = 2$, are noncommutative.
\end{definition}

\medskip

\begin{example}
Let $p = \frac{1234}{2413}$. Then
\begin{align*}
    p^2 & = \frac{1234}{4321},\quad p^{-1} = \frac{1234}{3142} \\
    p^3 & = \frac{1234}{3142},\quad p^{4} = \frac{1234}{1234}.
\end{align*}
\end{example}

\medskip

Next we introduce \emph{signature} of a permutation, denoted by $\sigma(p)$. Let $x_1,\cdots,x_n$ be $n$ variables, their \emph{discriminant} is defined by 
\begin{align*}
    P(x_1,\cdots,x_n) = \prod_{i < j}(x_i - x_j).
\end{align*}
Let $p$ be any permutation, then we have 
\begin{align*}
    \prod_{i < j}\left(x_{p(i)} - x_{p(j)} \right)
\end{align*}
is either $P(x_1,\cdots,x_n)$ or $- P(x_1,\cdots,x_n)$.

\medskip

\begin{definition}
The signature $\sigma(p)$  of a permutation $p$ is defined by
\begin{align*}
    P\left(x_{p(1)},\cdots,x_{p(n)}\right) = \sigma(p) P(x_1,\cdots,x_n).
\end{align*}
Hence, $\sigma(p) = \pm 1$.
\end{definition}

\medskip

\begin{theorem}
$\sigma(p_1\circ p_2) = \sigma(p_1) \sigma(p_2)$.
\end{theorem}
\begin{proof}
\begin{align*}
    \sigma(p_1\circ p_2) & = \frac{P\left(x_{p_1p_2(1)},\cdots,x_{p_1p_2(n)}\right)}{P(x_1,\cdots,x_n)} \\
    & = \frac{P\left(x_{p_1p_2(1)},\cdots,x_{p_1p_2(n)}\right)}{P(x_{p_2(1)},\cdots,x_{p_2(n)})}\cdot \frac{P\left(x_{p_2(1)},\cdots,x_{p_2(n)}\right)}{P(x_1,\cdots,x_n)}\\
    & = \sigma(p_1) \sigma(p_2).
\end{align*}
\end{proof}

\medskip

Given any pair of indices, $j\neq k$, we can define a permutation $p$ such that
\begin{align*}
    p(i) = \begin{cases}
    i, i\neq j\, {\rm or}\,\, k \\
    k, i = j \\
    j, i = k
    \end{cases}
\end{align*}
Such a permutation is called \emph{transposition}. And we claim that transposition has following properties:
\begin{enumerate}[label=(\arabic*)]
    \item The signature of a transposition $t$ is $-1$, i.e., $\sigma(t) = -1$.
    \item Every permutation $p$ can be written as a composition of transpositions, i.e., 
    \begin{align}\label{transposition}
        p = t_k\circ \cdots \circ t_1
    \end{align}
\end{enumerate}
\begin{proof}
~\begin{enumerate}[label=(\arabic*)]
    \item Assume $t$ interchanges $i_0$ and $j_0$, with $i_0<j_0$, then we have
    \begin{align*}
        P\left(t(x_1,\cdots,x_n)\right) & = P(x_1,\cdots,x_{j_0},\cdots,x_{i_0},\cdots,x_n) \\
        & = (x_{j_0} - x_{i_0})\prod_{i<j,(i,j)\neq(i_0,j_0)} (x_i - x_j) \\
        & = - \prod_{i<j}(x_i - x_j) \\
        & = - P(x_1,\cdots,x_n)
    \end{align*}
    Hence, $\sigma(t) = -1$.
    \item It is easy to see that $p = t_k\circ \cdots \circ t_1$ is equivalent to $I = p = t_k\circ \cdots \circ t_1 \circ p^{-1}$. Consider $(1,\cdots, n) = t_k\circ \cdots \circ t_1 \circ p^{-1}(1,\cdots,n)$. Then we claim a sequence of transposition can sort an array of numbers into ascending order.
\end{enumerate}
\end{proof}

With the results above, we have
\begin{align*}
    \sigma(p) = (-1)^{k}
\end{align*}
where $k$ is the number of transpositions in the decomposition (\ref{transposition}) of $p$.

\medskip

\section{Formula for Determinant}
\begin{theorem}
Assume that for $1\leq k \leq n$, 
\begin{align*}
    a_k = \left(
    \begin{matrix}
        a_{1k} \\
        \vdots \\
        a_{n k}
    \end{matrix}
    \right)\in\mathbb{R}^n.
\end{align*}
This is the same as 
\begin{align*}
    a_k = a_{1k}e_1 + \cdots + a_{nk}e_n.
\end{align*}
with multilinearity, we can write
\begin{align*}
    D(a_1,\cdots,a_n) & = D(a_{11}e_1 + \cdots + a_{n1}e_n, a_2,\cdots,a_n) \\
    & = a_{11}D(e_1,a_2,\cdots,a_n) + \cdots + a_{n1}D(e_n,a_2,\cdots,a_n)
\end{align*}
Next we can express $a_2$ as a linear combination of $e_1,\cdots,e_n$, and obtain a equation like above, with $n^2$ terms. Repeating this process, we have
\begin{align*}
    D(a_1,\cdots,a_n) = \sum_f a_{f_1 1}a_{f_2 2}\cdots a_{f_n n} D(e_{f_1}, e_{f_2},\cdots, e_{f_n})
\end{align*}
where the summation is over all functions $f$ mapping $\{1,\cdots,n\}$ into $\{1,\cdots,n\}$. If $f$ is not a permutation, then $f_i = f_j$ for some $i\neq j$. Then we have $D(e_{f_1}, e_{f_2},\cdots, e_{f_n}) = 0$. This shows that we only need to sum over permutations.

Since each permutation can be decomposed into k transpositions, thus we have 
\begin{align*}
    D(e_{f_1}, e_{f_2},\cdots, e_{f_n}) = \sigma(p) D(e_1, e_2,\cdots, e_n)
\end{align*}
for any permutation. Then the determinant can be represented as
\begin{align*}
    D(a_1,\cdots,a_n) = \sum_p \sigma(p) a_{p(1)1} a_{p(2)2} \cdots a_{p(n)n}
\end{align*}
where the summation is over all permutations.
\end{theorem}
\begin{proof}
The proof is within the explanation of the theorem, i.e., 
\begin{align*}
    D(a_1,\cdots,a_n) & = D\left(\sum^n_{j=1}a_{j1}e_j, \cdots, \sum^n_{j=1}a_{jn}e_j\right) \\
    & = \sum_{1\leq j_k \leq n, 1\leq k \leq n} a_{f_1 1}a_{f_2 2}\cdots a_{f_n n} D(e_{f_1}, e_{f_2},\cdots, e_{f_n}) \\
    & = \sum_p \sigma(p) a_{p(1)1} a_{p(2)2} \cdots a_{p(n)n}.
\end{align*}
\end{proof}

\begin{remark}
Determinant is defined by properties (i),(ii) and (iii) in Theorem \ref{determinant}.
\end{remark}

\medskip

\begin{theorem}
$\det A^T = \det A$.
\end{theorem}
\begin{proof}
Assume $A = (a_{ij})_{n\times n}$, then $A^T = (b_{ij})_{n\times n}, b_{ij} = a_{ji}$. Then we have 
\begin{align*}
    \det A^T & = \sum_p \sigma(p) b_{p(1)1} b_{p(2)2} \cdots b_{p(n)n} \\
    & = \sum_p \sigma(p) a_{1 p(1)} a_{2 p(2)} \cdots a_{n p(n)} \\
    & = \sum_p \sigma(p) a_{p^{-1}(1)} a_{p^{-1}(2) 2} \cdots a_{p^{-1}(n) n}
\end{align*}
we denote $p^{-1}$ by $\Tilde{p}$, then we have
\begin{align*}
    \det A^T = \sum_{\Tilde{p}} \sigma(\Tilde{p}) a_{\Tilde{p}(1)} a_{\Tilde{p}(2) 2} \cdots a_{\Tilde{p}(n) n} = \det A.
\end{align*}
\end{proof}

\medskip

\begin{theorem}
Let $A,B$ be two $n\times n$ matrices, then $\det (BA) = \det A\cdot \det B$.
\end{theorem}
\begin{proof}
Assume $A = D(a_1,\cdots,a_n)$, then $BA = (Ba_1,\cdots,Ba_n)$, which implies $\det BA = D(Ba_1,\cdots,Ba_n)$. 
\begin{enumerate}[label=(\arabic*)]
    \item Define for $\det B\neq 0$, that $C(a_1,\cdots,a_n) = \frac{\det BA}{\det B}$. It suffices to show that $C$ satisfies:
    \begin{enumerate}[label=(\roman*)]
        \item If $a_i = a_j$ for some $i\neq j$, then $C = 0$. Indeed, if $a_i = a_j$ for some $i\neq j$, then $Ba_i = Ba_j$. Thus, $D(Ba_1,\cdots,Ba_n) = 0$.
        \item $C$ is linear in $a_k, 1\leq k \leq n$. This is obvious.
        \item $C(e_1,\cdots,c_n) = 1$. Indeed, setting $a_i = e_i, 1\leq i\leq n$. And we get 
        \begin{align*}
            C(e_1,\cdots,e_n) & = \frac{D(Be_1,\cdots,Be_n)}{\det B} \\
            & = \frac{D(b_1,\cdots,b_n)}{\det B} \\
            & = \frac{\det B}{\det B} = 1.
        \end{align*}
        Then we claim $C(a_1,\cdots,a_n) = \det A$.
    \end{enumerate}
    \item If $\det B = 0$, then there exists $\varepsilon_n\to 0$ as $n\to\infty$ such that $\det (B+\varepsilon_n I)\neq 0$. Then we have 
    \begin{align*}
        \det \left((B+\varepsilon_n I)A\right) & = \det (B+\varepsilon_n I) \det A \\
        \overset{n\to\infty}&{=} \det A \det B.
    \end{align*}
\end{enumerate}
\end{proof}

\medskip

\begin{corollary}
Let $A$ be an $n\times n$ matrix, then $A$ is invertible if and only if $\det A\neq 0$.
\end{corollary}
\begin{proof}
~\begin{enumerate}[label=(\arabic*)]
    \item ($\Rightarrow$) If $A$ is invertible, then there exists $A^{-1}$ such that $A^{-1}A = I$. Then we have $\det A = 1/\det A^{-1} \neq 0$.
    \item If $\det A\neq 0$, then $A$ is both full  row rank and full column rank. Then, $A$ is bijective from $\mathbb{R}^n$ to $\mathbb{R}^n$. Thus, $A$ is invertible.
\end{enumerate}
\end{proof}

\medskip

\section{Laplace Expansion}
Now we discuss another property of determinant, starting with a lemma.

\medskip

\begin{lemma}
Let $A$ be an $n\times n$ matrix, whose first column is $e_1$:
\begin{align*}
    A = \left(
    \begin{matrix}
        1 & \times\\
        0 & A_{11}
    \end{matrix}
    \right),
\end{align*}
here $A_{11}$ denote the $(n-1)\times(n-1)$ submatrix formed by entries $a_{ij},i>1, j>1$. We claim that $$\det A = \det A_{11}.$$
\end{lemma}
\begin{proof}
First, we show that $\det A =\det \left(
    \begin{matrix}
        1 & 0\\
        0 & A_{11}
    \end{matrix}
    \right)$. And from properties (i) and (ii) that if we add suitable multiples of the first column of $A$ to the others, we can obtain $\left(
    \begin{matrix}
        1 & 0\\
        0 & A_{11}
    \end{matrix}
    \right)$, and the determinant will not change.

Define 
\begin{align*}
    C(A_{11}) = \det \left(
    \begin{matrix}
        1 & 0\\
        0 & A_{11}
    \end{matrix}
    \right)
\end{align*}
then it suffices to verify $C$ satisfies all three properties:
\begin{enumerate}[label=(\arabic*)]
    \item If $a_i, a_j\in A_{11}$ such that $a_i = a_j, i\neq j$, then we have $\left(
    \begin{matrix}
        0\\
        a_1
    \end{matrix}
    \right) = \left(
    \begin{matrix}
        0\\
        a_2
    \end{matrix}
    \right)$ Thus, $C(A_{11}) = 0$.
    \item Any linear operations of $A_{11}$ can be extended to $\left(
    \begin{matrix}
        0\\
        a_i
    \end{matrix}
    \right)$, then $C$ is multilinear.
    \item When $A_{11} = I_{(n-1)\times(n-1)}$, then we have $\left(
    \begin{matrix}
        1 & 0\\
        0 & A_{11}
    \end{matrix}
    \right) = I_{n\times n}$, then $C(A_{11}) = 1$.
\end{enumerate}
\end{proof}

Now we present another approach to prove this lemma.
\begin{proof}
\begin{align*}
    \det \left(
    \begin{matrix}
        1 & \times\\
        0 & A_{11}
    \end{matrix}
    \right) & = \det \left(
    \begin{matrix}
        1 & 0\\
        \times & A_{11}^T
    \end{matrix}
    \right) = \det \left(
    \begin{matrix}
        1      & 0 & \cdots & 0\\
        a_{12} &  &\\
        \vdots &  & A_{11}\\
        a_{1n}
    \end{matrix}
    \right)\\
    & = D\left(e_1 + \sum^n_{k=2}a_{1k}e_k, \widetilde{A_{11}}\right) \\
    & = D(e_1, \widetilde{A_{11}}) + \sum^n_{k=2}a_{1k} D(e_k, \widetilde{A_{11}}) \\
    & = \det A_{11}.
\end{align*}
\end{proof}

\medskip

\begin{theorem}[Laplace expansion]
For any $j = 1,\cdots,n$, 
\begin{align*}
    \det A = \sum^n_{i=1}(-1)^{i+j} a_{ij}c\det A_{ij}
\end{align*}
where $A_{ij}$ is the $(ij)$th minor of $A$.
\end{theorem}
\begin{proof}
The $j$th column $a_j = \sum a_{ij} e_i$. Hence,
\begin{align*}
    \det A & = \sum^N_{i=1} a_{ij} D(a_1,\cdots,a_{j-1},e_i, a_{j+1},\cdots,a_n) \\
    & = \sum^n_{i=1} (-1)^{i+j} a_{ij} \det A_{ij}
\end{align*}
where we need the lemma below.
\end{proof}

\medskip

\begin{lemma}
Let $A$ be a matrix with $j$th column being $e_i$. Then 
\begin{align*}
    \det A = (-1)^{i+j} \det A_{ij}.
\end{align*}
\end{lemma}
\begin{proof}
\begin{align*}
    \det A & = D(a_1,\cdots,a_{j-1}, e_i, a_{j+1}, \cdots, a_n) \\
    & = (-1)^{j-1} D(e_i, a_1,\cdots,a_{j-1}, a_{j+1}, \cdots, a_n) \\
    & = (-1)^{i+j-2} \det \left(
    \begin{matrix}
        1 & \times\\
        0 & A_{11}
    \end{matrix}
    \right) \\
    & = (-1)^{i+j} \det A_{11}.
\end{align*}
where the last step comes from lemma above. 
\end{proof}

\medskip

\section{Cramer's Rule}

If $A_{n\times n}$ is invertible, then for all $u\in\mathbb{R}^n$, $Ax = u$ has a unique solution $x = A^{-1}u$. Assume $A = (a_1, \cdots, a_n)$ and $x = \sum x_j e_j$, then we have 
\begin{align*}
    u = \sum x_j a_j.
\end{align*}

Now consider $A_k = (a_1, \cdots, a_{k-1}, \underbrace{u}_{k\,{\rm th}}, a_{k+1}, \cdots, a_n)$. Then we have 
\begin{align*}
    \det A_k = \sum x_j \det (a_1, \cdots, a_{k-1}, a_j, a_{k+1}, \cdots, a_n) = x_k \det A
\end{align*}
hence, we have
$$x_k = \frac{\det A_k}{\det A}.$$
And since 
\begin{align*}
    \det A_k = \sum^n_{j=1} (-1)^{j+k} u_j \det A_{jk}
\end{align*}
we have 
\begin{align*}
    x_k = \sum^n_{j=1} (-1)^{j+k} u_j \frac{\det A_{jk}}{\det A}.
\end{align*}
Comparing it with $x = A^{-1} u$, we have the following result.

\medskip

\begin{theorem}
The inverse matrix $A^{-1}$ of an invertible matrix $A$ has the form
\begin{align*}
    \left(A^{-1}\right)_{kj} = (-1)^{j+k} \frac{\det A_{jk}}{\det A}.
\end{align*}
\end{theorem}

\medskip

\section{Trace of A Matrix}
\begin{definition}
The trace of a square matrix $A$, denoted by $\tr A$, is the sum of all diagonal entries:
\begin{align*}
    \tr A = \sum^n_{i=1} a_{ii}.
\end{align*}
\end{definition}

\medskip

\begin{theorem}
~\begin{enumerate}[label=(\roman*)]
    \item Trace is a linear functional on matrices.
    \item Trace is commutative: $\tr AB = \tr BA$.
\end{enumerate}
\end{theorem}
\begin{proof}
The proof is obvious.
\end{proof}

\medskip

\begin{definition}
Let $A$ be an $n\times n$ matrix, then we have 
\begin{align*}
    \tr AA^T = \sum^n_{i=1}(a_{ii})^2
\end{align*}
and the Euclidean norm (or Hilbert-Schmidt norm) of matrix $A$ is defined by \begin{align*}
    \|A\| = \sqrt{\tr AA^T} = \sqrt{\sum^n_{i=1}(a_{ii})^2}.
\end{align*}
\end{definition}

\medskip

\begin{theorem}
Similar matrices have the same trace and determinant.
\end{theorem}
\begin{proof}
Assume $A$ and $B$ are similar, then there exists an invertible matrix $S$ such that $A = SBS^{-1}$.
\begin{enumerate}[label=(\arabic*)]
    \item $\tr A = \tr SBS^{-1} =\tr SS^{-1}B = \tr B$.
    \item $\det A = \det SBS^{-1} = \det S\cdot \det B \cdot \det S^{-1} = \det I\cdot \det B = \det B$.
\end{enumerate}
\end{proof}

\begin{remark}
Let $A,B,C,D$ be $n\times n$ matrices, in general, the following equations do not hold
\begin{align*}
    \det \left(
    \begin{matrix}
        A & B\\
        C & D
    \end{matrix}
    \right) & = \det A \det D - \det C \det B \\
    \det \left(
    \begin{matrix}
        A & B\\
        C & D
    \end{matrix}
    \right)& = \det(AD - CB).
\end{align*}
\end{remark}

\medskip

\begin{theorem}
Let $A,B,C,D$ be $n\times n$ matrices and $AC=CA$, then
\begin{align*}
    \det \left(
    \begin{matrix}
        A & B\\
        C & D
    \end{matrix}
    \right) & = \det (AD - CB).
\end{align*}
\end{theorem}
\begin{proof}
~\begin{enumerate}[label=(\arabic*)]
    \item If $\det A \neq 0$, then we have
    \begin{align*}
    \left(\begin{matrix}
        A & B\\
        C & D
    \end{matrix}\right) 
    \left(\begin{matrix}
        I & -A^{-1}B\\
        0 & I
    \end{matrix}\right)  = 
    \left(\begin{matrix}
        A & 0\\
        C & D-CA^{-1}B
    \end{matrix}\right).
    \end{align*}
    Thus, we can have 
    \begin{align*}
    \det \left(\begin{matrix}
        A & B\\
        C & D
    \end{matrix}\right) & = \det A \det (D-CA^{-1}B) \\
    & = \det (AD - ACA^{-1}B) \\
    & = \det (AD - CAA^{-1}B) \\
    & = \det (AD - CB).
    \end{align*}
    \item If $A$ is singular, then there exists $\varepsilon\to 0$, such that, $\det A_k = \det (A+\varepsilon I) \neq 0$. Thus, we have $A_k C = C A_k$ and then
    \begin{align*}
        \det \left(\begin{matrix}
        A_k & B\\
        C & D
    \end{matrix}\right) = \det (A_k D - CB) \overset{k\to\infty}{=} \det (AD - CB).
    \end{align*}
\end{enumerate} 
\end{proof}

\begin{remark}
Similar to the theorem above, we can have following results, that:
\begin{align*}
    \det \left(\begin{matrix}
        A & B\\
        C & D
    \end{matrix}\right) = \begin{cases}
    \det (AD - BC),\, {\rm if}\,\, CD = DC; \\
    \det (DA - CB),\, {\rm if}\,\, AB = BA; \\
    \det (DA - BC),\, {\rm if}\,\, BD = DB; \\
    \det (AD - CB),\, {\rm if}\,\, AC = CA.
    \end{cases}
\end{align*}
\end{remark}

\medskip

\begin{theorem}
Let $A,B$ be $n\times n$ matrices, then $\det (I-AB) = \det (I-BA)$.
\end{theorem}
\begin{proof}
~\begin{enumerate}[label=(\arabic*)]
    \item If $\det A \neq 0$, then we have
    \begin{align*}
        \det (I-AB) & = \det A \left(A^{-1} - B\right) \\
        & = \det A \det \left(A^{-1} - B\right) \\
        & = \det \left(A^{-1} - B\right) \det A \\
        & = \det \left(A^{-1} - B\right)A \\
        & = \left(I - BA\right).
    \end{align*}
    \item If $\det A = 0$, then we can approximate $A_k = A + \varepsilon I$ such that $\det A_k \neq 0$ and let $\varepsilon\to 0$.
\end{enumerate}
\end{proof}

\begin{remark}
In general, $\det (A - BC) \neq \det (A - CB)$.
\end{remark}

\medskip

\section{Complex Matrix}
Let $T\in\mathscr{L}(X,X)$, where $X$ is a complex linear space and $\dim X = n$. With chosen basis of $X$, $T$ can be represented by a matrix $A$. For a complex $n\times n$ matrix $A = (a_1,\cdots,a_n)$, we have $\det A = D(a_1,\cdots,a_n)$.
\begin{remark}
In general, $\det (A+iB)\neq \det A + i \det B$, for $A, B$ being real matrices.
\end{remark}

\medskip

\begin{theorem}
Let $A,B$ be real matrices. Then $A,B$ are similar as real matrices is equivalent to that they are similar as complex matrices, i.e., $A \overset{\mathbb{R}}{\sim} B\iff A \overset{\mathbb{C}}{\sim} B$.
\end{theorem}
\begin{proof}
~\begin{enumerate}[label=(\arabic*)]
    \item ($\Rightarrow$) This is trivial.
    \item ($\Leftarrow$) If $A \overset{\mathbb{C}}{\sim} B$, then there exists a matrix $M = P + iQ$, where $P,Q$ are real matrices, such that $B = MAM^{-1}$. Then we have
    \begin{align*}
        BM & = MA \\
        \Rightarrow B(P + iQ) & = (P + iQ)A \\
        \Rightarrow BP + i BQ & = PA + i QA
    \end{align*}
    which implies $BP=PA$ and $BQ = QA$. If either $P,Q$ are nonsingular, then we have $A \overset{\mathbb{R}}{\sim} B$.
    
    Consider $M_t = P + tQ$, where $t$ can be real or complex. Then $\det (P+tQ)$ is a polynomial in $t$. And $\det M_i \neq 0$, then there exists $t\in\mathbb{R}$ such that $\det (P+tQ)\neq 0$. Since $BM_t = M_t A$, then $B = M_t A M_t^{-1}$. Thus, $A \overset{\mathbb{R}}{\sim} B$.
\end{enumerate}
\end{proof}

\medskip

Next we discuss the determinant of some special matrices.

\begin{example}[Vandermonde matrix]
Let $n\geq 2$, and $a_1, \cdots, a_n$ are scalars, $n\times n$ Vandermonde matrix is defined as following
\begin{align*}
    V(a_1,\cdots,a_n) = \left(\begin{matrix}
        1 & 1 & \cdots & 1\\
        a_1 & a_2 \cdots & a_n \\
        a_1^2 & a_2^2 & \cdots & a_n^2 \\
        \vdots & \vdots & \ddots & \vdots \\
        a_1^{n-1} & a_2^{n-1} & \cdots & a_n^{n-1}
    \end{matrix}\right)
\end{align*}
Then, $\det V(a_1,\cdots,a_n) = \prod_{i < j}(a_j - a_i)$.
\end{example}

\medskip

\begin{example}[Cauchy matrix]
Given $2n$ numbers, $a_k, b_k, 1\leq k\leq n$, such that $a_i + b_i \neq 0$ for all $i,j$. The Cauchy matrix is defined as following
\begin{align*}
    C(a_1,\cdots,a_n, b_1,\cdots,b_n) = \left(\frac{1}{a_i + b_j}\right)_{n\times n}
\end{align*}
where $c_{ij} = \frac{1}{a_i + b_j}$. Then, the determinant of $C$ is 
\begin{align*}
    \det C = \frac{\prod_{j>i}(a_j - a_i)\prod_{j>i}(b_j - b_i)}{\prod_{i,j}(a_i + b_j)}.
\end{align*}
\end{example}

\medskip

\section{Exercises}
\begin{exercise}
Let $A,B,C$ be $n\times n$ matrices satisfying $AB=BA$. Show that
$$
\det\left(  A+BC\right)  =\det\left(  A+CB\right).
$$
\end{exercise}
\begin{proof}
~\begin{enumerate}[label=(\arabic*)]
    \item If $B$ is invertible, since $AB=BA$, then we have $A=B^{-1}AB$. Then we have
    \begin{align*}
        \det (A+BC) &= \det (B^{-1}(A+BC)B) \\
        & = \det(B^{-1}AB+CB) \\
        & = \det(A+CB).
    \end{align*}
    \item If $B$ is not invertible. We can set a new matrix $M = \begin{pmatrix}
    C & -I \\
    A & B
    \end{pmatrix}$, and we can solve for the determinant of this matrix. Since $AB=BA$, then $\det(M)=\det(CB-(-I)A)=\det(CB+A)$. Also, we have $-IB=B(-I)$, then the determinant can be presented as $\det (M)=\det(BC-(-I)A)=\det(BC+A)$. Then we have $\det(A+BC)=\det(A+CB)$. 
\end{enumerate}
\end{proof}

\begin{remark}
In (2), we used if $AB=BC$, then $\det(M)=\det(CB-(-I)A)$. We should give proper proof to this. Suppose matrix $M = \begin{pmatrix}
P & Q \\
R & S
\end{pmatrix}$ and we have $RS=SR$. Then, if $S$ is invertible, we have 
\begin{align*}
    \det \begin{pmatrix}
    P & Q \\
    R & S
    \end{pmatrix}
    = &\det 
    \begin{pmatrix}
    P-RS^{-1}Q & 0 \\
    R & S
    \end{pmatrix} \\
    & = \det(PS-RS^{-1}QS) \\
    & = \det(PS-RSS^{-1}Q) \\
    & = \det(PS-RQ)
\end{align*}
If $S$ is not invertibe, then there exists $\varepsilon_k \rightarrow 0$ such that $\det S_k = \det(B+\varepsilon_k I) \neq 0$ and $S_k R = R S_k$. Then $\det \begin{pmatrix}
P & Q \\
R & S_k
\end{pmatrix} = \det (PS_k-QR)$. Taking $k\rightarrow \infty$ will prove this case. The proof is complete. Similarly, we can prove that if $QS=SQ$, then $\det M = \det(SP-QR)$.
\end{remark}

\medskip

\begin{exercise}
Let $A,B,C$ be $n\times n$ matrices. Is it always true that
$$
\det\left(  A+BC\right)  =\det\left(  A+CB\right)  ?
$$
Prove or find a counter example.
\end{exercise}
\begin{proof}
In general, it is not true. Take $A = \begin{pmatrix}
    1 & 0 \\
    0 & 10
    \end{pmatrix},
    B = 
    \begin{pmatrix}
    2 & 3 \\
    4 & 5
    \end{pmatrix}$
and $C = \begin{pmatrix}
    0 & 1 \\
    1 & 6
    \end{pmatrix}$.
Then we have $\det(A+BC) = 76$ and $\det(A+CB) = 85$.
\end{proof}

\medskip

\begin{exercise}
Let $n\geq2$. Given $\left(  2n-1\right)  $ scalars $x_{1}%
,\cdots,x_{n-1}$ and $y_{1},\cdots,y_{n}$, we can define an $n\times n$ matrix
$A=\left(  a_{ij}\right)  $ such that%
\begin{align*}
a_{ij} &  =x_{j}\text{ if }i>j,\\
a_{ij} &  =y_{j}\text{ if }i\leq j.
\end{align*}
Show that%
$$
\det A=y_{n}%
%TCIMACRO{\dprod \limits_{k=1}^{n-1}}%
%BeginExpansion
{\displaystyle\prod\limits_{k=1}^{n-1}}
%EndExpansion
\left(  y_{k}-x_{k}\right).
$$
\end{exercise}
\begin{proof}
We can know that $A$ has the form 
\begin{align*}
    A = \begin{pmatrix}
    y_1 & y_2 & y_3 & \cdots & y_n \\
    x_1 & y_2 & y_3 & \cdots & y_n \\
    \vdots & \vdots & \vdots & \ddots & \vdots \\
    x_1 & x_2 & x_3 & \cdots & y_n
    \end{pmatrix}
\end{align*}
We can do elementary row operations that starting from the first row, and then apply $\text{row}_i=\text{row}_i+(-1)\text{row}_{i+1}$. Then we get new matrix
\begin{align*}
    A = \begin{pmatrix}
    y_1-x_1 & 0 & 0 & \cdots & 0 \\
    0 & y_2-x_2 & 0 & \cdots & 0 \\
    \vdots & \vdots & \vdots & \ddots & \vdots \\
    x_1 & x_2 & x_3 & \cdots & y_n
    \end{pmatrix}
\end{align*}
Then it is obvious that $\det(A)=y_n \prod_{k=1}^{n-1} (y_k-x_k)$.
\end{proof}

\medskip

\chapter{Spectral Theory}

\section{Eigenvalues and Eigenvectors}

\begin{definition}
Let $A$ be an $n\times n$ matrix. Suppose that for a nonzero vector $v$ and a scalar number $\lambda$, such that 
\begin{align*}
    Av = \lambda v
\end{align*}
then $\lambda$ is called an eigenvalue of $A$ and $v$ an eigenvector of $A$ corresponding to $\lambda$.
\end{definition}

Let $v$ be an eigenvector of $A$ corresponding to $\lambda$, we have, for any positive integer $k$, 
$$A^k v = \lambda^k v.$$
and more generally, for any polynomial $p$, we have 
$$p(A)v = p(\lambda)v.$$

\medskip

\begin{theorem}
$\lambda$ is an eigenvalue of $A$ if and only if $\det (\lambda I-A) = 0$. The polynomial
$$p_A(\lambda) = \det (\lambda I-A)$$
is called the characteristic polynomial of $A$.
\end{theorem}

\medskip

\begin{theorem}
Eigenvectors of a matrix $A$ corresponding to distinct eigenvalues
are linearly independent.
\end{theorem}
\begin{proof}
Let $\lambda_k, 1\leq k \leq n$ be $n$ distinct eigenvalues and $v_k, 1\leq k \leq n$ be corresponding eigenvectors. Now we prove it by induction.
\begin{enumerate}[label=(\arabic*)]
    \item When $k = 1$, the theorem holds.
    \item Suppose it holds for $k = N$. When $k = N+1$, suppose 
    \begin{align*}
        \sum^{N+1}_{k=1}c_k v_k = 0,
    \end{align*}
    then we have
    \begin{align*}
        \sum^{N+1}_{k=1}c_k \lambda_{N+1} v_k = 0.
    \end{align*}
    Applying $A$ to both sides of the first equation above, then we have
    \begin{align*}
        & \sum^{N+1}_{k=1}c_k \lambda_k v_k = 0 = \sum^{N+1}_{k=1}c_k \lambda_{N+1} v_k \\
        \Rightarrow & \sum^{N+1}_{k=1}c_k (\lambda_k  - \lambda_{N+1}) v_k = 0 \\
        \Rightarrow & c_k (\lambda_k  - \lambda_{N+1}) = 0
    \end{align*}
    and since $c_k = 0, 1\leq k\leq n$, we have $c_{N+1} = 0$. Thus, the theorem holds for $N+1$. 
\end{enumerate}
\end{proof}

\medskip

\begin{corollary}
If the characteristic polynomial $p_A$ of an $n\times n$ matrix $A$ has $n$ distinct roots, then $A$ has a basis formed by $n$ linearly independent eigenvectors.
\end{corollary}

\medskip

\begin{corollary}
If $A$ has $n$ distinct eigenvalues, then $A$ is diagonalizable in the sense that $A$ is similar to a diagonal matrix.
\end{corollary}
\begin{proof}
Let $\lambda_k,1\leq k\leq n$ be $n$ distinct eigenvalues of $A$, with corresponding eigenvectors $v_k,1\leq k\leq n$ such that $Av_k = \lambda_k v_k,1\leq k\leq n$. Let $S = (v_1, v_2,\cdots, v_n)$, then we have
\begin{align*}
    AS = S \begin{pmatrix}
    \lambda_1 &  &  &  \\
     & \lambda_2 &  & \\
     &   & \ddots &  \\
     &  &  & \lambda_n
    \end{pmatrix}
\end{align*}
which implies $A = S\Lambda S^{-1}$.
\end{proof}

\medskip

\begin{theorem}
Let $\lambda_k,1\leq k\leq n$ be eigenvalues of $A$, with the same multiplicity they have as roots of the characteristic equation of $A$. Then 
\begin{align*}
    \det A = \sum^n_{k=1}\lambda_k \,\,\text{and}\,\, \tr A = \sum^n_{k=1}\lambda_k.
\end{align*}
\end{theorem}
\begin{proof}
We claim that $\lambda_1,\lambda_n,\cdots,\lambda_n$ are $n$ roots of polynomial, which has the following form
\begin{align*}
    p_A(\lambda) & = \det (\lambda I - A) \\
    & = \sum_p \sigma(p) \prod^n_{k=1} \left(\lambda \delta_{p_k k} - a_{p_k k}\right) \\
    & = \lambda^n - \left(\tr A \right)\lambda^{n-1} +\cdots + (-1)^n \prod^n_{k=1}\lambda_k.
\end{align*}
Then we have
\begin{enumerate}[label=(\arabic*)]
    \item Let $\lambda = 0$, then we have $\det (-A) = (-1)^n \prod^n_{k=1}\lambda_k$, which implies $\det A = \prod^n_{k=1}\lambda_k$.
    \item According to elementary algebra, the polynomial $p_A$ can be written as 
    \begin{align*}
        p_A = \prod^n_{k=1} (\lambda - \lambda_k)
    \end{align*}
    which implies the coefficient of $\lambda^{n-1}$ is $ - \sum^n_{k=1}\lambda_k$. Thus, we have $\tr A = \sum^n_{k=1}\lambda_k$.
\end{enumerate}
\end{proof}

\medskip
\section{Spectral Mapping Theorem}
\begin{theorem}[Spectral Mapping Theorem]
~\begin{enumerate}[label=(\alph*)]
    \item Let $q$ be any polynomial, $A$ a square matrix, $\lambda$ an eigenvalue of $A$. Then $q(\lambda)$ is an eigenvalue of $q(A)$.
    \item Every eigenvalue of $q(A)$ is of the form $q(\lambda)$, where $\lambda$ is an eigenvalue of $A$.
\end{enumerate}
\end{theorem}
\begin{proof}
~\begin{enumerate}[label=(\alph*)]
    \item We have $Av = \lambda v$, which implies $q(A)v = q(\lambda)v$. Indeed, we have
    \begin{align*}
        q(A) = \sum^m_{k=0} c_k A^k \Rightarrow q(A)v = \sum^m_{k=0} c_k \lambda^k v = q(\lambda)v.
    \end{align*}
    \item Let $\mu$ be an eigenvalue of $q(A)$, then we have $\det (\mu I - q(A)) = 0$. Suppose $$q(\lambda) - \mu = c\prod^m_{i=1}(\lambda - \lambda_i),$$ 
    then substituting by $A$, we have $$\prod^m_{i=1} \det (\lambda_i I - A) = 0.$$
    Thus, for some $\lambda_i$, $\det (\lambda_i I - A) = 0$, which implies $\mu = q(\lambda_i)$, where $\lambda_i$ is an eigenvalue of $A$. 
\end{enumerate}
\end{proof}

\begin{remark}
Let $p_A = \det(\lambda I - A)$, then every eigenvalue of $p_A(A)$ is zero.
\end{remark}

\medskip

\begin{theorem}[Cayley-Hamilton]
Every matrix $A$ satisfies its own characteristic equation, i.e.,
$$p_A(A) = 0.$$
\end{theorem}
\begin{proof}
Let $Q(s) = sI - A$ and $P(s)$ defined as the matrix of cofactors of $Q(s)$, i.e., 
$$P_{ij}(s) = (-1)^{i+j}D_{ji}(s)$$
where $D_{ij}(s)$ is the determinant of $(j,i)$th minor of $Q(s)$. Then we have 
$$P(s)Q(s) = \det (Q(s)) I = p_A(s)I.$$
Since the coefficients of $Q$ commutes with $A$, we have 
\begin{align*}
    P(A)Q(A) = p_A(A)I = 0
\end{align*}
hence, $p_A(A) = 0$.
\end{proof}

\medskip

\begin{lemma}
Let $P(s), Q(s)$ and $R(s)$ be polynomials in $s$ with $n\times n$ matrices $P_k, Q_k, R_s$ as coefficients. Suppose
\begin{align*}
    P(s) = \sum P_k s^k, Q(s) = \sum Q_k s^k, R(s) = \sum R_k s^k
\end{align*}
and 
$$P(s)Q(s) = R(s).$$
Also, $A$ commutes wit each $Q_k$, then we have 
$$P(A)Q(A) = R(A).$$
\end{lemma}
\begin{proof}
Since $P(s)Q(s) = R(s)$, we have
\begin{align*}
    \sum P_k s^k \sum Q_k s^k = \sum R_k s^k
\end{align*}
which implies 
\begin{align*}
    R_k = \sum_{i+j=k} P_i Q_j.
\end{align*}
Then, substituting by $A$, we have
\begin{align*}
    P(A)Q(A) & = \sum P_k A^k \sum Q_k A^k \\
    & = \sum_{i,j} P_i A^i Q_j A^j \\
    & = \sum_{i,j} P_i Q_j A^{i+j} \\
    & = \sum_{i+j=k} P_i Q_j A^{i+j} = \sum R_k A^k = R(A).
\end{align*}
\end{proof}

\medskip

\section{Generalized Eigenvectors and Spectral Theorem}
\begin{definition}\label{spectral_theorem}
A nonzero vector $u$ is said to be a generalized eigenvector of $A$ corresponding to eigenvalue $\lambda$ if 
$$\left(A - \lambda I\right)^m u = 0$$
for some $m\in\mathbb{N}$.
\end{definition}

\medskip

\begin{theorem}[Spectral Theorem]
Every vector in $\mathbb{C}^n$ can be written as a sum of eigenvectors of $A$, genuine or generalized.
\end{theorem}
\begin{proof}
Let $x$ be any vector, then $p_A(A)x = 0$. We factor polynomial as $$p_A(\lambda) = \prod^J_{j=1}\left(\lambda - \lambda_j\right)^{m_j}$$ where $\lambda_j$ are distinct eigenvalues of $A$. Then we have
\begin{align*}
    p_A(A) = \prod^J_{j=1}\left(A - \lambda_j\right)^{m_j} = 0 \\
    \Rightarrow \prod^J_{j=1}\left(A - \lambda_j\right)^{m_j}x = 0.
\end{align*}

Let $p_j = \left(\lambda - \lambda_j\right)^{m_j}$, then $x\in N_{p_1(A)p_2(A)\cdots p_J(A)}$, i.e., $x$ belongs to the null space of $p_1p_2\cdots p_J(A)$. We claim: 
\begin{align*}
    N_{p_1p_2\cdots p_J(A)} = \oplus^J_{i=1} N_{p_i(A)}.
\end{align*}
If this is true, then $x = \sum^J_{i=1}x_i, x_i\in N_{p_j(A)}$. Then we need a lemma.

\begin{lemma}
Let $p, q$ be a pair of polynomials, with complex coefficients, and $p,q$ have no common zeros. Then, we have
\begin{enumerate}[label=(\arabic*)]
    \item There exist two polynomials $a,b$, such that $ap+bq = 1$.
    \item Let $A$ be a square matrix, then 
    $$N_{p(A)q(A)} = N_{p(A)} \oplus N_{q(A)}.$$
    \item Let $P_k k = 1, \cdots, m$ be polynomials and they have no common zeros, then 
    $$N_{p_1(A)\cdots p_m(A)} = N_{p_1(A)} \oplus\cdots\oplus N_{p_m(A)}.$$
\end{enumerate}
\end{lemma}
\begin{proof}
~\begin{enumerate}[label=(\arabic*)]
    \item Let $\mathcal{P} = \{ap+bq\}$, where $a,b$ are two polynomials, and let $d$ be a nonzero polynomial in $P$ with lowest degree. 
    
    First, we claim that $d$ divides both $p$ and $q$.Indeed, if not, then the division algorithm yields a remainder, i.e.,
    $$r = p - md.$$
    where the degree of $r$ is less than that of $d$. Since $p$ and $d$ belong to $\mathcal{P}$, then $r\in \mathcal{P}$, which is a contradiction. 
    
    Second, we claim that $d$ has degree zero. Suppose not, then by the fundamental theorem of algebra, $d$ would have a root. Since $d$ divides $p$ and $q$, and $p$ and $q$ have no common zeros, $d$ is a nonzero constant. Thus, $1\in \mathcal{P}$.
    \item From (1), there are two polynomials $a$ and $b$ such that 
    $$a(A)p(A) + b(A)q(A) = I.$$
    For any $x$, we have
    $$x = a(A)p(A)x + b(A)q(A)x \overset{\Delta}{=} x_1 + x_2 $$
    and it is easy to see that if $x\in N_{p(A)q(A)}$, then $x\in N_{p(A)}$ and $x\in N_{q(A)}$. Also, suppose $x\in N_{p(A)}\cap N_{q(A)}$, the above equation implies 
    $$x = a(A)p(A)x + b(A)q(A)x = 0,$$
    Hence, $N_{p(A)q(A)} = N_{p(A)} \oplus N_{q(A)}$.
    \item The third argument follows naturally.
\end{enumerate}
\end{proof}
Now the proof of the theorem is completed.
\end{proof}

\medskip

\section{Minimal Polynomial}

We denote by $\mathcal{P}_A$ the set of all polynomials such that $p(A) = 0$. It is obvious $\mathcal{P}_A$ forms a linear space. Denote by $m = m_A$ a polynomial of smallest degree in $\mathcal{P}_A$, and we normalized $m$ to have coefficient $1$ at its highest degree. 

Now we claim that any $p\in\mathcal{P}_A$ is a multiple of $m$. Indeed, we can write $p = qm +r$, where the degree of $r$ is less that that of $m$. Then we have 
\begin{align*}
    r(A) = p(A) - q(A)m(A) = 0
\end{align*}
then $r\in\mathcal{P}_A$, hence, $r = 0$, which proved the argument. And this polynomial $m$ is called the \emph{minimal polynomial} of $A$.

Now we consider generalized eigenvector. We denote by $N_m = N_m(\lambda)$ the null space of $(A - \lambda I)^m$. The subspaces
$N_m$, consist of generalized eigenvectors; they are indexed increasingly, i.e.,
\begin{align*}
    N_1\subset N_2 \subset N_3 \subset \cdots \subset \mathbb{C}^n.
\end{align*}
We denote by $d = d(\lambda)$ the smallest index such that
\begin{align*}
    N_d & = N_{d+k}, k\geq 1 \\
    N_d & \neq N_{d-1}
\end{align*}
and $d$ is called the \emph{index} of the eigenvalue $\lambda$.

\medskip

\begin{remark}
$A$ maps $N_d$ into itself, i.e., $N_d$ is an invariant subspace under the matrix $A$.
\end{remark}
\begin{proof}
If $v\in N_d$, then $(A - \lambda I)^d v = 0$. Then, we have $(A - \lambda I)^d Av = A(A - \lambda I)^d v = 0$. Thus, $Av \in N_d$.
\end{proof}

\medskip

\begin{theorem}
Let $\lambda_1,\cdots,\lambda_k$ be distinct eigenvalues of $A$, whose index is $d\left(\lambda_j\right) = d_j, 1\leq j \leq k$. Then,
\begin{enumerate}[label=(\arabic*)]
    \item $\mathbb{C}^n = \oplus^k_{j=1}N_{d_j}(\lambda_j).$
    \item The minimal polynomial is $m_A = \prod^k_{j=1}\left(\lambda - \lambda_j \right)^{d_j}$.
\end{enumerate}
\end{theorem}
\begin{proof}
~\begin{enumerate}[label=(\arabic*)]
    \item $\mathbb{C}^n$ is the span of generalized eigenvectors and others follows from spectral theorem.
    \item For any $x\in\mathbb{C}^n$, we have $x = \sum^k_{j=1}x_j, x_j\in N_{d_j}(\lambda_j)$. Then, we have 
    \begin{align*}
        \prod^k_{j=1} \left(A - \lambda_j I\right)^{d_j}x = \sum^k_{j=1}\left(\prod^k_{j=1} \left(A - \lambda_j I\right)^{d_j} \right)x_j = 0
    \end{align*}
    Hence, we have 
    \begin{align*}
        \prod^k_{j=1} \left(A - \lambda_j I\right)^{d_j} = 0.
    \end{align*}
    Thus, we have $m(\lambda) = \prod^k_{j=1} \left(\lambda - \lambda_j I\right)^{d_j} = 0$. Since $m(A) = 0$, then $m\in \mathcal{P}_A$. Then we can have that $m$ is a multiple of $m_A$. Suppose $m_A = \prod^k_{j=1} \left(\lambda - \lambda_j I\right)^{e_j}, e_j\leq d_j$. Then, we have 
    \begin{align*}
        \mathbb{C}^n = N_{m_A(A)} = \oplus^k_{j=1}N_{(A-\lambda_j I)^{e_j}} = \oplus^k_{j=1}N_{(A-\lambda_j I)^{d_j}}
    \end{align*}
    which implies $e_j = d_j, 1\leq j \leq k$.
\end{enumerate}
\end{proof}

\medskip

\begin{theorem}
Suppose $A$ and $B$ similar, then $A, B$ has the same distinct eigenvalues $\lambda_1, \cdots, \lambda_k$. Furthermore, the null space $N_{(A-\lambda_j I)^m}$ and $N_{(B-\lambda_j I)^m}$ has the same dimension for $1\leq j \leq k, m \geq 1$.
\end{theorem}
\begin{proof}
Since $A$ and $B$ similar, then there exists nonsingular $S$ such that $A = SBS^{-1}$. Then we have 
\begin{align*}
    (A - \lambda I)^m = S(A - \lambda I)^m S^{-1}.
\end{align*}
If $v\in N_{(A-\lambda_j I)^m}$, then we have $S^{-1}v\in N_{(B-\lambda_j I)^m}$. Thus, $\dim N_{(A-\lambda_j I)^m} = \dim N_{(B-\lambda_j I)^m}$.
\end{proof}

\begin{remark}
~\begin{enumerate}[label=(\arabic*)]
    \item $A - \lambda I$ maps $N_{i+1}(\lambda)$ into $N_{i}(\lambda)$, where $N_{j}(\lambda) = N_{(A - \lambda I)^j}$.
    \item $A - \lambda I$ defines a map from $N_{i+1}/N_i$ to $N_i/N_{i-1}$, for $i\geq 1$, and $N_0 = \{0\}$.
    \begin{proof}
    For all $x,y\in N_{i+1}$ and $x-y\in N_{i}$, we have $(A - \lambda I)x, (A - \lambda I)y \in N_{i}$ and $(A - \lambda I)x - (A - \lambda I)y \in N_{i-1}$.
    \end{proof}
\end{enumerate}
\end{remark}

\medskip

\begin{lemma}
The map $$A-\lambda I: N_{i+1}/N_i\to N_i/N_{i-1}$$ is one-to-one. Hence, $$\dim N_{i+1}/N_{i} \leq \dim N_{i}/N_{i-1}.$$
\end{lemma}
\begin{proof}
Let $B = A - \lambda I$, if $\left\{B\{x\}_{N_{i+1}/N_{i}} \right\}_{N_{i}/N_{i-1}} = \{0\}$, then $Bx \in N_{i-1}$, which implies $\{x\}\in N_i$ and $\{x\}_{N_{i+1}/N_{i}} = \{0\}_{N_{i+1}/N_{i}}$. Thus, $A - \lambda I$ is one-to-one. 
\end{proof}

\medskip

\section{Jordan Canonical Form}
\subsection{Proof of Jordan Canonical Form}

We want to construct a basis for $N_{d_j}(\lambda_j)$. For simplicity, assume $\lambda_j = 0$ and $d_j = 0$. Also, we have $N_1\subset N_2 \subset \cdots \subset N_d$, $A:N_{i+1}\to N_{i}, i\geq 1$ and $A:N_{i+1}/N_{i}\to N_{i}/N_{i-1}$. Now we preset how to construct Jordan Canonical form.
\begin{enumerate}[label=Step \Roman*:]
    \item Let $l_0 = \dim (N_d/N_{d-1}) \geq 1$. Let $x_1,\cdots,x_{l_0}$ be vectors such that $\{x_1\}, \cdots \{x_{l_0}\} \in N_d$ form a basis of $N_d/N_{d-1}$.
    \item Let $l_1  = \dim (N_{d-1}/N_{d-2}) \geq l_0$, then $\{Ax_1\},\cdots, \{Ax_{l_0}\} \in N_{d-1}$ are linearly independent. If $l_1 > l_0$, we can pick $x_{l_0+1}, \cdots, x_{l_1}$ such that $\{Ax_1\},\cdots, \{Ax_{l_0}\}, x_{l_0+1}, \cdots, x_{l_1}$ form a basis of $N_{d-1}/N_{d-2}$.
    \item Continue this process until $N_1$. Let $l_{d-1} = \dim N_1$ and $A:N_2\to N_1$ and add vectors $x_{l_{d-2}+1},\cdots,x_{l_{d-1}}$, and the rest is the similar. We thus constructed a basis of $N_d$.
    \item We present the vectors in a list as below:
    \begin{align*}
        \begin{matrix}
            x_1 & Ax_1 & \cdots & & A^{d-1}x_1 \\
            \vdots &  &  & & \vdots \\
            x_{l_0} & Ax_{l_0} & \cdots & & A^{d-1}x_{l_0} \\
            x_{l_0+1} & Ax_{l_0+1} & \cdots & A^{d-2}x_{l_0+1} & \\
            \vdots &  &  &  & \\
            x_{l_1} & Ax_{l_1} & \cdots & A^{d-2}x_{l_1} & \\
            \vdots &  &  &  & \\
            x_{l_{d-2}+1} &  &  &  & \\
            \vdots &  &  &  & \\
            x_{l_{d-1}} &  &  &  & 
        \end{matrix}
    \end{align*}
\end{enumerate}
Also, we have 
\begin{align*}
    \dim N_d & = \dim N_1 + \dim N_2/N_1 + \cdots + \dim N_d/N_{d-1} \\
    & = l_{d-1} + l_{d-2} + \cdots + l_0. \\
\end{align*}
and we claim that these vectors are linearly independent. Under this basis, we have 
\begin{align*}
    N_d = \oplus^{l_{d-1}}_{k=1} M_K
\end{align*}
where $M_k$ is the span of vectors in the $k$th row. Under the basis of $M_k$, $A$ has the representation
\begin{align*}
    J_m = \begin{pmatrix}
    0 & 1 &  &  \\
     &\ddots & \ddots & \\
     &   & \ddots & 1 \\
     &  &  & 0
    \end{pmatrix}
\end{align*}
which is called a \emph{Jordan block} where $J_m(i,j) = 1$ for $j = i+1$ and $J_m(i,j) = 0$ otherwise.

\medskip

\begin{theorem}
Any matrix $A$ is similar to its Jordan canonical form which consists diagonal blocks of the form 
\begin{align*}
    J_m = \begin{pmatrix}
    \lambda & 1 &  &  \\
     &\ddots & \ddots & \\
     &   & \ddots & 1 \\
     &  &  & \lambda
    \end{pmatrix}
\end{align*}
where $\lambda$ is the eigenvalue of $A$.
\end{theorem}

\medskip

\subsection{Another Proof}
Now we present more details about Jordan canonical form from other materials\cite{8}. We start from the beginning and consider nilpotent operator.
\begin{example}
Let $N\in\mathcal{L}(\mathbb{R}^4,\mathbb{R}^4)$ be the nilpotent operator defined by
$$N(x_1,x_2,x_3,x_4) = (0,x_1,x_2,x_3).$$
If $x = (1,0,0,0)$, then $\{N^3x, N^2x, Nx,x\}$ is a basis for $\mathbb{R}^4$. We denote by $M$ the matrix spanned by $\{N^3x, N^2x, Nx,x\}$. The matrix of $N$ with respect to this basis is 
$$
J = M^{-1}NM = \begin{pmatrix}
    0 & 0 & 0 & 1 \\
    0 & 0 & 1 & 0\\
    0 & 1 & 0 & 0 \\
    1 & 0 & 0 & 0
\end{pmatrix}\begin{pmatrix}
    0 & 0 & 0 & 0 \\
    1 & 0 & 0 & 0\\
    0 & 1 & 0 & 0 \\
    0 & 0 & 1 & 0
\end{pmatrix} \begin{pmatrix}
    0 & 0 & 0 & 1 \\
    0 & 0 & 1 & 0\\
    0 & 1 & 0 & 0 \\
    1 & 0 & 0 & 0
\end{pmatrix} =  \begin{pmatrix}
    0 & 1 & 0 & 0 \\
    0 & 0 & 1 & 0\\
    0 & 0 & 0 & 1 \\
    0 & 0 & 0 & 0
\end{pmatrix}.
$$
\end{example}

\medskip

\begin{example}
Let $N\in\mathcal{L}(\mathbb{R}^6,\mathbb{R}^6)$ be the nilpotent operator defined by
$$N(x_1,x_2,x_3,x_4,x_5,x_6) = (0,x_1,x_2,0,x_4,0).$$
and there does not exist a vector $x\in\mathbb{R}^6$ such that $\{N^5x, N^4x, N^3x, N^2x, Nx, x\}$ form a basis of $\mathbb{R}^6$. If we take $v_1 = (1,0,0,0,0,0), v_2 = (0,0,0,1,0,0)$ and $v_3 = (0,0,0,0,0,1)$, then $\{N^2v_1, Nv_1, v_1, Nv_2, v_2, v_3\}$ form a basis of $\mathbb{R}^6$. The matrix of $N$ with respect to this basis is 
\begin{align*}
    J = \begin{pmatrix}
    0 & 1 & 0 & 0 & 0 & 0 \\
    0 & 0 & 1 & 0 & 0 & 0 \\
    0 & 0 & 0 & 0 & 0 & 0 \\
    0 & 0 & 0 & 0 & 1 & 0 \\
    0 & 0 & 0 & 0 & 0 & 0 \\
    0 & 0 & 0 & 0 & 0 & 0
\end{pmatrix}.
\end{align*}
\end{example}

\medskip

Next, we show that every nilpotent operator $N\in\mathcal{L}(X,X)$ behaves similarly to the examples above. Specifically, there is a finite collection of vectors $v_1, \cdots, v_n\in X$ such that there is a basis of $X$ consisting of the vectors of the form $N^k v_j \in X$ where $1\leq j \leq n$ and $k$ varies from $0$ to the largest nonnegative integer $m_j$ such that $N^{m_j}v_j\neq 0$. 

\medskip

\begin{theorem}
Suppose $N\in \mathcal{L}(X,X)$ is nilpotent. Then there exist vectors $v_1, \cdots, v_n\in X$ and nonnegative integers $m_1, \cdots, m_n$ such that 
\begin{enumerate}[label=(\arabic*)]
    \item $N^{m_1}v_1, \cdots, Nv_1, v_1, \cdots, N^{m_n}v_n, \cdots, N v_n, v_n$ is a basis of $X$.
    \item $N^{m_1+1}v_1 = N^{m_2+1}v_2 = \cdots = N^{m_n+1}v_n = 0$.
\end{enumerate}
\end{theorem}
\begin{proof}
We prove by induction and the theorem obviously holds for $\dim X = 1$, since the only nilpotent operator is $0$ and we can pick any nonzero vector as $v_1$ and $m_1 = 0$. 

Because $N$ is nilpotent, then $N$ is not injective. Thus $N$ is not surjective and hence $R_N$ is a subspace of $X$, i.e., $\dim R_N \leq \dim X$. Thus we can apply the induction to the restriction operator $N|_{R_N}\in\mathcal{L}(R_N)$. We can ignore the case where $R_N = \{0\}$, since we can pick $v_1,\cdots,v_n$ be any basis and $m_1 = \cdots = m_n = 0$. 

By induction applied to $N|_{R_N}\in\mathcal{L}(R_N)$, there exist $v_1,\cdots,v_n\in R_N$ and $m_1,\cdots,m_n$ such that 
\begin{equation}\label{basis_1}
    N^{m_1}v_1, \cdots, Nv_1, v_1, \cdots, N^{m_n}v_n, \cdots, N v_n, v_n
\end{equation}
is a basis of $R_N$ and 
\begin{align*}
    N^{m_1+1}v_1 = N^{m_2+1}v_2 = \cdots = N^{m_n+1}v_n = 0.
\end{align*}
Since $v_j\in R_N, 1\leq j \leq n$, then for any $j$, there exists a $u_j\in X$ such that $v_j = Nu_j$. Then $N^{k+1}u_j = N^k v_j$ for each $j$ and each nonnegative integer $k$. 

We claim 
\begin{equation}\label{basis_2}
    N^{m_1+1}u_1, \cdots, Nu_1, u_1, \cdots, N^{m_n+1}u_n, \cdots, N u_n, u_n
\end{equation}
is linearly independent in $X$. Indeed, suppose $$\sum^n_{j=1}\sum^{m_j+1}_{i=0}N^{i}u_j = 0$$
then we apply $N$ to both sides and we have
\begin{align*}
    \sum^n_{j=1}\sum^{m_j+1}_{i=0}c_{ij}N^{i+1}u_j = \sum^n_{j=1}\sum^{m_j+1}_{i=0}c_{ij}N^{i}v_j = 0
\end{align*}
which implies $c_{ij} = 0$ for all $i,j$.

Now we extend (\ref{basis_2}) into a basis 
\begin{equation}\label{basis_3}
    N^{m_1+1}u_1, \cdots, Nu_1, u_1, \cdots, N^{m_n+1}u_n, \cdots, N u_n, u_n, w_1,\cdots, w_p
\end{equation}
of $X$. Also, each $Nw_j$ is in the range of $N$ and hence $Nw_j$ is in the span of (\ref{basis_1}), and for each $1\leq j\leq p$, there exists $x_j$ in the span of (\ref{basis_2}) such that $Nx_j = Nw_j$. And we define 
$$u_{n+j} = w_j - x_j$$
then we have $Nu_{n+j} = 0$. Furthermore, 
\begin{equation}\label{basis_4}
    N^{m_1+1}u_1, \cdots, Nu_1, u_1, \cdots, N^{m_n+1}u_n, \cdots, N u_n, u_n, u_{n+1},\cdots, u_{n+p}
\end{equation}
form a basis of $X$ since its span contains $w_j, 1\leq j\leq p$. This basis has the required form, completing the proof.
\end{proof}

\medskip

\begin{definition}
Suppose $T\in\mathcal{L}(X,X)$. A basis of $X$ is called a Jordan basis for $T$ if, with respect to this basis, $T$ has a block diagonal representation
\begin{align*}
    T = \begin{pmatrix}
    A_1 &  & 0 \\
     & \ddots &  \\
    0 &  & A_p 
    \end{pmatrix},
\end{align*}
where each $A_j$ is an upper-triangular matrix of the form
\begin{align*}
    A_j = \begin{pmatrix}
    \lambda_j & 1 & & 0 \\
     & \ddots &  \ddots & \\
     &  & \ddots & 1 \\
    0 & & & \lambda_j
    \end{pmatrix}.
\end{align*}
and $\lambda_j$ is an eigenvalue of $T$.
\end{definition}

\medskip

\begin{theorem}
Suppose $X$ is a complex vector space. If $T\in\mathcal{L}(X,X)$, then there exists a basis of $X$ which is a Jordan basis for $T$.
\end{theorem}
\begin{proof}
First consider a nilpotent operator $N\mathcal{L}(X,X)$ and the vector $v_1,\cdots,v_n \in X$ given by previous theorem. For each $j$, $N$ maps the first vector in the list $\{N^{m_j}v_j, \cdots, Nv_j,v_j\}$ to $0$ and each vector other than $N^{m_j}v_j$ to the previous one. Then, the previous theorem gives a basis of $X$ with respect to which, $N$ has a block diagonal matrix, where each one has the form 
\begin{align*}
    \begin{pmatrix}
    0 & 1 & & 0 \\
     & \ddots &  \ddots & \\
     &  & \ddots & 1 \\
    0 & & & 0
    \end{pmatrix}.
\end{align*}
Thus, the theorem holds for nilpotent operators. 

Now suppose $T\in\mathcal{L}(X,X)$, and let $\lambda_1,\cdots,\lambda_m$ be distinct eigenvalues of $T$. Then we have the generalized eigenspace decompisition
\begin{align*}
    X = G(\lambda_1, T)\oplus \cdots \oplus G(\lambda_m, T)
\end{align*}
where each $(T-\lambda_j I)|_{G(\lambda_j, T)}$ is nilpotent. Thus, some basis of each $G(\lambda_j,T)$ is a Jordan basis for $(T-\lambda_j I)|_{G(\lambda_j, T)}$. Putting these bases together gives a basis of $X$ that is a Jordan basis for $T$.
\end{proof}

\medskip

\begin{example}
Consider 
\begin{align*}
    A = \begin{pmatrix}
    1 & -1 & 0 \\
    -1 & 4 &  -1 \\
    -4 & 13 & -3 
    \end{pmatrix}.
\end{align*}
The characteristic polynomial of $A$ is 
$$\lambda(\lambda-1)^2$$
and the generalized eigenspace corresponding to $0$ is just the ordinary eigenspace, so there will be only one single Jordan block corresponding to $0$ in the Jordan form of $A$. Moreover, this block has size $1$ since $1$ is the exponent of $\lambda$ in the characteristic (and hence in the minimal polynomial as well) polynomial of $A$.

Now we determine the dimension of the eigenspace corresponding to $\lambda = 1$, which is the dimension of the null space of
\begin{align*}
    A - I = \begin{pmatrix}
    0 & -1 & 0 \\
    -1 & 3 &  -1 \\
    -4 & 13 & -4 
    \end{pmatrix}.
\end{align*}
and row-reducing gives
\begin{align*}
    \begin{pmatrix}
    0 & -1 & 0 \\
    -1 & 3 &  -1 \\
    -4 & 13 & -4 
    \end{pmatrix} \to \begin{pmatrix}
    -1 & 3 &  -1 \\
    0 & -1 & 0 \\
    0 & 0 & 0 
    \end{pmatrix}.
\end{align*}
Thus, the dimension of the eigenspace corresponding to $\lambda = 1$ is $1$, since the null space is of dimension $1$, implying that there is only one Jordan block corresponding to $1$ in the Jordan form of $A$. Thus, the Jordan form of $A$ is
\begin{align*}
    J = \begin{pmatrix}
    0 & 0 &  0 \\
    0 & 1 & 0 \\
    0 & 0 & 1
    \end{pmatrix}.
\end{align*}
\end{example}

\medskip

\begin{example}
Consider 
\begin{align*}
    A = \begin{pmatrix}
    5 & -1 & 0 & 0 \\
    9 & -1 & 0 & 0 \\
    0 & 0 & 7 & -2 \\
    0 & 0 & 12 & -3
    \end{pmatrix}.
\end{align*}
The characteristic polynomial of $A$ is 
$$(\lambda - 2)^2(\lambda - 3)(\lambda - 1)$$
From the multiplicities, the generalized eigenspaces corresponding to $\lambda = 3$ and $\lambda = 1$ are the ordinary eigenspaces, so each of these give blocks of size $1$ in the Jordan form.

The eigenspace corresponding to $\lambda = 2$ is the null space of 
\begin{align*}
    A - 2I = \begin{pmatrix}
    3 & -1 & 0 & 0 \\
    9 & -3 & 0 & 0 \\
    0 & 0 & 5 & -2 \\
    0 & 0 & 12 & -5
    \end{pmatrix}
\end{align*}
and row-reducing gives
\begin{align*}
    \begin{pmatrix}
    3 & -1 & 0 & 0 \\
    9 & -3 & 0 & 0 \\
    0 & 0 & 5 & -2 \\
    0 & 0 & 12 & -5
    \end{pmatrix} \to
    \begin{pmatrix}
    3 & -1 & 0 & 0 \\
    0 & 0 & 0 & 0 \\
    0 & 0 & 5 & -2 \\
    0 & 0 & 0 & -1
    \end{pmatrix}.
\end{align*}
Thus, the eigenspace is of dimension $1$, which implies there is only one Jordan block in the Jordan form of $A$, with size $2\times 2$. Hence, the Jordan form of $A$ is
\begin{align*}
    J = \begin{pmatrix}
    2 & 1 & 0 & 0 \\
    0 & 1 & 0 & 0 \\
    0 & 0 & 3 & 0 \\
    0 & 0 & 0 & 1
    \end{pmatrix}.
\end{align*}

Next, we find the \emph{Jordan basis} which puts $A$ into its Jordan form. Recall that this should be a basis consisting of \emph{Jordan chains}. For the block of size $1$, the chain will be of length $1$ and consists of exactly one eigenvector for the corresponding eigenvalue. For $\lambda = 3$ and $\lambda = 1$, the corresponding eigenvectors are 
\begin{align*}
    \begin{pmatrix}
    0 \\
    0 \\
    -1 \\
    2
    \end{pmatrix}, \begin{pmatrix}
    0 \\
    0 \\
    -1 \\
    3
    \end{pmatrix}.
\end{align*}
Then we can find a eigenvector for $\lambda = 2$, which is 
\begin{align*}
    v_1 = \begin{pmatrix}
    1 \\
    3 \\
    0 \\
    0
    \end{pmatrix}
\end{align*}
and we need to find the final vector in the Jordan chain for $\lambda = 2$. And the Jordan chain has the form of $(v, (A-2I)v)$. Then we can pick 
\begin{align*}
    \begin{pmatrix}
    3 & -1 & 0 & 0 \\
    9 & -3 & 0 & 0 \\
    0 & 0 & 5 & -2 \\
    0 & 0 & 12 & -5
    \end{pmatrix} v = \begin{pmatrix}
    1 \\
    3 \\
    0 \\
    0
    \end{pmatrix}
\end{align*}
which implies 
\begin{align*}
    v = \begin{pmatrix}
    1 \\
    2 \\
    0 \\
    0
    \end{pmatrix}.
\end{align*}
Thus, the Jordan basis corresponding to $A$ is
\begin{align*}
    \begin{pmatrix}
    1 \\
    2 \\
    0 \\
    0
    \end{pmatrix}, \begin{pmatrix}
    1 \\
    3 \\
    0 \\
    0
    \end{pmatrix}, 
    \begin{pmatrix}
    0 \\
    0 \\
    -1 \\
    2
    \end{pmatrix}, \begin{pmatrix}
    0 \\
    0 \\
    -1 \\
    3
    \end{pmatrix}.
\end{align*}
\end{example}

\medskip

\begin{example}
Consider
\begin{align*}
    A = \begin{pmatrix}
    1 & -1 & -2 & 3 \\
    0 & 0 & -2 & 3 \\
    0 & 1 & 1 & -1 \\
    0 & 0 & -1 & 2
    \end{pmatrix}.
\end{align*}
The characteristic polynomial of $A$ is 
$$(\lambda - 1)^4$$
and we need to determine the dimension of the eigenspace corresponding to $1$. And, $A - I$ can reduce as following
\begin{align*}
    \begin{pmatrix}
    0 & -1 & -2 & 3 \\
    0 & -1 & -2 & 3 \\
    0 & 1 & 0 & -1 \\
    0 & 0 & -1 & 1
    \end{pmatrix} \to \begin{pmatrix}
    1 & -1 & -2 & 3 \\
    0 & 0 & -2 & 2 \\
    0 & 0 & 0 & 0 \\
    0 & 0 & 0 & 0
    \end{pmatrix}
\end{align*}
which implies there are two Jordan blocks corresponding to $1$ in the Jordan form of $A$. Then, there are two possibilities:
\begin{align*}
    \begin{pmatrix}
    1 & 1 & 0 & 0 \\
    0 & 1 & 0 & 0 \\
    0 & 0 & 1 & 1 \\
    0 & 0 & 0 & 1
    \end{pmatrix}\,\, {\rm or}\,\, \begin{pmatrix}
    1 & 1 & 0 & 0 \\
    0 & 1 & 1 & 0 \\
    0 & 0 & 1 & 0 \\
    0 & 0 & 0 & 1
    \end{pmatrix}
\end{align*}
with corresponding minimal polynomial $(\lambda - 1)^2$ or $(\lambda - 1)^3$.

To determine which it is, we need to determine the length of the Jordan chains. We start with ordinary eigenvectors
\begin{align*}
    \begin{pmatrix}
    1 \\
    0 \\
    0 \\
    0
    \end{pmatrix}, \begin{pmatrix}
    1 \\
    1 \\
    1 \\
    1
    \end{pmatrix}
\end{align*}
and each will give a Jordan chain. Consider $v$ such that
\begin{align*}
    (A - I)v = \begin{pmatrix}
    1 \\
    0 \\
    0 \\
    0
    \end{pmatrix}
\end{align*}
and we cannot find a solution for $v$, which implies this eigenvector is its own Jordan chain. Now we consider $w$ such that
\begin{align*}
    (A - I)w = \begin{pmatrix}
    1 \\
    1 \\
    1 \\
    1
    \end{pmatrix}
\end{align*}
and it gives us 
\begin{align*}
    w = \begin{pmatrix}
    1 \\
    1 \\
    -1 \\
    0
    \end{pmatrix}.
\end{align*}
Then we try to find $u$ such that
\begin{align*}
    (A - I)u = \begin{pmatrix}
    1 \\
    1 \\
    -1 \\
    0
    \end{pmatrix}
\end{align*}
and it gives us
\begin{align*}
    u = \begin{pmatrix}
    1 \\
    -1 \\
    0 \\
    0
    \end{pmatrix}.
\end{align*}
Thus, we can find a Jordan chain of length $3$:
\begin{align*}
    (u, (A-I)u,(A-I)^2u) = 
    \begin{pmatrix}
        \begin{pmatrix}1 \\ -1 \\ 0 \\ 0 \end{pmatrix}, & 
        \begin{pmatrix}1 \\ 1 \\ -1 \\ 0 \end{pmatrix}, & 
        \begin{pmatrix}1 \\ 1 \\ 1 \\ 1 \end{pmatrix}
    \end{pmatrix}.
\end{align*}
Thus, the Jordan form of $A$ is 
\begin{align*}
    J = \begin{pmatrix}
    1 & 1 & 0 & 0 \\
    0 & 1 & 1 & 0 \\
    0 & 0 & 1 & 0 \\
    0 & 0 & 0 & 1
    \end{pmatrix}.
\end{align*}
\end{example}

\medskip

\section{Commuting Maps}
\begin{lemma}
If $A,B$ have same eigenvalues $\{\lambda_j\}$ and if for each $\lambda_j$, we have
$$\dim N_m(\lambda_j) = \dim M_m(\lambda_j),$$
where $N_m(\lambda_j) = N_{(A - \lambda_j I)^m}$ and $M_m(\lambda_j) = N_{(B - \lambda_j I)^m}$, then $A, B$ are similar.
\end{lemma}
\begin{proof}
By the construction of Jordan canonical form.
\end{proof}
\begin{remark}
The inverse of this lemma is also true.
\end{remark}

\medskip

\begin{theorem}
Suppose $A, B$ are $n\times n$ matrices, such that $AB = BA$, then there is a basis of $\mathbb{C}^n$ which consists of eigenvectors and generalized eigenvectors of both $A$ and $B$.
\end{theorem}
\begin{proof}
Let $\{\lambda_j\}^K_{j=1}$ be $K$ distinct eigenvalues of $A$, then 
\begin{align*}
    \mathbb{C}^n = \oplus^K_{j=1} N_j
\end{align*}
where $N_j = N_{(A - \lambda_j I)^{d(\lambda_j)}}$. For any $x\in\mathbb{C}^n$, since $AB = BA$, then we have
\begin{align*}
    (A - \lambda_j I)^{d(\lambda_j)} B x = B (A - \lambda_j I)^{d(\lambda_j)} x.
\end{align*}
If $x\in N_j$, then $ (A - \lambda_j I)^{d(\lambda_j)} x = 0$, which implies $Bx\in N_j$. Then, $B$ is a map from $N_j$ into $N_j$. Applying Spectral theorem (\ref{spectral_theorem}) to $B|_{N_j}$, then we obtain a spectral decomposition of each $N_j$, i.e., $N_j$ has a basis consisting of eigenvectors and generalized eigenvectors of $B$. Thus, we obtain a basis of $\mathbb{C}^n$. 
\end{proof}

\begin{remark}
If $A,B$ are both diagonalizable and $AB = BA$, then $A,B$ can be diagonalized at the same time, i.e., there exists nonsingular matrix $S$ such that $S^{-1}AS$ and $S^{-1}BS$ are both diagonal.
\end{remark}

\medskip

\begin{theorem}
Every square matrix $A$ is similar to its transpose $A^T$.
\end{theorem}
\begin{proof}
Since $\dim N_A = \dim N_{A^T}$, then $\dim N_{(A-\lambda I)^m} = \dim N_{(A^T-\lambda I)^m}$. Then, $A$ and $A^T$ have the same Jordan canonical form. Thus, $A$ and $A^T$ are similar.
\end{proof}

\medskip

\begin{theorem}
Let $\lambda, \mu$ be distinct eigenvalues of $A$. Suppose $u$ is an eigenvector with respect to $\lambda$ and $v$ is an eigenvector with respect to $\mu$, i.e., $Au = \lambda u, Av = \mu v$. Then $u^Tv = 0$.
\end{theorem}
\begin{proof}
We have 
\begin{align*}
    v^T A u & = u^T A^T v \\
    \Rightarrow \lambda v^T u & = \mu u^T v.
\end{align*}
Since $\lambda \neq \mu$, we have $u^T v = 0$.
\end{proof}

\medskip

\section{Exercises}
\begin{exercise}
Let $A$ be an invertible $n\times n$ matrix, show that there exists a
polynomial $g$ such that
$$
A^{-1}=g\left(A\right).
$$
\end{exercise}
\begin{proof}
Since $A$ is invertible, then $A$ has no zero engenvalues. Thus, the characteristic polynomial $P(x)$ for $A$ has constant terms, which can be written as $P(x)= x^n + a_{n-1} x^{n-1} + \cdots + a_0$. Also, we know that $P(A)=0$, thus we have
\begin{align*}
    & A^n + a_{n-1} A^{n-1} + \cdots + a_0 = 0 \\
    \Rightarrow & A^{-1} = -\frac{1}{a_0}(A^{n-1}+a_{n-1}A^{n-2}+\cdots+a_1) = g(A)
\end{align*}
Then $A^{-1}=g(A)$, the proof is complete. 
\end{proof}

\medskip

\begin{exercise}
Let
\begin{align*}
  A=\left(
    \begin{array}
    [c]{cc}%
    A_{1} & \\
    & A_{2}%
    \end{array}
    \right).  
\end{align*}
Show that the minimal polynomial $m_{A}$ is the least common multiple of
$m_{A_{1}}$ and $m_{A_{2}}$.
\end{exercise}
\begin{proof}
From the form of $A$, we can know that $\det(\lambda-IA)=\det(\lambda-IA_1)\det(\lambda-IA_1)$. Then, for any polynomial $T(x)$ such that $T(A)=0$, then we have $T(A_1)=0$ and $T(A_2)=0$. And since $m_A$, $m_{A_1}$ and $m_{A_2}$ are minimal polynomials corresponding to $A$, $A_1$ and $A_2$, then we have $T(x)=m_1 m_{A_1}(x)$ and $T(x)=m_2 m_{A_1}(x)$ for some $m_1, m_2$. Also, we have $m_A(x)|T(x)$, then we have $m_{A_1}(x)|m_A(x)$ and $m_{A_1}(x)|m_A(x)$, then $m_A$ is the least common multiple of $m_{A_{1}}$ and $m_{A_{2}}$.
\end{proof}

\medskip

\begin{exercise}
Find the minimal polynomial $m_{A}$ for%
$$
A=\left(
\begin{array}
[c]{ccc}%
1 & -1 & 1\\
0 & 2 & 0\\
0 & 0 & 2
\end{array}
\right)  .
$$
\end{exercise}
\begin{proof}
The characteristic polynomial for $A$ is that $P(\lambda)=(\lambda-1)(\lambda-2)^2$. Then the minimal polynomial is $m_A=(\lambda-1)(\lambda-2)$.
\end{proof}

\medskip

\begin{exercise}
Let $A$ be an $n\times n$ matrix where $n\geq2$ satisfying
$\operatorname*{rank}A=1$.
\begin{enumerate}[label=(\arabic*)]
    \item Show that there exists two column vectors $a,b$ such that $A=ab^{T}$.
    \item Show that the minimal polynomial
    $$m_{A}=\lambda^{2}-\left(  a^{T}b\right)  \lambda.$$
\end{enumerate}
\end{exercise}
\begin{proof}
~\begin{enumerate}[label=(\arabic*)]
    \item Since $\text{rank}A=1$, then the image of $A$ is one-dimensional. Thus, there exist $u,v\in \mathbb{R}^n$ such that $Au=kv$ for a fixed $v$. It also holds for a basis for $\mathbb{R}^n$, then every column of $A$ is a multiple of $v$. Then there exists $(w_1, w_2,\cdots,w_n)\in\mathbb{R}^n$, such that
    \begin{align*}
        A=v(w_1, w_2,\cdots,w_n)
    \end{align*}
    then we denote $v=a$, and $(w_1, w_2, \cdots, w_n) = b^T$, where $a,b\in\mathbb{R}^n$.Then $A=ab^T$.
    \item We have $A^2 = ab^Tab^T = a(b^Ta)b^T = (b^Ta)ab^T = (b^Ta)A$, which implies $q(A)=A^2-(b^Ta)A=0$. This polynomial satisfies that $q(A)=0$, then $m_a|q(\lambda) = \lambda^2-(b^Ta)\lambda$. Also, $m_A$ cannot be $\lambda$ or $\lambda-(b^Ta)$, since this means $A$ is a scalar. Thus, $m_A = \lambda^2-(b^Ta)\lambda$. The proof is complete.
\end{enumerate}
\end{proof}

\medskip

\begin{exercise}
Let $A_{k}$, $1\leq k\leq K$ be $n\times n$ matrices satisfying%
$$
A_{i}A_{j}=A_{j}A_{i}\text{ for any }1\leq i,j\leq k.
$$
Show the existence of a basis of $\mathbb{C}^{n}$ which consists of
eigenvector and generalized eigenvectors of $A_{k}$ for each $1\leq k\leq K$.
\end{exercise}
\begin{proof}
Let $\{\lambda_j\}^J_{j=1}$ be $J$ distinct eigenvalues of $A_1$, and then we have
\begin{align*}
    \mathbb{C}^n = \bigoplus^J_{j=1}N_j
\end{align*}
where $N_j=N_{(A_1-\lambda_j I)^{d_j}}$, $d_j$ is index of $j$th eigenvalue $\lambda_j$. For $\forall x\in\mathbb{C}^n$, since $A_1 A_i=A_i A_1,2\leq i \leq K$, then we have $(A_1-\lambda_j I)^{d_j}A_i=A_i(A_1-\lambda_j I)^{d_j}$. Thus, if $x\in N_k$
\begin{align*}
    (A_1-\lambda_j I)^{d_j}A_ix =A_i(A_1-\lambda_j I)^{d_j}x = 0
\end{align*}
which means $A_i x\in N_j$. Thus, $A_i$ is a mapping from $N_j$ to $N_j$. Now we apply Spectral Theorem to the linear mapping $A_i$ and we know that $N_j$ has a basis consisting of eigenvevtors and generalized eigenvevtors of $A_i$. And it is true for all $A_i, 2\leq i \leq K$. Thus, a basis of $\mathbb{C}^{n}$ consists of
eigenvectors and generalized eigenvectors of $A_{j}$ for each $1\leq j\leq K$. The proof is complete.
\end{proof}

\noindent \textbf{Method II of proof}
\begin{proof}
We will prove it by induction on $\dim V$ and $1\leq k\leq K$. And assume we have pairwise commuting operators $A_1, A_2,\cdots, A_K$ on $V$. 

When $\dim V = 1$, and in this case, all $A_i$ are scalars. Take $B = \{1\}$, and $k$ be arbitrary. Assume the result is true whenever $\dim V < l$, and $A_1, A_2, \cdots, A_k$ are pairwise commuting operators on $V$ if $\dim V < l$. And we want to show that if $\dim V = l$, and $A_1, \cdots, A_{k+1}$ are pairwise commuting operators on $V$, then there exists a basis of generalized eigenvectors. 

For $A_1$, we have 
\begin{align*}
    \mathbb{C}^l = \bigoplus^m_{j=1}N_{\lambda_j}(A_1)
\end{align*}
since $A_1A_i = A_iA_1, 2 \leq i \leq k+1$, we have $A_i:N_{\lambda_j}(A_1)\rightarrow N_{\lambda_j}(A_1), \forall j = 1,\cdots, m$ and $\forall i = 1,2,\cdots, k+1$. 
\begin{enumerate}[label=(\arabic*)]
    \item If $m = 1$, then there exists a basis $B$ of $\mathbb{C}^1$ consisting of generalized eigenvectors for $A_2, \cdots, A_k,A_{k+1}$. Any vectors in $\mathbb{C}^l$ is a generalized eigenvectors for $A_1$ because $\mathbb{C}^l = N_{\lambda_1}(A_1)$, then any vectors in $B$ is a a generalized eigenvectors for $A_1, A_2, \cdots, A_{k+1}$.
    \item If $m > 1$, then $N_{\lambda_j}(A_1)\neq \mathbb{C}^l, \forall j = 1,\cdots, k+1$. On $N_{\lambda_j}(A_1)$, we have $A_1|_{N_{\lambda_j}(A_1)}, A_2|_N_{\lambda_j}(A_1), \cdots, A_{k+1}|_N_{\lambda_j}(A_1)$. By induction, there exists a basis $\beta_j, j=1,2,\cdots, m$ of $N_{\lambda_j}(A_1)$, which are generalized eigenvectors for $A_1|_{N_{\lambda_j}(A_1)}, \cdots, A_{k+1}|_N_{\lambda_j}(A_1)$. Take 
    \begin{align*}
        \beta = \bigcup^m_{j=1}\beta_j
    \end{align*}
    then $\beta$ is a basis for $\mathbb{C}^l$ consisting of generalized eigenvectors for $A_1, A_2, \cdots, A_{k+1}$.
\end{enumerate}
\end{proof}

\medskip

\begin{exercise}
Let $\lambda$ be an eigenvalue of an $n\times n$ matrix $A$. Suppose
that%
\begin{align*}
    \dim N_{1}\left(\lambda\right) & = 2, \dim N_{2}\left(\lambda\right) = 4\\
    \text{ and }\dim N_{3}\left(\lambda\right) & = \dim N_{4}\left(\lambda\right) = 5,
\end{align*}
Find the Jordan blocks of $A$ corresponding to $\lambda$.
\end{exercise}
\begin{proof}
Since $\dim N_{3}\left(\lambda\right) = \dim N_{4}\left(\lambda\right) = 5$, then we can know that the index $d(\lambda) = 3$, then we can know the Jordan blocks of $A$ corresponding to $\lambda$ is 
\begin{align*}
    J = \left(\begin{array}
    [c]{ccccc}%
    \lambda & 1 & 0 & 0 & 0\\
    0 & \lambda & 1 & 0 & 0\\
    0 & 0 & \lambda & 0 & 0\\
    0 & 0 & 0 & \lambda & 1\\
    0 & 0 & 0 & 0 & \lambda
    \end{array}\right).
\end{align*}
We can verify that this is the Jordan blocks we want. We can compute $N_{(J-\lambda I)}$, $N_{(J-\lambda I)^2}$, $N_{(J-\lambda I)^3}$ and $N_{(J-\lambda I)^4}$. We have
\begin{align*}
    J-\lambda I = \left(\begin{array}
    [c]{ccccc}%
    0 & 1 & 0 & 0 & 0\\
    0 & 0 & 1 & 0 & 0\\
    0 & 0 & 0 & 0 & 0\\
    0 & 0 & 0 & 0 & 1\\
    0 & 0 & 0 & 0 & 0
    \end{array}\right)
\end{align*}
and it is obvious that $\dim N_{(J-\lambda I)} = 2$, since there are two $0$ column vectors. Similarly, we have 
\begin{align*}
    (J-\lambda I)^2 = \left(\begin{array}
    [c]{ccccc}%
    0 & 0 & 1 & 0 & 0\\
    0 & 0 & 0 & 0 & 0\\
    0 & 0 & 0 & 0 & 0\\
    0 & 0 & 0 & 0 & 0\\
    0 & 0 & 0 & 0 & 0
    \end{array}\right), 
    (J-\lambda I)^3 = (J-\lambda I)^4 = \left(\begin{array}
    [c]{ccccc}%
    0 & 0 & 0 & 0 & 0\\
    0 & 0 & 0 & 0 & 0\\
    0 & 0 & 0 & 0 & 0\\
    0 & 0 & 0 & 0 & 0\\
    0 & 0 & 0 & 0 & 0
    \end{array}\right)
\end{align*}
and we can know that $\dim N_{(J-\lambda I)^2} = 4$ and $\dim N_{(J-\lambda I)^3} = \dim N_{(J-\lambda I)^3} = 5$. The proof is complete.
\end{proof}

\medskip

\begin{exercise}
Let $A$ be a $5\times5$ rank one matrix, find all possible Jordan canonical forms of $A$. The order of Jordan blocks should be ignored.
\end{exercise}
\begin{proof}
Since $A$ is rank one matrix, then there exists two column vectors $a,b$ such that $A=ab^T$, also we know that the minimal polynomial for $A$ is $m_A(\lambda) = \lambda^2 - \alpha \lambda$. So $A$ has eigenvalue $0$ with multiplicity $4$ and $\alpha$ with multiplicity $1$. There are several possible Jordan forms for eigenvalue $0$, which are
\begin{align*}
    \left(\begin{array}
    [c]{ccccc}%
    0 & 0 & 0 & 0 & 0\\
    0 & 0 & 0 & 0 & 0\\
    0 & 0 & 0 & 0 & 0\\
    0 & 0 & 0 & 0 & 0\\
    0 & 0 & 0 & 0 & \alpha
    \end{array}\right),
    \left(\begin{array}
    [c]{ccccc}%
    0 & 1 & 0 & 0 & 0\\
    0 & 0 & 0 & 0 & 0\\
    0 & 0 & 0 & 0 & 0\\
    0 & 0 & 0 & 0 & 0\\
    0 & 0 & 0 & 0 & \alpha
    \end{array}\right),
    \left(\begin{array}
    [c]{ccccc}%
    0 & 1 & 0 & 0 & 0\\
    0 & 0 & 1 & 0 & 0\\
    0 & 0 & 0 & 0 & 0\\
    0 & 0 & 0 & 0 & 0\\
    0 & 0 & 0 & 0 & \alpha
    \end{array}\right),
    \left(\begin{array}
    [c]{ccccc}%
    0 & 1 & 0 & 0 & 0\\
    0 & 0 & 0 & 0 & 0\\
    0 & 0 & 0 & 1 & 0\\
    0 & 0 & 0 & 0 & 0\\
    0 & 0 & 0 & 0 & \alpha
    \end{array}\right)
\end{align*}
and $$
    \left(\begin{array}
    [c]{ccccc}%
    0 & 1 & 0 & 0 & 0\\
    0 & 0 & 1 & 0 & 0\\
    0 & 0 & 0 & 1 & 0\\
    0 & 0 & 0 & 0 & 0\\
    0 & 0 & 0 & 0 & \alpha
    \end{array}\right).$$
Since the null space of $A-0I$ has dimension $4$ and one of them is generated by eigenvalue $\alpha$. Thus, $\dim N_{A-0I}=3$, which means that there are $3$ blocks corresponding to eigenvalue $0$. Thus, we can know that all possible Jordan canonical forms of $A$ are
\begin{align*}
    \left(\begin{array}
    [c]{ccccc}%
    0 & 1 & 0 & 0 & 0\\
    0 & 0 & 0 & 0 & 0\\
    0 & 0 & 0 & 0 & 0\\
    0 & 0 & 0 & 0 & 0\\
    0 & 0 & 0 & 0 & 0
    \end{array}\right), \text{and}
    \left(\begin{array}
    [c]{ccccc}%
    0 & 1 & 0 & 0 & 0\\
    0 & 0 & 0 & 0 & 0\\
    0 & 0 & 0 & 0 & 0\\
    0 & 0 & 0 & 0 & 0\\
    0 & 0 & 0 & 0 & \alpha
    \end{array}\right).
\end{align*}
\end{proof}

\medskip

\begin{exercise}
Let%
$$
A=\left(
\begin{array}
[c]{ccc}%
1 & 1 & 1\\
0 & 1 & 0\\
0 & 0 & 1
\end{array}
\right)  .
$$
Find its eigenvectors and generalized eigenvectors. Find its Jordan canonical
form $J$ and the corresponding matrix $S$ so that%
$$
A=SJS^{-1}\text{.}%
$$
\end{exercise}
\begin{proof}
Taking $A-\lambda I = 0$, we can have characteristic polynomial $p_A(\lambda) = (1-\lambda)^3$, which gives us eigenvalues $1$. Now we determine the null space of $A-1\dots I$
\begin{align*}
    A-I =\left(\begin{array}
    [c]{ccc}%
    0 & 1 & 1\\
    0 & 0 & 0\\
    0 & 0 & 0
    \end{array}\right).
\end{align*}
So this eigenspace is dimensional-$2$. Hence there are two Jordan blocks corresponding to the eigenvalue 1 in the Jordan form. So we have its Jordan canonical form
\begin{align*}
    J = \left(\begin{array}
    [c]{ccc}%
    1 & 1 & 0\\
    0 & 1 & 0\\
    0 & 0 & 1
    \end{array}\right).
\end{align*}
Then we can know the eigenvectors corresponding to $1$ are 
\begin{align*}
    v_1 = \left(\begin{array}
    [c]{c}%
    1 \\
    0 \\
    0 
    \end{array}\right) 
    \text{and}\,\,
    v_2 = \left(\begin{array}
    [c]{c}%
    1 \\
    1 \\
    -1 
    \end{array}\right).
\end{align*}
Each of these will give Jordan chain and we compute $(A-I)w_1 = v_1$ and $(A-I)w_2 = v_2$. The second equation does not have solution, so we can know that
\begin{align*}
    w_1 = \left(\begin{array}
    [c]{c}%
    1 \\
    1 \\
    0 
    \end{array}\right).
\end{align*}
Then we have the engenvectors and generalized engenvectors, which are 
\begin{align*}
    \left(\begin{array}
    [c]{c}%
    1 \\
    1 \\
    0 
    \end{array}\right),
    \left(\begin{array}
    [c]{c}%
    1 \\
    0 \\
    0 
    \end{array}\right),
    \text{and}\,\,
    \left(\begin{array}
    [c]{c}%
    1 \\
    1 \\
    -1 
    \end{array}\right).
\end{align*}
Thus, we can find $S$, such that $AS=JS$, and we have
\begin{align*}
    S = \left(\begin{array}
    [c]{ccc}%
    1 & 1 & 1\\
    0 & 1 & 1\\
    0 & 0 & -1
    \end{array}\right).
\end{align*}
\end{proof}

\medskip

\begin{exercise}
Let $P$ be the linear space of polynomials with real coefficients
equipped with the scalar product%
$$
\left(  f,g\right)  =\int_{0}^{1}f\left(  x\right)  g\left(  x\right)  dx.
$$
\begin{enumerate}[label=(\arabic*)]
    \item Using Gram-Schmidt process to generate an orthonormal basis of the span of vectors $\left\{  1,x^{2}\right\}$.
    \item Find the projection of polynomial $x$ on the span of vectors $\left\{  1,x^{2}\right\}$.
\end{enumerate}
\end{exercise}
\begin{proof}
~\begin{enumerate}[label=(\arabic*)]
    \item Set $y_1 = 1$ and $y_2 = x^2$, using Gram-Schmidt process, we can have 
    \begin{align*}
        x_1 & = \frac{y_1}{\left \|y_1\right \|} = \frac{1}{\sqrt{\int^1_0 1dx}} = 1 \\
        x_2 & = \frac{y_2-(y_2,x_1)x_1}{\left \|y_2-(y_2,x_1)x_1\right \|} = \frac{x^2 - \frac{1}{3}}{\sqrt{\int^1_0 (x^2 - 1/3)^2}dx} = \frac{3\sqrt{5}x^2 - \sqrt{5}}{2}.
    \end{align*}
    \item Finding the projection of polynomial $x$ on the span of vectors $\left\{  1,x^{2}\right\}$ is equivalent to findind the solution for $a,b$ in the equations
    \begin{align*}
        (1,x-(a+bx^2)) & = 0 \\
        (x^2,x-(a+bx^2)) & = 0 
    \end{align*}
    which gives us $b = \frac{15}{16}, a = \frac{3}{16}$. Thus, the projection is $\left(\frac{3}{16},\frac{15}{16}\right)$.
\end{enumerate}
\end{proof}

\medskip

\begin{exercise}
Find the least squares solution to the over-determined system%
\begin{align*}
3x-y  &  =1,\\
x+y  &  =1,\\
2x+3y  &  =2.
\end{align*}
\end{exercise}
\begin{proof}
Writting these equations into $AX=b$, where $A = \left(\begin{array}
    [c]{cc}%
    3 & -1\\
    1 & 1\\
    2 & 3
    \end{array}\right), X = \left(\begin{array}
    [c]{c}%
    x\\
    y\\
    \end{array}\right)$, and $b = \left(\begin{array}
    [c]{c}%
    1\\
    1\\
    2
    \end{array}\right)$, then the least square least solution can be determined by $z = (A^TA)^{-1}A^T b = \left(\begin{array}
    [c]{c}%
    0.4638\\
    0.3768\\
    \end{array}\right)
    $.
\end{proof}


\medskip

\chapter{Euclidean Structure}
\section{Scalar Product and Distance}
\begin{definition}
An Euclidean structure in a linear space $X$ over $\mathbb{R}$ is furnished by a real-valued function of two vector arguments called a scalar product and denoted by $(x,y)$, which has the following properties:
\begin{enumerate}[label=(\roman*)]
    \item $(x,y)$ is a bilinear function.
    \item Symmetricity: $(x,y) = (y,x)$.
    \item Positivity: $(x,x) > 0$ except for $x = 0$. 
\end{enumerate}
\end{definition}
\begin{remark}
Scalar product is also called inner product or dot product.
\end{remark}

\medskip

\begin{definition}
The Euclidean length (or the norm) of $x$ is defined by 
\begin{align*}
    \|x\| = \sqrt{(x,x)}.
\end{align*}
For any $x,y\in X$, $\|x- y\|$ is called the distance of these two vectors.
\end{definition}

\medskip

\begin{theorem}[Schwarz Inequality]
For any $x,y\in X$, $\left|(x,y)\right| \leq \|x\| \|y\|$.
\end{theorem}
\begin{proof}
Consider, for all $t$, we have
\begin{align*}
    q(t) & = \|x+ty\|^2 \\
    & = (x+ty,x+ty) \\
    & = (x,x) + (x,ty) + (ty,x) + (ty,ty) \\
    & = \|x\|^2 + 2t (x,y) + t^2 \|y\|^2  \geq 0
\end{align*}
Thus, we have 
\begin{align*}
    4\left|(x,y)\right|^2 - & 4 \|x\|^2 \|y\|^2 \leq 0 \\
    \Rightarrow \left|(x,y)\right| & \leq \|x\| \|y\|.
\end{align*}
\end{proof}

\medskip

\begin{definition}
Suppose $x,y\neq 0$, we define the angle $\theta$ between $x$ and $y$ by
\begin{align*}
    \cos \theta = \frac{(x,y)}{\|x\| \|y\|}.
\end{align*}
\end{definition}

\medskip

\begin{corollary}
$$\|x\| = \max_{\|y\| = 1} (x,y).$$
\end{corollary}
\begin{proof}
Let $y = \frac{x}{\|x\|}$, then $\|x\| = \left(x, \frac{x}{\|x\|}\right) \leq \max_{\|y\| = 1} (x,y)$. With Schwarz inequality, we have $\max_{\|y\| = 1} (x,y) \leq \|x\|$. Thus, we have the desired result.
\end{proof}

\medskip

\begin{theorem}[Triangle Inequality]
$$\|x+y\|\leq \|x\| + \|y\|.$$
\end{theorem}
\begin{proof}
$$\|x+y\|^2 = (x+y, x+y) = \|x\|^2 + 2(x,y) + \|y\|^2 \leq (\|x\| + \|y\|)^2.$$
\end{proof}

\medskip

\begin{definition}
Two vectors $x$ and $y$ are called orthogonal (perpendicular) if $(x,y) = 0$, denoted by $x \perp y$.
\end{definition}

\medskip

\begin{theorem}[Pythagorean Theorem]
$\|x-y\|^2 = \|x\|^2 + \|y\|^2$ holds if $x \perp y$.
\end{theorem}

\medskip

\begin{definition}
Let $X$ be a finite-dimensional linear space with an Euclidean structure. A basis $x_1, \cdots, x_n$ is called orthonormal if 
$$(x_i, x_j) = \delta_{ij}, \forall 1\leq i,j\leq n.$$
\end{definition}

\medskip

\begin{theorem}[Gram-Schmidt]
Given a basis $y_1,\cdots,y_n$ of a finite-dimensional linear space $X$, then there is an orthonormal basis $x_1,\cdots,x_n$ such that $x_k$ is a linear combination of $y_1,\cdots,y_n, 1\leq k \leq n$.
\end{theorem}
\begin{proof}
Define
\begin{align*}
    x_1 & = \frac{y_1}{\|y_1\|} \\
    x_2 & = \frac{y_2 - (y_2,x_1)x_1}{\|y_2 - (y_2,x_1)x_1\|} \\
    & \vdots \\
    x_k & = \frac{y_{k+1} - \sum^k_{j=1}(y_{k+1},x_j)x_j}{\|y_{k+1} - \sum^k_{j=1}(y_{k+1},x_j)x_j\|}.
\end{align*}
We claim that $x_1,\cdots,x_n$ form an orthonormal basis of $X$.
\end{proof}

\medskip

Let $x_1,\cdots,x_n$ be an orthonormal basis of $X$ and assume 
\begin{align*}
    x = \sum^n_{j=1} a_j x_j, y = \sum^n_{j=1} b_j y_j,
\end{align*}
then $(x,y) = \sum^n_{j=1} a_jb_j$ and $\|x\|^2 = \sum^n_{j=1} a_j^2$.  The mapping $x\mapsto (x_1,\cdots,x_n)$ carries Euclidean structure of $X$ into $\mathbb{R}^n$, and we could identify $x$ with $\mathbb{R}^n$.

Consider inner product $(x,y)$, for $\forall x\in X$, we fix $y$, then $(x,y)$ is a linear functional on $X$. We can write it as 
\begin{align*}
    y \mapsto l_y \in X'
\end{align*}
then $y$ is in dual space of $X$.

\medskip

\begin{theorem}
Every linear functional $l\in X'$ can be written in the form $l(x) = (x,y)$ for some $y\in X$. The mapping $l \mapsto y$ is an isomorphism of $X$ and $X'$.
\end{theorem}
\begin{proof}
Let $x_1,\cdots,x_n$ be an orthonormal basis of $X'$. Let $y = \sum^n_{j=1}l(x_j)x_j$, then for any $x = \sum^n_{j=1}a_jx_j$, we have
\begin{align*}
    l(x) = \sum^n_{j=1}l(a_j) x_j = \sum^n_{j=1}\sum^n_{i=1} l(x_j) a_i (x_j, x_i) = (x,y).
\end{align*}
\end{proof}

\medskip

\section{Orthogonal Complement and Projection}
\begin{definition}
Let $Y$ be a subspace of $X$. The orthogonal complement of $Y$ is $$Y^\perp = \{x\in X | (x,y) = 0, \forall y \in Y\}.$$
Recall that we defined before $Y^\perp = \{l\in X' | l(y) = 0, \forall y \in Y\}$, these two definitions match if we identify $X'$ with $X$.
\end{definition}


\medskip

\begin{theorem}
For any subspace $Y\subset X$, we have
\begin{align*}
    X = Y \oplus Y^\perp.
\end{align*}
\end{theorem}
\begin{proof}
Let $y_1,\cdots,y_k$ be an orthogonal basis of $Y$. We can expend it to a basis of $X$: $y_1,\cdots,y_k, \widetilde{y_{k+1}}, \cdots, \widetilde{y_{n}}$. With Gram-Schmidt theorem, we obtain an orthonormal basis $y_1,\cdots,y_k,y_{k+1},\cdots,y_n$ of $X$. 

We claim that $Y^\perp = {\rm span}\{y_{k+1},\cdots,y_n\}$. Indeed, ${\rm span}\{y_{k+1},\cdots,y_n\}\subset Y^\perp$ and $\dim Y^\perp = n-k$, and then they are equal.
\end{proof}

\medskip

\begin{definition}
Given a subspace $Y$ of $X$, $X = Y \oplus Y^\perp$, and for any $x\in X$, $x = y + y^\perp$, where $y\in Y, y^\perp$. The component $y$ is called the orthogonal projection of $x$ into $Y$, denoted by $$y = P_Y x.$$ 
\end{definition}

\medskip

\begin{theorem}
$P_Y$ is linear and $P_Y^2 = P_Y$, i.e., $P_Y$ is a projection.
\end{theorem}
\begin{proof}
Let $y_1,\cdots,y_n$ be an orthonormal basis of $X$, $Y = {\rm span}\{y_1,\cdots,y_k\}$ and $Y^\perp = {\rm span}\{y_{k+1},\cdots,y_n\}$. Then for any $x\in X$, $x = \sum^n_{j=1}a_j y_j$, and we have
\begin{align*}
    P_Y(x) = \sum^k_{j=1} a_j y_j.
\end{align*}
Thus, $P_Y^2 = P_Y$ follows naturally.
\end{proof}

\medskip

\begin{theorem}
Let $Y$ be a linear subspace of Euclidean space $X$, and $x\in X$, then 
\begin{align*}
    \|x - P_Y x\| = \min_{z\in Y} \|x - z\|.
\end{align*}
\end{theorem}
\begin{proof}
For any $x\in X$, we can write $x = x_1 + x_2$, where $x_1\in P_Y x, x_2\in Y^\perp$. Then, for any $z\in Y$, we have
\begin{align*}
    \|z - x\| = \|z - x_1\|^2 + \|x_2\| \geq \|x_2\|^2
\end{align*}
and the equation obtain equal sign only when $z = x_1 = P_Y x$.
\end{proof}

\medskip

\section{Adjoint}

Let $X,U$ be two Euclidean spaces and $A:X\to U$ is a linear map, then we can define its transpose $A':U'\to X'$ defined as follows, for any $l\in U'$:
$$(A' l, x) = (l, Ax).$$
We can identify $U'$ with $U$, $X'$ with $X$.

\begin{definition}
The transpose of a map $A$ of Euclidean space $X$ into $U$ is called the adjoint of $A$, denoted by $A^*:U\to X$, which is defined as follows:
$$(A^* y, x) = (y, Ax).$$
\end{definition}

\medskip

\begin{theorem}
~\begin{enumerate}[label=(\roman*)]
    \item If $A,B$ are two linear maps of $X$ into $U$, then $(A+B)^* = A^* + B^*$.
    \item If $A:X\to U, C:U\to V$, then $(CA)^* = A^* C^*$.
    \item If $A$ is a bijection from $X$ onto $U$, then $\left(A^{-1}\right)^* = \left(A^*\right)^{-1}$.
    \item $\left(A^*\right)^* = A$.
\end{enumerate}
\end{theorem}
\begin{proof}
~\begin{enumerate}[label=(\roman*)]
    \item For $\forall x\in X, \forall y\in U$, we have
    \begin{align*}
        \left((A+B)^*y, x\right) & = (y, (A+B)x) \\
        & = (y, Ax + Bx) \\
        & = (y, Ax) + (y, Bx) \\
        & = (A^*y, x) + (B^*y, x) \\
        & = ((A^* + B^*)y, x).
    \end{align*}
    \item We have $CA:X\to V, (CA)^*:V\to X$, and then for $\forall z\in V, \forall x\in X$,
    \begin{align*}
        \left((CA)^*z, x\right) & = (z, (CA)x) \\
        & = (z, C(Ax)) \\
        & = (C^* z, Ax) \\
        & = (A^* C^* z, x).
    \end{align*}
    \item Claim $I^* = I$, and $I = A^{-1}A: X\to X$. Indeed,  for $\forall x_1, x_2\in X$,
    \begin{align*}
        (I^* x_1, x_2) & = (x_1, Ix_2) = (x_1, x_2) = (Ix_1, x_2).
    \end{align*}
    Then we have 
    \begin{align*}
        \left(A^{-1}A\right)^* & = A^* \left(A^{-1}\right)^* = I \\
        \Rightarrow \left(A^*\right)^{-1} & = \left(A^{-1}\right)^*. 
    \end{align*}
    \item For $\forall x\in X$, we have
    \begin{align*}
        \left(A^{**}x, y\right) & = (x, A^*y) \\
        & = (A^*y, x) \\
        & = (y, Ax) \\
        & = (Ax, y).
    \end{align*}
\end{enumerate}
\end{proof}

\begin{remark}
If we choose an orthogonal basis of $X$ and $U$, then $X$ and $U$ can be identified as $\mathbb{R}^m$ and $\mathbb{R}^n$. And $A:X\to U$ can be represented by a matrix $A:\mathbb{R}^m\to \mathbb{R}^n$, also $A^*:U\to A$ is represented by the transpose of $A$, denoted by $A'$.
\end{remark}

\section{Overdetermined Systems}

Consider a matrix equation
\begin{align*}
    Ax = p,
\end{align*}
where $p_1,\cdots.p_m$ are the measured values, and $A$ is an $m\times n$ matrix. We shall
consider the case where the number $m$ of measurements exceeds the number $n$ of quantities. 

\medskip

\begin{theorem}
Let $A$ be $m\times n$ matrix, $m > n$, and suppose that $A$ has only the trivial nullvector $0$. Then the vector $x$ that mninimizes $\|Ax - p\|$ is the unique solution $z$ to the equation
$$A^* A z = A^* p.$$
\end{theorem}
\begin{proof}
The equation $A^* A z = A^* p$ has a unique solution if and only if $A^* A$ is invertible. Indeed, suppose $A^* Ay = 0$ for some $y \in \mathbb{R}^n$, then 
\begin{align*}
    (A^* A y, y) & = 0 \\
    \Rightarrow (Ay, Ay) & = 0 \\
    \Rightarrow Ay & = 0.
\end{align*}
Since $A$ has only trivial nullvector, then $y = 0$, then the solution is unique.

Let $z$ be the minimizer, then we have
\begin{align*}
    Az - p \perp R_A,
\end{align*}
and for $\forall x\in\mathbb{R}^n$, $(Az - p, Ax) = 0$, which implies $(A^*Az - A^*p, x) = 0$. Thus, since it holds for all $x$, then we have $A^*Az = A^*p$.
\end{proof}

\begin{remark}
If $A$ has a nontrivial nullvector, then the minimizer cannot be unique.
\end{remark}

\medskip

\begin{theorem}
For any subspace $Y$ of $X$, $P_Y^* = P_Y$.
\end{theorem}
\begin{proof}
For $\forall x_1, x_2\in X$, we have
\begin{align*}
    (x_1, P_Y x_2) = (P_Y x_1, P_Y x_2) = (P_Y x_1, x_2).
\end{align*}
which implies $P_Y^* = P_Y$.
\end{proof}

\medskip

\section{Isometry and Orthogonal Group}

\begin{definition}
A mapping of a Euclidean space into itself is called an isometry if it preserves the distance of any paired points, i.e., for any $x, y\in X$, then 
\begin{align*}
    \|M(x) - M(y)\| = \|x - y\|.
\end{align*}
\end{definition}

\medskip

\begin{theorem}
Let $M$ be an isometry from $X$ into itself and $M(0) = 0$. Then:
\begin{enumerate}[label=(\roman*)]
    \item $M$ is linear.
    \item $M^* M = I$ and $\det M = \pm 1$.
    \item $M$ is invertible and its inverse is an isometry.
\end{enumerate}
\end{theorem}
\begin{proof}
~\begin{enumerate}[label=(\roman*)]
    \item For all $x\in X$, write $Mx = x'$, and then $\|x\| = \|x'\|$. And for any $x,y\in X$, we have $\|x' - y'\|^2 = \|x - y\|^2$. Then we have
    \begin{align*}
        \|x'\|^2 - 2(x', y') + \|y'\|^2 & = \|x\|^2 - 2(x, y) + \|y\|^2 \\
        \Rightarrow (x', y') & = (x, y).
    \end{align*}
    
    Let $z = x + y$, then we have $\|z'-(x'+y')\| - \|z-(x+y)\| = 0$, which implies $z' = x' + y'$. Also, let $z = cy$, then we have
    \begin{align*}
        \|z' - cy'\|^2 & = (z' - cy', z' - cy') \\
        & = \|z'\|^2 - 2c (z', y') + c^2 \|y'\|^2 \\
        & = \|z\|^2 - 2c (z, y) + c^2 \|y\|^2 \\
        & = \|z - cy\| = 0
    \end{align*}
    then $z' = cy$. Combined, we have $M$ is linear.
    \item For all $x,y\in X$, $(x,y) = (Mx, My) = (M^* M x, y)$, which implies $M^* M = I$.
    \item $M$ is an isometry, implying $M$ is invertible. And since it one-to-one, also it is onto. Then $M^{-1}$ is also isometry. Indeed, 
    \begin{align*}
        \|M^{-1}x - M^{-1}y\| = \|M M^{-1}x - M M^{-1}y\| = \|x - y\|.
    \end{align*}
\end{enumerate}
\end{proof}

\begin{remark}
Conversely, if $M:X\to X$ is a linear map such that $M^* M = I$, then $M$ preserves the distance.
\end{remark}
\begin{proof}
$\|Mx\|^2 = (Mx, Mx) = (M^* M x, x) = (x, x) = \|x\|^2$.
\end{proof}

\medskip

\begin{definition}
A matrix that maps $\mathbb{R}^n$ onto itself isometrically is called orthogonal.
\end{definition}

\medskip

\begin{proposition}[Properties of orthogonal]
A matrix $M$ is orthogonal if and only if its columns are pairwise orthogonal unit vectors.
\end{proposition}
\begin{proof}
~\begin{enumerate}[label=(\arabic*)]
    \item ($\Rightarrow$) If $M$ is orthogonal, then $M$ preserves the product $$(Me_i, Me_j) = (c_i, c_j) = (e_i,e_j) = \delta_{ij}$$ 
    where $M = (c_1,\cdots,c_n)$. Then $(c_i, c_j) = \delta_{ij}$ implies its columns are pairwise orthogonal unit vectors.
    \item ($\Leftarrow$) If $(c_i, c_j) = \delta_{ij}$ for all $1\leq i, j\leq n$, then we have $M^* M = I$ naturally. Thus, $M$ is isometry.
\end{enumerate}
\end{proof}

\medskip

\begin{proposition}
A matrix $M$ is orthogonal if and only if its rows are pairwise orthogonal unit vectors.
\end{proposition}
\begin{proof}
It is obvious since $M^* = M^{-1}$.
\end{proof}

\begin{remark}
If $M$ is orthogonal, then $M$ maps any orthogonal basis into another orthogonal basis. The inverse is also true.
\end{remark}
\begin{proof}
Suppose $u_1,\cdots,u_n$ is orthogonal basis and $v_1,\cdots,v_n$ is also orthogonal basis, and $v_k = Mu_k, 1\leq k \leq n$. Then we have
$V = (v_1,\cdots,v_n) = M (u_1,\cdots,u_n) = MU$. Thus we have $M = VU^{-1}$. Hence, $M$ is orthogonal.
\end{proof}

\medskip

\section{Norm of a Linear Map}

\begin{definition}
The norm of a linear map $A:X\to U$ is defined by
\begin{align*}
    \|A\| = \sup_{\|x\| = 1} \|Ax\| = \sup_{\|x\| \leq 1} \|Ax\|.
\end{align*}
\end{definition}

\medskip

\begin{theorem}
Let $A$ be a linear mapping from the Euclidean space $X$ into the
Euclidean space $U$, where $\|A\|$ is its norm. Then, 
\begin{enumerate}[label=(\roman*)]
    \item For any $z\in X$, $\|Az\|\leq \|A\|\cdot \|z\|$.
    \item $\|A\| = \sup_{\|x\| = \|z\| = 1} (x, Az)$.
\end{enumerate}
\end{theorem}
\begin{proof}
~\begin{enumerate}[label=(\roman*)]
    \item If $z = 0$, then it holds. If $z\neq 0$, then we have 
    \begin{align*}
        \|Az\| = \left\|A \frac{z}{\|z\|}\|z\|\right\| = \|z\|\cdot \left\|A \frac{z}{\|z\|}\right\| \leq \|A\|\cdot \|z\|.
    \end{align*}
    \item For any $x,z\in X$, and $\|x\| = \|z\| = 1$, then 
    \begin{align*}
        (x, Az) \leq \|x\| \cdot \|Az\| \leq \|Az\| \\
        \Rightarrow \sup_{\|x\| = \|z\| = 1} (x, Az) \leq \|Az\|.
    \end{align*}
    
    If $\|A\| = 0$, then $\sup (x, Az) = 0$, which implies $\|A\| \leq \sup (x, Az)$. If $\|A\| \neq 0$, then for all $\varepsilon > 0$, there exists $z \in X$ and $\|z\| = 1$ such that 
    \begin{align*}
        \|Az\| \geq \|A\| - \varepsilon.
    \end{align*}
    Take $x = \frac{Az}{\|Az\|}$, then we have 
    \begin{align*}
        (x, Az) = \frac{\|Az\|^2}{\|Az\|} & = \|Az\| \geq \|A\| - \varepsilon \\
        \Rightarrow \|A\| - \varepsilon & \leq \sup_{\|x\| = \|z\| = 1} (x, Az) \\
        \xrightarrow{\varepsilon \to 0} \|A\| & \leq \sup_{\|x\| = \|z\| = 1} (x, Az).
    \end{align*}
    Thus, we have $\|A\| = \sup_{\|x\| = \|z\| = 1} (x, Az)$.
\end{enumerate}
\end{proof}

\begin{remark}
For any $1\leq i,j\leq n$, $\left|a_{ij}\right| \leq \|A\|$.
\end{remark}
\begin{proof}
With the second argument in the previous theorem, $a_{ij} = (e_i,Ae_j) \leq \|A\|$.
\end{proof}

\medskip

\begin{theorem}
~\begin{enumerate}[label=(\roman*)]
    \item For all $k\in\mathbb{R}$, $\|kA\| = |k|\cdot \|A\|$.
    \item $A,B:X\to U$, then $\|A + B\|\leq \|A\| + \|B\|$.
    \item $A:X\to U, B:U\to V$, then $\|BA\| \leq \|B\|\cdot \|A\|$.
    \item $\left\|A^*\right\| = \|A\|$.
\end{enumerate}
\end{theorem}
\begin{proof}
~\begin{enumerate}[label=(\roman*)]
    \item If $k = 0$, then it holds. If $k\neq 0$, then $\|kAx\| = |k| \cdot \|Ax\|$, which implies
    \begin{align*}
        \|kA\| = \sup_{\|x\|=1} \|kAx\| = \sup_{\|x\|=1} |k| \cdot \|Ax\| = |k|\cdot \|A\|.
    \end{align*}
    \item For any $x\in X$, and $\|x\| = 1$, we have 
    \begin{align*}
        \|A + B\| \leq \sup \|(A+b)x\|  = \sup \|Ax + Bx\| \leq \sup  \|Ax\| + \sup \|Bx\| = \|A\| + \|B\|.
    \end{align*}
    \item For any $x\in X$, and $\|x\| = 1$, we have 
    \begin{align*}
        \|BA\| \leq \sup\|BAx\|\leq \|B\|\cdot \|Ax\| \leq \|B\|\cdot \|A\|.
    \end{align*}
    \item $A^*:U\to X$, for all $x\in X$ and all $u\in U$, we have $(A^* u, x) = (u, Ax) = (Ax, u)$. Thus, 
    \begin{align*}
        \left\|A^*\right\| = \sup_{\|x\| = \|u\| = 1} (A^*u,x) = \sup_{\|x\| = \|u\| = 1} (Ax,u) = \|A\|.
    \end{align*}
\end{enumerate}
\end{proof}

\medskip

\begin{theorem}
Let $X$ be a finite-dimensional Euclidean space and $A:X\to X$ is an invertible linear map. Let $B:X\to X$ be linear map such that 
\begin{align*}
    \|B - A\| \leq \frac{1}{\|A^{-1}\|}
\end{align*}
then $B$ is invertible.
\end{theorem}
\begin{proof}
Let $C = A - B$, then $B = A - C = A\left(I - A^{-1}C\right)$. 

It suffices to show that $I - A^{-1}C$ is invertible. Suppose by contradiction that for some $x\neq 0$, $\left(I - A^{-1}C\right)x = 0$. Then we have
\begin{align*}
    \|x\| & = \|A^{-1}C x\| \\
    & \leq \|A^{-1}C\| \|x\| \\
    & \leq \|A^{-1}\| \|C\| \|x\|.
\end{align*}
Since $\|C\| \leq \frac{1}{\|A^{-1}\|}$, then we have $\|x\| < \|x\|$, which is a contradiction. 

Then, $I - A^{-1}C$ is invertible, so is $B$.
\end{proof}

\begin{remark}
Invertible matrices form an open and dense set.
\end{remark}

\medskip

\section{Completeness and Local Compactness}

\begin{definition}
A sequence of vectors $\{x_n\}$ in Euclidean space $X$ converges to $x\in X$ if $\|x_k - x\| \to 0$ if $k\to \infty$. We write $\lim_{k\to\infty}x_k = x$.
\end{definition}

\medskip

\begin{theorem}
A sequence $\{x_n\}\subset X$ is called a Cauchy sequence if $\forall \varepsilon > 0$, there exists $N > 0$,  such that for all $k, j \geq N$, $\|x_k - x_j\| < \varepsilon$. 
\begin{enumerate}[label=(\roman*)]
    \item Completeness: every Cauchy sequence in a finite-dimensional Euclidean space is convergent.
    \item Locally compactness: any bounded sequence in a finite-dimensional Euclidean space contains a convergent subsequence. 
\end{enumerate}
\end{theorem}
\begin{proof}
~\begin{enumerate}[label=(\roman*)]
    \item Let $x,y$ be two vectors in $X$, and $a_j, b_j$ are their $j$th component respectively, then 
    \begin{align*}
        |a_j - b_j| \leq \|x - y\|.
    \end{align*}
    Denote by $a_{k,j}$ the $j$th component of $x_k$. Since $\{x_k\}$ is Cauchy sequence, then the sequence $\{a_{k,j}\}$ is also a Cauchy sequence. Then $\{a_{k,j}\}$ converges to a real number $a_j$. Denote $x = (a_1, \cdots, a_n)$, then we have 
    \begin{align*}
        \|x_k - x\| = \sum^n_{j=1} |a_{k,j} - a_j|^2
    \end{align*}
    it foolows that $\lim_{k\to\infty} x_k = x$.
    \item Since $|a_{k,j}| \leq \|x_k\|$, then $|a_{k,j}| \leq M \in\mathbb{R}$ for all $k$. Since real numbers are locally compactness, then the theorem follows.
\end{enumerate}

\end{proof}

\medskip

\begin{corollary}
$$\|A\| = \max_{\|x\| = 1} \|Ax\|.$$
\end{corollary}
\begin{proof}
We know that $\|A\| = \sup_{\|x\| = 1} \|Ax\|$, then there exists a sequence $\{x_k\}$ such that $\|x_k\| = 1$ and $\lim_{k\to\infty} \|Ax_k\| = \|A\|$. Also there exists convergent subsequence $\|x_{n_k}\|$ of $\{x_k\}$, and denote by $\lim_{k\to\infty} x_{n_k} = x$. Then, we have $\|x\| = \lim_{k\to\infty} \|x_{n_k}\| = 1$. 

Since $\lim_{k\to\infty} Ax_{n_k} = Ax$, indeed, we have
\begin{align*}
    \|Ax_{n_k} - Ax\| \leq \|A\| \cdot \|x_{n_k} - x\| \to 0,
\end{align*}
then we have 
\begin{align*}
    \|A\| = \lim_{k\to\infty} \|Ax_{n_k}\| = \|Ax\|.
\end{align*}
\end{proof}

\medskip

\begin{theorem}
Let $X$ be an Euclidean space and suppose $X$ is locally compact in the sense that any bounded sequence has a convergent subsequence. Then $X$ is of finite dimension.
\end{theorem}
\begin{proof}
Assume $\dim X = \infty$, then there exists a sequence $\{x_k\}^\infty_{k=1}$ and any finite set of vectors are linearly independent (otherwise $X$ will be finite-dimensional). Then there exists $\{e_k\}^\infty_{k=1}$ such that $(e_i, e_j) = \delta_{ij}$. And for any $i\neq j$, we have $\|e_i - e_j\|^2  = 2$. 

Thus, $\{e_k\}^\infty_{k=1}$ has no convergent subsequence, which is a contradiction.
\end{proof}

\medskip

\begin{definition}
A sequence of linear mappings $\{A_n\}$ from $X$ to $Y$ converges to $A:X\to Y$ if $\lim_{n\to\infty} \|A_n - A\| = 0$.
\end{definition}

\medskip

\begin{theorem}
In finite-dimensional spaces, then $\lim_{n\to\infty}A_n = A$ if and only if, for all $x\in X$, $\lim_{n\to\infty} A_n x = Ax$, which is called weak convergence.
\end{theorem}
\begin{proof}
~\begin{enumerate}[label=(\arabic*)]
    \item ($\Rightarrow$) If $\lim_{n\to\infty}A_n = A$, then 
    \begin{align*}
        \|A_n x - Ax\| \leq \|A_n - A\|\cdot \|x\| \xrightarrow{n \to 0} 0.
    \end{align*}
    \item ($\Leftarrow$) 
    \begin{enumerate}[label=(\alph*)]
        \item First, we show that $\{A_n\}$ is bounded. Let $\{e_i\}$ be an orthogonal basis of $X$. For each $e_i, 1\leq i\leq \dim X = N$, we have
        \begin{align*}
            \lim_{n\to\infty} A_n e_i = A e_i,
        \end{align*}
        then $\|A_n e_i\| \leq a_i$ for all $n$ and some $a_i \geq 0$. For any $x\in X$, $x = \sum^N_{i=1}x_i e_i$, then
        \begin{align*}
            \|A_n x\| & = \left\|\sum^N_{i=1}x_i A_n e_i\right\| \\
            & \leq \sum^N_{i=1}a_i \|x_i\| \\
            & \leq \left(\sum^N_{i=1}a_i^2\right)^{\frac{1}{2}} \cdot \left(\sum^N_{i=1}\|x_i\|^2\right)^{\frac{1}{2}} \\
            &  = \left(\sum^N_{i=1}a_i^2\right)^{\frac{1}{2}} \|x\|.
        \end{align*}
        Thus, $\|A_n\| \leq \left(\sum^N_{i=1}a_i^2\right)^{\frac{1}{2}}$, which implies $\{A_n\}$ is bounded.
        \item Second, we prove the convergence.
        
        Let $A:X\to Y$ be defined as $Ax = \lim_{n\to\infty}A_n x$. We claim $A$ is linear. For any $x\in X, \|x\| = 1$ and $\forall \varepsilon > 0$, there exists a sequence $\{x_k\}^{N_\varepsilon}_{k=1}$ such that $\|x_k\| = 1$ and for $1\leq k \leq N_\varepsilon$, $\|x_k - x\| \leq \varepsilon$. Thus, we have
        \begin{align*}
            \|A_n x_k  - Ax_k\| & \leq \|A_n x - A_n x_k\| + \|A_n x - A x\| + \|Ax_k - Ax\| \\
            & \leq a\|x - x_k\| + \|A_n x - A x\| + \|A\|\cdot \|x_k - x\| \\
            & \leq (a + \|A\|)\|x_k - x\| + \|A_n x - A x\| \xrightarrow{n\to\infty} 0. 
        \end{align*}
        Hence, $\lim_{n\to\infty}A_n = A$.
    \end{enumerate}
\end{enumerate}
\end{proof}

\medskip

\section{Complex Euclidean Structure}

\begin{definition}
A complex Euclidean structure over a linear space $X$ over $\mathbb{C}$ is furnished by a complex valued function, called a scalar product or inner product, denoted by $(x,y)$, such that
\begin{enumerate}[label=(\arabic*)]
    \item $(x,y)$ is linear in $x$ when $y$ is fixed.
    \item Conjuate: $(x,y) = \overline{(y,x)}$ for all $x,y\in X$.
    \item Positivity: $(x,x) > 0$ for all $x\neq 0$.
\end{enumerate}
\end{definition}
\begin{remark}
For $x$ fixed, $(x, y)$ is a skew linear function of $y$, i.e.,
\begin{align*}
    (x, \alpha y_1 + \beta y_2) & = \overline{(\alpha y_1 + \beta y_2,x)} \\
    & = \overline{(\alpha y_1,x) + (\beta y_2,x)} \\
    & = \overline{\alpha} \overline{(y_1,x)} + \overline{\beta} \overline{(y_2,x)} \\
    & = \overline{\alpha}(x, y_1) + \overline{\beta}(x, y_2).
\end{align*}
\end{remark}

\medskip

\begin{definition}
The norm of $x\in X$ is defined as: $\|x\| = \sqrt{(x,x)}$.
\end{definition}

\medskip

\begin{theorem}[Schwarz Inequality]
$$|(x,y)|\leq \|x\|\cdot \|y\|.$$
\end{theorem}
\begin{proof}
We have 
\begin{align*}
    \|x+y\|^2 & = \|x\| + (x,y) + (y,x) + \|y\|^2 \\
    & = \|x\| + 2\Re (x,y) + \|y\|^2.
\end{align*}
For any $\lambda$, $\|\lambda x + y\|^2 = \lambda^2 \|x\|^2 + 2\lambda \Re (x,y) + \|y\|^2 \geq 0$, then we have
\begin{align*}
    4\left|\Re (x,y)\right|^2 & - a\|x\|^2 \|y\|^2 \leq 0 \\
    \Rightarrow \left|\Re (x,y)\right| & \leq \|x\|\cdot \|y\|. 
\end{align*}
Since $\left(e^{i\theta}x,y\right) = e^{i\theta}(x,y)$, then we pick $\theta$ such that $e^{i\theta}(x,y) > 0$. Thus, we have
\begin{align*}
    \left|(x,y)\right| = \left|\Re (e^{i\theta}x,y)\right| \leq \|x\|\cdot \|y\|.
\end{align*}
\end{proof}

\medskip

\begin{theorem}[Triangle Inequality]
$$\|x+y\| \leq \|x\| + \|y\|.$$
\end{theorem}
\begin{proof}
We have
\begin{align*}
    \|x+y\|^2 & = \|x\| + 2 \Re(x,y) + \|y\|^2 \\
    & \leq \|x\| + 2\left|(x,y)\right| + \|y\|^2 \\
    & \leq \|x\| + 2 \|x\| \cdot \|y\| + \|y\|^2 \\
    & = (\|x\| + \|y\|)^2.
\end{align*}
\end{proof}

\medskip

Let $X, Y$ be two complex Euclidean spaces and $AX\to Y$. The adjoint $A^*$ of $A$ is defined as follows 
\begin{align*}
    (x, A^* y) = (Ax, y)
\end{align*}
then $A^* = \overline{A^T}$ as matrix. Fix $y$, $(Ax, y)$ is linear in $X$. We claim that there exists $z\in X$ such that $(x,z) = (Ax, y)$, then $A^* y = z$. Indeed, we define in $\mathbb{C}^n$:
\begin{align*}
    (x,y) = \left((x_1,\cdots,x_n), (y_1,\cdots,y_n)\right) = \sum^n_{i=1}x_i \overline{y_i}.
\end{align*}
Then, for $A = (a_{ij})_{n\times n}:\mathbb{C}^n \to \mathbb{C}^n$, we have
\begin{align*}
    (Ax, y) & = \sum_{1\leq i,j\leq n}a_{ij} x_j \overline{y_j} \\
    & = \sum_{1\leq i,j\leq n}a_{ij} \overline{\overline{a_{ij}} y_i} \\
    & = (x, A^* y).
\end{align*}

\medskip

\begin{definition}
A linear mapping of a complex Euclidean space into itself is called unitary if it is isometry.
\end{definition}

\medskip

\begin{theorem}
$M$ is unitary if and only if $M^*M = 1$. Hence, for a unitary map $M$, $|\det M| = 1$.
\end{theorem}
\begin{proof}
~\begin{enumerate}[label=(\arabic*)]
    \item ($\Rightarrow$) Since $M$ is isometric, then we have $(x,y) = (Mx, My) = (M^* M x, y)$, which implies $M^* M = I$.
    
    Also, $\det M \det M^* = 1$ and $\det M = \det \overline{M}^T = \overline{\dim M}$. Thus, we have $|\det M| = 1$.
    \item ($\Leftarrow$) If $M^*M = 1$, then we have
    \begin{align*}
        (x,y) = (M^* M x, y) = (Mx, My)
    \end{align*}
    which implies $M$ is isometric.
\end{enumerate}
\end{proof}

\medskip

\begin{theorem}
Let $A:\mathbb{R}^n \to \mathbb{R}^n$ or $A:\mathbb{C}^n \to \mathbb{C}^n$, then 
\begin{align*}
    \|A\| = \left(\sum_{1\leq i,j\leq n} \left|a_{ij}\right|^2\right)^{\frac{1}{2}}.
\end{align*}
\end{theorem}
\begin{proof}
For any $x\in \mathbb{C}^n$, we have
\begin{align*}
    \|Ax\|^2 & = \sum^n_{i=1} \left|\sum^n_{j=1}a_{ij} x_j\right|^2 \\
    & \leq \sum^n_{i=1} \left(\sum^n_{j=1}\left|a_{ij}\right|^2 \right) \left(\sum^n_{j=1} x_j^2 \right) \\
    & = \|x\| \left(\sum^n_{i=1}\sum^n_{j=1}\left|a_{ij}\right|^2 \right),
\end{align*}
then we have $\|A\|\leq \left(\sum^n_{i=1}\sum^n_{j=1}\left|a_{ij}\right|^2 \right)$.
\end{proof}

\begin{definition}
The quantity $\left(\sum^n_{i,j=1}\left|a_{ij}\right|^2 \right)^{\frac{1}{2}}$ is called the Hilbert-Schmidt norm of the matrix $A$, denoted by $\|A\|_{HS}$. 
\end{definition}

\begin{remark}
$\sum^n_{i,j=1}\left|a_{ij}\right|^2 = \tr A^*A = \tr AA^*$.
\end{remark}

\medskip

\section{Spectral Radius}

\begin{definition}
The spectral radius $r(A)$ of a linear mapping $A:X\to X$ is defined as
\begin{align*}
    r(A) = \max_{j} \left|\lambda_j \right|
\end{align*}
where $\lambda_j$ are all possible eigenvalues.
\end{definition}

\begin{remark}
$r(A), \|A\|, \|A\|_{HS}$ are measures of the sign of $A$.
\end{remark}

\medskip

\begin{proposition}
$\|A\| \geq r(A)$.
\end{proposition}
\begin{proof}
Let $\lambda$ be an eigenvalue of $A$ such that $r(a) = \left|\lambda\right|$. Assume $x\neq 0$, $Ax = \lambda x$, then we have $\|A\|\cdot \|x\| \geq \|Ax\| = |\lambda|\cdot \|x\|$. Thus, $\|A\| \geq r(A)$.
\end{proof}

\begin{remark}
~\begin{enumerate}[label=(\arabic*)]
    \item Take $A = \begin{pmatrix}
    0 & 1  \\
    0 & 0  
    \end{pmatrix}$, then $r(A) = 0$ and $\|A\| \neq 0$. Then we can have strictly inequality.
    \item If there exists orthogonal basis formed by eigenvectors, then $\|A\| = r(A)$.
    \begin{proof}
    Suppose $\{x_k\}$ is the orthogonal basis and $Ax_k = \lambda_k x_k$. And for any $x\in X$, $x = \sum^n_{k=1}x_k e_k$, then 
    \begin{align*}
        Ax = \sum^n_{k=1} x_k \lambda_k e_k
    \end{align*}
    where $\|x\|^2 = \sum^n_{i=1}|x_k|^2$. Then we have 
    \begin{align*}
        \|Ax\|^2 = \sum^n_{k=1} |x_k \lambda_k|^2 \leq r(A)^2 \|x\|^2. 
    \end{align*} 
    which implies 
    \begin{align*}
        \frac{\|Ax\|}{\|x\|} = \left\|A \frac{x}{\|x\|}\right\| \leq r(A).
    \end{align*}
    Since this is true for any $x$, and $\|A\| = \max_{\|x\|=1}\|Ax\|$, therefore $\|A\|\leq r(A)$. With previous proposition, we have $\|A\| = r(A)$.
    \end{proof}
\end{enumerate}
\end{remark}

\medskip

\begin{theorem}[Gelfand's Formula]
$$r(A) = \lim_{k\to\infty} \left(\left\|A^k\right\|\right)^{1/k}. $$
\end{theorem}
\begin{proof}
First, we need a lemma.
\begin{lemma}
If $r(A) < 1$, then $\lim_{k\to\infty}A^k = 0$.
\end{lemma}
\begin{proof}
Let $J$ be the Jordan canonical form of $A$, then $r(A) = r(J) < 1$. And we have 
\begin{align*}
    J =  \begin{pmatrix}
    J_1 &  &  \\
     & \ddots  &  \\
     &   & J_l
    \end{pmatrix}
\end{align*}
where $$J_s = \left(\begin{smallmatrix}
    \lambda & 1 &  & \\
     & \ddots  & \ddots & \\
     &   & \ddots &  \\
     &   &   & \lambda
\end{smallmatrix}\right)_{n_s\times n_s}, 1\leq s\leq l.$$
Then we have 
\begin{align*}
    J_s^k = \begin{pmatrix}
    \lambda^k & \binom{k}{1}\lambda^{k-1} & \binom{k}{2}\lambda^{k-2} & \cdots & \binom{k}{n_s-1} \lambda^{k - n_s +1} \\
     & \ddots  & \ddots &  & \vdots \\
     &         & \ddots &  \ddots & \vdots \\
     &         &        &  \ddots & \binom{k}{1}\lambda^{k-1} \\
     &         &        &         & \lambda^k
    \end{pmatrix}.
\end{align*}
We claim $\lim_{k\to\infty}\binom{k}{j}\lambda^{k-j} = 0, 0\leq j\leq n_s-1$ for $|\lambda| < 1$. Then we have $\lim_{k\to\infty} J_s^k = 0$, which implies $\lim_{k\to\infty} J^k = 0$. Since $A = MJM^{-1}$, then $\lim_{k\to\infty} A^k = 0$.
\end{proof}

Now we complete the proof of the theorem.
\begin{enumerate}[label=(\arabic*)]
    \item Since $\|A\|\geq r(A)$, then we have $\left\|A^k\right\| \geq \left(r\left(A^k\right) \right)^{\frac{1}{k}} = \left(r(A)^k \right)^{\frac{1}{k}}  = r(A)$. Then,
    \begin{align*}
        \liminf_{k\to\infty} \left(\left\|A^k\right\|\right)^{1/k} \geq r(A).
    \end{align*}
    \item Consider for any $\varepsilon > 0$ and let $A_\varepsilon = \frac{A}{\varepsilon + r(A)}$. Since $r(A_\varepsilon) < 1$, the above lemma implies that $\lim_{k\to\infty}A_\varepsilon^k = 0$ and hence $\lim_{k\to\infty}\left\|A_\varepsilon^k\right\| = 0$. In particular, there exists $K > 0$ such that for all $k \geq K$, $\left\|A_\varepsilon^k\right\| < 1$. Then, 
    \begin{align*}
        \left(\left\|A_\varepsilon^k\right\| \right)^{\frac{1}{k}} & < 1 \\
        \Rightarrow \left\|A_\varepsilon^k\right\|^{\frac{1}{k}} & < \varepsilon + r(A) \\
        \Rightarrow \limsup_{k\to\infty} \left\|A_\varepsilon^k\right\|^{\frac{1}{k}} & \leq \varepsilon + r(A)
    \end{align*}
    Thus, as $\varepsilon \to 0$, we have $\limsup_{k\to\infty} \left\|A_\varepsilon^k\right\|^{\frac{1}{k}} \leq r(A)$.
\end{enumerate}
Combined results above gives us $r(A) = \lim_{k\to\infty} \left(\left\|A^k\right\|\right)^{1/k}$.
\end{proof}

\medskip

\begin{corollary}
Suppose $A_1 A_2 = A_2 A_1$, then $r(A_1 A_2) \leq r(A_1) r(A_2)$.
\end{corollary}
\begin{proof}
We can have $r(A_1 A_2) = \lim_{k\to\infty} \left\|(A_1 A_2)^k\right\|^{1/k}$, and $(A_1 A_2)^k = A_1^k A_2^k$ since they commute. Thus, 
\begin{align*}
    \left\|(A_1 A_2)^k\right\|^{1/k} \leq \left(\left\|A_1^k\right\|^{1/k} \right) \left(\left\|A_2^k\right\|^{1/k} \right)
\end{align*}
which implies $r(A_1 A_2) \leq r(A_1) r(A_2)$.
\end{proof}

\medskip

\section{Exercises}
\begin{exercise}
Suppose $1\leq k\leq n$ and $x_{1},x_{2},\cdots x_{k}$ are $k$ vectors
in $\mathbb{R}^{n}$ satisfying for any $1\leq i,j\leq k$,%
$$
\left(  x_{i},x_{j}\right)  =\delta_{ij}.
$$
For each $1\leq j\leq k$, let $a_{j}$ be the first component of $x_{j}$. Show
that%
$$
\sum_{j=1}^{k}a_{j}^{2}\leq1.
$$
\end{exercise}
\begin{proof}
Since $(x_i,x_j) = \delta_{ij}, 1\leq i,j\leq n$, then we can arrange $x_1, x_2,\cdots, x_n$ into a matrix and denote it by $A = (x_1, x_2,\cdots, x_n)$, then we have $A$ is an orthogonal matrix with determinant $1$. Then $\det A^* = 1$. 

Now we pick a vector $z = (1,0,\cdots,0)^T \in \mathbb{R}^{n}$. Then we have $A^*z = (a_1, a_2, \cdots, a_n)^T$, and therefore the first component of the vector $AA^*z$ is $\sum_{j=1}^{k}a_{j}^{2}$, which means $AA^*z = \left(\sum_{j=1}^{k}a_{j}^{2}, \cdots \right)^T$. Also, we have $\left\|AA^*z \right\| \leq \left\| I z \right\| = 1$. We denote other components of $AA^*z$ as $w_2, w_3, \cdots, w_n$, then we have 
\begin{align*}
    \sum_{j=1}^{k}a_{j}^{2} \leq \left\|AA^*z \right\|^{1/2} & = \sqrt{\sum_{j=1}^{k}a_{j}^{2} + w_2^2 + \cdots + w_n^2} = 1 \\
    \Rightarrow \sum_{j=1}^{k}a_{j}^{2} & \leq 1.
\end{align*}
\end{proof}

\medskip

\begin{exercise}
Let $A$ be an $m\times n$ matrix, $c_{j}$ $1\leq j\leq n$ be column
vectors of $A$ and $r_{i}$, $1\leq i\leq m$ be row vectors of $A$, show that%
$$
\left\Vert A\right\Vert \geq\max_{1\leq j\leq n}\left\Vert c_{j}\right\Vert
\text{ and }\left\Vert A\right\Vert \geq\max_{1\leq i\leq m}\left\Vert
r_{i}\right\Vert .
$$
Here we view $A$ as a linear map from $\mathbb{R}^{n}$ to $\mathbb{R}^{m}$.
\end{exercise}
\begin{proof}
For $j$th column $c_j$ of $A$, we pick a unit vector $e_j = (0,\cdots, 0,1,0,\cdots,0)^T \in \mathbb{R}^n$, where $j$th entry is $1$, others are all zero. Then we can have $Ae_j = c_j$. Thus, we have 
\begin{align*}
    \left\|c_j \right\| \leq \left\|A \right\| \left\|e_j  \right\| = \left\|A \right\|
\end{align*}
since this is true for all $1\leq j \leq n$, then we have $\max_{1\leq j \leq n}\left\|c_j \right\| \leq \left\|A \right\|$. 

For $i$th row $r_i$ of $A$, we can consider $A^* = (r_1, r_2,\cdots, r_m)$. And still, we pick a vector $e_i' = (0,\cdots, 0,1,0,\cdots,0)^T \in \mathbb{R}^m$, where $i$th entry is $1$, others are all zero. And then we take $A^* e_i' = r_i$, which gives us 
\begin{align*}
    \left\|r_i \right\| \leq \left\|A^* \right\| \left\|e_i'  \right\| = \left\|A^* \right\| = \left\|A \right\|
\end{align*}
in the last step we used the fact that $\left\|A^* \right\| = \left\|A \right\|$ . This is true for all $1\leq i \leq m$, then we have $\max_{1\leq i \leq m}\left\|r_i \right\| \leq \left\|A \right\|$. The proof is complete.
\end{proof}

\medskip

\begin{exercise}
Let
$$
A=\left(
\begin{array}
[c]{cc}%
1 & 2\\
0 & 3
\end{array}
\right)  .
$$
Find the spectral radius, operator norm and Hilbert-Schmidt norm of
$A:\mathbb{R}^{n}\rightarrow\mathbb{R}^{n}$.
\end{exercise}
\begin{proof}
The eigenvalues of $A$ are $1$ and $3$, and then we can know that the spectral radius is $r(a) = \max |\lambda| = 3$. The operator norm of $A$ is  the largest eigenvalues of $AA^T$, which is 
$$
AA^T=\left(
\begin{array}
[c]{cc}%
5 & 6\\
6 & 9
\end{array}
\right).
$$
And the charasteristic polynomial is $\lambda^2 - 14\lambda + 9 = 0$, which gives us norm of $A$ is $\max_{j=1,2} \lambda_j = 7+2\sqrt{10}$. The Hilbert-Schmidt norm of $A$ is $\left\| A\right\| = \left(\sum_{i,j}|a_{ij}^2| \right)^{1/2} = \sqrt{14}$.
\end{proof}


\chapter{Spectral Theory of Self-Adjoint Mappings}
\section{Self-Adjoint Mapping}
\begin{definition}
A linear mapping $A$ of a linear or complex Euclidean space into itself is said to be self-adjoint if $A^* = A$. 
\end{definition}
\begin{remark}
If we pick an orthogonal basis of Euclidean space, we can view $A$ as a matrix $A = (a_{ij})_{n\times n}$, then 
\begin{enumerate}[label=(\roman*)]
    \item In real case, $A^* = A \iff A^T = A \iff A$ is real symmetric, i.e., $a_{ij} = a_{ji}$. 
    \item In complex case, $A^* = A \iff \overline{A^T} = A$, i.e., $a_{ij} = \overline{a_{ji}}$. We say $A$ is Hermitian matrix.
\end{enumerate}
\end{remark}

\medskip

\begin{theorem}\label{theorem_hermitian}
A self-adjoint map $H$ of complex Euclidean space $X$ into itself has real eigenvalues and the set of eigenvectors which is formed by orthogonal basis of $X$.
\end{theorem}
\begin{proof}
We can view $H$ as a matrix. Then,
\begin{enumerate}[label=(\roman*)]
    \item All eigenvalues are real. Assume $\lambda$ is eigenvalue, such that $Hx = \lambda x$. Then we have
    \begin{align*}
        (Hx, x) & = (x, Hx) \\
        \Rightarrow (\lambda x, x) & = (x, \lambda x) \\
        \Rightarrow \lambda (x, x) & = \overline{\lambda} (x, x)
    \end{align*}
    and since $\|x\|\neq 0$, we have $\lambda = \overline{\lambda}$. Thus, $\Im \lambda = 0$, which implies $\lambda$ is real.
    \item There are no generalized eigenvectors. Suppose $x\in N_{(N - \lambda_i I)^2}$ and $\lambda = 0$ for simplicity, then $H^2 x = 0$, which implies
    \begin{align*}
        (H^2 x, x) & = 0 \\
        \Rightarrow (Hx, Hx) & = 0 \\
        \Rightarrow \|Hx\|^2 & = 0 \\
        \Rightarrow Hx & = 0.
    \end{align*}
    Hence, $x\in N_{(N - \lambda_i I)}$, and the index of $\lambda_i$ is $1$. Thus, there are no generalized eigenvectors.
    \item Eigenvectors of $H$ are orthogonal. Suppose $\lambda \neq \mu$ are two eigenvalues such that $Hx = \lambda x, Hy = \mu y$. Then we have
    \begin{align*}
        (Hx, y) & = (x, Hy) \\
        \Rightarrow (\lambda x, y) & = (x, \mu y) \\
        \Rightarrow \lambda(x, y) & = \overline{\mu} (x, y) = \mu (x, y),
    \end{align*}
    which implies $(x,y) = 0$. 
\end{enumerate}
\end{proof}

\begin{remark}
With the theorem abovem, $X = \oplus_{\lambda_i}N_{(H-\lambda_i I)}$. We can pick orthogonal basis for $N_{(H-\lambda_i I)}$ for each $\lambda_i$, then we get orthogonal basis.
\end{remark}


\medskip

\begin{corollary}
Any Hermitian matrix can be diagonalized by a unitary matrix.
\end{corollary}
\begin{proof}
Let $\{x_k\}^n_{k=1}$ be the orthogonal basis consisting of eigenvectors of $A$, such that $Ax_k = \lambda_k x_k$, and $\lambda_k$ is real. Let $U = (x_1, x_2,\cdots, x_n)$, which is unitary, then we have 
\begin{align*}
    AU = U \begin{pmatrix}
    \lambda_1 &  &  \\
     & \ddots  &  \\
     &   & \lambda_n
    \end{pmatrix},
\end{align*}
which implies $A = U\Lambda U^{-1}$.
\end{proof}

\medskip

\begin{theorem}
A self-adjoint map $H$ of real Euclidean space $X$ into itself has real eigenvalues and a set of eigenvectors which is formed by orthogonal basis of $X$.
\end{theorem}
\begin{proof}
We pick an orthogonal basis of $X$ and $H$ can be represented by a matrix $A$ if we identify $X$ as $\mathbb{R}^n$, then we have $A^T = A$. We can extend $A$ to a map from $\mathbb{C}^n$ to $\mathbb{C}^n$, denoted by $\widetilde{A}$, then $\widetilde{A}^* = \widetilde{A}$. Then $\widetilde{A}$ is Hermitian matrix and we can apply the theorem (\ref{theorem_hermitian}). 

We claim $\sigma(A) = \sigma \left(\widetilde{A}\right)$ and $$N_{(A - \lambda I)} = \Re N_{\left(\widetilde{A} - \lambda I\right)}.$$
Indeed, $\widetilde{A}$ can be diagonalized by a unitary matrix $U$ such that $A = U\Lambda U^{-1}$, where $\Lambda$ is a real diagonal matrix. Then, $A$ can be diagonalized by a real matrix $M$ such that $A = M\Lambda M^{-1}$. Thus, the eigenvalues of $A$ are all real. Ans since
\begin{align*}
    \dim_{\mathbb{R}} N_{(A - \lambda I)} = \dim_{\mathbb{C}} N_{\left(\widetilde{A} - \lambda I\right)},
\end{align*}
we have 
\begin{align*}
    \mathbb{R}^n = \oplus_{\lambda \in \sigma(A)} N_{(A - \lambda I)}.
\end{align*}
Thus, we can find a set of eigenvectors that form an orthogonal basis of $\mathbb{R}^n$.
\end{proof}

\medskip

\begin{corollary}
Any symmetric matrix can be diagonalized by an orthogonal matrix.
\end{corollary}

\medskip

\section{Quadratic Forms}
Consider a quadratic form
\begin{align*}
    q(y) = \sum^n_{i,j=1} h_{ij} y_i y_j = (y, Hy),
\end{align*}
where $y = (y_1, \cdots, y_n)^T \in \mathbb{R}^n$, and $H = (h_{ij})_{n\times n}$ is symmetric. Let $x = Ly$, then $y = L^{-1}x$ and
\begin{align*}
    q(y) & = \left(L^{-1}x, H L^{-1}x\right) \\
    & = \left(x, \left(L^{-1}\right)^T H L^{-1}x\right) \\
    & = (x, Mx).
\end{align*}

\medskip

\begin{definition}
Two symmetric matrices $A$ and $B$ are called congruent if there exists an invertible matrix $S$ such that $A = S^T B S$.
\end{definition}

\medskip

\begin{theorem}
Given $q(x) = (x, Ax)$, there exists an invertible matrix $L$, such that
\begin{align*}
    q\left(L^{-1}x\right) = \sum^n_{i=1} d_i x_i^2
\end{align*}
for some constants $d_i$, and we can make $d_i = 0$ or $d_i = \pm 1$.
\end{theorem}
\begin{proof}
Since $A$ is self-adjoint, then there exists a unitary matrix $Q$ such that $A = Q\Lambda Q^*$, where 
\begin{align*}
    \Lambda = \begin{pmatrix}
    \lambda_1 &  &  \\
     & \ddots  &  \\
     &   & \lambda_n
    \end{pmatrix},
\end{align*}
where $\lambda_k, 1\leq k \leq n$ are real eigenvalues of $A$. Then, we have
\begin{align*}
    q(x) & = (x, Ax) \\
    & = (x, Q\Lambda Q^* x) \\
    & = (Q^* x, \Lambda Q^* x) \\
    \Rightarrow q(Qx) & = (Q^* Q x, \Lambda Q^* Q x) \\
    & = (Ix, \Lambda I x) \\
    & = (x, \Lambda x) = \sum^n_{i=1} \lambda_i x_i^2
\end{align*}
and we can pick $L = Q^{-1} = Q^*$. Thus the proof is complete.
\end{proof}

\medskip

\section{Law of Inertia}
\begin{theorem}[Sylvester’s Law of Inertia]
Let $H$ be symmetric, $q(x) = (x, Hx)$ and $L$ be an invertible matrix such that
\begin{align*}
    q\left(L^{-1}x\right) = \sum^n_{i=1} d_i x_i^2
\end{align*}
for some constants $d_i$. Then the numbers of positive, negative and zero terms of $d_i$ equal to the numbers of positive, negative and zero eigenvalues of $H$.
\end{theorem}
\begin{proof}
Let $p_+, p_-$ and $p_0$ be the numbers of positive, negative and zero terms of $d_i$. Let $S$ be a subspace of $\mathbb{R}^n$, we say $q > 0$ on $S$ if for all $u\in S, u\neq 0$, $q(u) > 0$.

We claim that $$p_+ = \max_{q > 0\,\, {\rm on}\,\, S} \dim S. $$
Let $S = L^{-1}\left({\rm span}\{e_i, d_i > 0\} \right)$, then $
\dim S = p_+, q > 0\,\, {\rm on}\,\, S$. 
\begin{enumerate}[label=(\arabic*)]
    \item For $y\in S$, $y = L^{-1}x$, where $x = \sum_{d_i > 0} d_i c_i^2$, then 
    \begin{align*}
        q(y) = q\left(L^{-1}x\right) = \sum^n_{i=1} d_i x_i^2 = \sum_{d_i > 0} d_i c_i^2 \geq 0.
    \end{align*}
    Thus, we have $p_+ \leq \max \dim S$.
    \item Let $S$ be any subspace with $\dim S > p_+$, we claim that $q$ cannot be positive on $S$. For all $u\in S$, we have $Lu = \sum^n_{i=1}u_i e_i$. 
    
    Define $pu = L^{-1}\left(\sum_{d_i>0}u_i e_i \right)$, which is a projection. Then we have
    \begin{align*}
        \dim ps \leq p_+ < \dim S.
    \end{align*}
    Then there exists $y\in S, y\neq 0$, such that $Lpy = 0$. Assume $L = I$, then $py = 0$, which implies $q(y) = 0$. 
\end{enumerate}
\end{proof}

\begin{remark}
Two symmetric square matrices of the same size have the same
numbers of positive, negative and zero eigenvalues if and only if they are congruent.
\end{remark}

\medskip

\section{Spectral Resolution}

\begin{definition}
The set of eigenvalues of $H$ is called the spectral of $H$.
\end{definition}

\medskip

Let $H$ be a self-adjoint map from $X$ into $X$, then we have 
\begin{align*}
    X = \oplus_{\lambda_j \in \sigma(H)}N_{H - \lambda_j I}.
\end{align*}
Hence, $N(\lambda_j) = N_{H - \lambda_j I}$ are orthogonal subspaces. Let $\lambda_j, 1\leq j\leq k$ be distinct eigenvalues. For any $x\in X$, then $x$ can be represented as $x = \sum^k_{j=1}x_j$, where $x_j\in N(\lambda_j)$. We say $x_j$ is orthogonal projection of $X$ to $N(\lambda_j)$, i.e., 
\begin{align*}
    x_j = P_{N(\lambda_j)}(x).
\end{align*}
Applying $H$, we have
\begin{align*}
    Hx = \sum^k_{j=1} \lambda_j x_j.
\end{align*}
Let $P_j$ be the orthogonal projection of $X$ onto $N(\lambda_j)$, we have
\begin{align*}
    I = \sum^k_{j=1} P_j,
\end{align*}
which follows $x = \sum^k_{j=1}x_j$ and hence 
\begin{align*}
    H = \sum^k_{j=1} \lambda_j P_j.
\end{align*}

\medskip

\begin{theorem}
The operators $P_j$ have the following properties:
\begin{enumerate}[label=(\roman*)]
    \item $P_j P_k = 0$ for $j\neq k, P_j^2 = P_j$.
    \item Each $P_j$ is self-adjoint, i.e., $P_j^* = P_j$.
\end{enumerate}
\end{theorem}
\begin{proof}
~\begin{enumerate}[label=(\roman*)]
    \item For any $x\in X$, $x = \sum^k_{j=1}x_j$, where $x_j\in N(\lambda_j)$, we have $P_j P_k x = P_j x_k = 0$. Thus, $P_j P_k = 0$ for $j\neq k$.
    \item For any $x = \sum^k_{j=1}x_j \in X$ and $y = \sum^k_{j=1}y_j \in X$, where $x_j, y_j \in N(\lambda_j)$, we have
    \begin{align*}
        \left(P_j x, y\right) = \left(x_j, y\right) = \left(x_j, \sum^k_{j=1}y_j\right) = \left(x_j, y_j\right)
    \end{align*}
    where in the last step we used the fact that $N_i$ is orthogonal to $x_j$ for $i\neq j$. Similarly, we have
    \begin{align*}
        \left(x, P_j y\right) = \left(x_j, y_j\right).
    \end{align*}
    Thus, we have $P_j^* = P_j$.
\end{enumerate}
\end{proof}

\medskip

\begin{definition}
A decomposition of the form:
\begin{align*}
    I = \sum^k_{j=1} P_j
\end{align*}
where $P_j$ is a projection, i.e., satisfies $P_j P_k = 0$ for $j\neq k, P_j^2 = P_j$ and $P_j^* = P_j$, is called a resolution of the identity. Any self-adjoint map $H$ defineds a resolution of identity.
\end{definition}

\medskip

\begin{definition}
Let $H$ be a self-adjoint map, then 
\begin{align*}
    H = \sum^k_{j=1} \lambda_j P_j
\end{align*}
is called the spectral resolution of $H$.
\end{definition}

\medskip

Given any polynomial $p$, we have
\begin{align*}
    p(H) = \sum^k_{j=1} p\left(\lambda_j\right) P_j,
\end{align*}
since $H^m = \left(\sum^k_{j=1} \lambda_j P_j\right)^m = \sum^k_{j=1} \lambda_j^m P_j^m = \sum^k_{j=1} \lambda_j^m P_j$. Move generally, given a convergent series $f(t) = \sum^\infty_{k=0} a_k t^k$, we have 
\begin{align*}
    f(H) = \sum^k_{j=1} f\left(\lambda_j\right) P_j.
\end{align*}
In particular, 
$$e^H = \sum^\infty_{m=0}\frac{H^m}{m!} = \sum^\infty_{j=1} e^{\lambda_j} P_j.$$

\medskip

\begin{theorem}
Let $H, K$ be two self-adjoint matrices that commute. Then they have a common resolution of identity $I = \sum^k_{j=1} P_j$, such that
\begin{align*}
    H = \sum^k_{j=1} \lambda_j P_j, K = \sum^k_{j=1} \mu_j P_j
\end{align*}
where $\lambda_j\in\sigma(H), \mu_j\in\sigma(K)$.
\end{theorem}
\begin{proof}
Since $X = \oplus^l_{j=1} N\left(\lambda_j\right)$, where $\lambda_j, 1\leq j\leq l$ are distinct eigenvalue of $H$. We claim $N\left(\lambda_j\right)$ is invariant under $K$, i.e., $K: N\left(\lambda_j\right) \to N\left(\lambda_j\right)$. 

Indeed, for any $x\in N\left(\lambda_j\right)$, $HKx = KH x = \lambda_j Kx$, which implies $Kx$ is also an eigenvector of $H$. So $k$ maps $N\left(\lambda_j\right)$ into $N\left(\lambda_j\right)$, we now apply spectral resolution of $K$ over $N\left(\lambda_j\right)$, which gives us the theorem.
\end{proof}

\medskip

\section{Anti-Self Adjoint Mappings}
\begin{definition}
A linear mapping $A$ of Euclidean space into itself is called anti-self-adjoint if $A^* = -A$.
\end{definition}
\begin{remark}
If $A$ is anti-self adjoint, then $(iA)^* = iA$. Thus $A$ can be unitary diagonalized. Also, if $A$ is anti-self adjoint, then $AA^* = A^* A$.
\end{remark}

\medskip

\begin{definition}
A mapping $A$ from a compact Euclidean space into itself is normal if it commutes with its adjoint operator, i.e., $AA^* = A^* A$.
\end{definition}

\medskip

\begin{theorem}
A normal map $N$ has an orthonormal basis consists of eigenvectors.
\end{theorem}
\begin{proof}
Define $H = \frac{N+N^*}{2}$ and $A = \frac{N-N^*}{2}$, then both $H$ and $A$ are self-adjoint. Then we have
\begin{align*}
    HA & = \frac{1}{4}\left(N^2 + N^*N - NN^* - \left(N^*\right)^2 \right) \\
    & = \frac{1}{4}\left(N^2 - \left(N^*\right)^2 \right)\\
    & = AH.
\end{align*}
Then there exists a basis consisting of eigenvectors for both $A$ and $H$. Then we claim that the eigenvectors of $A+H$ are equal to that of $N$, since if we have $Av = \lambda v$ and $Hv = \mu v$, then $(A+H)v = N v = (\lambda + \mu)v$.
\end{proof}

\medskip

\begin{corollary}
If $N$ is a normal matrix, then $N$ can be unitary diagonalized.
\end{corollary}

\medskip

\begin{theorem}
Let $U$ be a unitary map of a complex Euclidean space into itself, that is an isometry linear map. Then,
\begin{enumerate}[label=(\roman*)]
    \item There is an orthogonal basis consisting of genuine eigenvectors of $U$.
    \item The eigenvalues of $U$ are complex numbers with absolute value $1$.
\end{enumerate}
\end{theorem}
\begin{proof}
~\begin{enumerate}[label=(\roman*)]
    \item Since $U$ is unitray, i.e., $UU^* = U^*U = I$, then $U$ is self-adjoint, which implies $U$ is normal. Thus,  the previous theorem indicates the argument.
    \item Isometry preserves the distance, then 
    \begin{align*}
        \|Ux\| = |\lambda| \|x\| = \|x\|.
    \end{align*}
    Thus, $|\lambda| = 1$.
\end{enumerate}
\end{proof}

\medskip

\section{Rayleigh Quotient}

\begin{definition}
Let $H$ be self-adjoint, then the quotient
\begin{align*}
    R_H(x) = \frac{(x,Hx)}{(x,x)}
\end{align*}
is called the Rayleigh quotient of $H$.
\end{definition}

\begin{remark}
~\begin{enumerate}[label=(\arabic*)]
    \item $R_H(kx) = R_H(x)$.
    \item $R_H(x)$ is continuous and real valued, thus it has maximum and minimum.
\end{enumerate}
\end{remark}

\medskip

\begin{theorem}
Maximum and minimum of $R_H(x)$ are eigenvalues of $H$.
\end{theorem}
\begin{proof}
We can view $R_H$ as a real continuous map on the unit sphere. Hence, $R_H$ has a maximum and a minimum. Let $\|x\|=1$, then for any $y$, we have 
\begin{align*}
    R_H(x) = \max_{\|y\|=1} \frac{(y,Hy)}{(y,y)}.
\end{align*}
then we have $\dv{R_H(x+ty)}{t}\bigg|_{t = 0} = 0$. Also, we have
\begin{align*}
    \dv{R_H(x+ty)}{t}\bigg|_{t = 0} & = \dv{}{t}\bigg|_{t = 0} \frac{(x,Hx) + 2t\Re (x,Hy) + t^2 (y,Hy)}{\|x\|^2 + 2t \Re (x,y) + t^2 \|y\|^2} \\
    & = \frac{2\Re (x,Hy)}{\|x\|^2} - \frac{(x,Hx) 2 \Re (x,y)}{\|x\|^4} \\
    & = 2\Re (x,Hy) - (x,Hx) 2 \Re (x,y) \\
    \Rightarrow \Re (Hx,y) & =  \Re ((x,Hx)x,y)
\end{align*}
which implies $Hx = (x,Hx)x$. Thus, $x$ is an eigenvector, and $(x,Hx) = R_H(x), \|x\|=1$ is an eigenvalue.
\end{proof}

\medskip

\begin{corollary}
\begin{align*}
    \max_{x\neq 0}R_H(x) & = \max_{\lambda\in \sigma(H)}\lambda \\
    \min_{x\neq 0}R_H(x) & = \min_{\lambda\in \sigma(H)}\lambda.
\end{align*}
\end{corollary}

\begin{remark}
Every eigenvector $x$ of $H$ is a critical point of $R_H$, that is, the first derivative of $R_H$ are zero when $x$ is an eigenvector of $H$. Conversely, the eigenvectors are the only critical points of $R_H(x)$. The value of the Rayleigh quotient at an eigenvector is the corresponding eigenvalue of $H$.
\end{remark}

\medskip

\section{Minimax Principle}

\begin{theorem}
Let $H$ be a self-adjoint map of a Euclidean space $X$ of finite dimension. Denote the eigenvalues of $H$, arranged in increasing order by $\lambda_1 \leq \lambda_2 \leq \cdots \leq \lambda_n$. Then,
\begin{align*}
    \lambda_j = \min_{\dim S=j} \max_{x\in S, x\neq 0} R_H(x),
\end{align*}
where $S$ is a linear subspace of $X$.
\end{theorem}
\begin{proof}
Let $\{x_j\}^n_{j=1}$ be the orthogonal basis of $X$ such that $Hx_j = \lambda_j x_j$. For any $x\in X$ and $x = \sum^n_{j=1}c_j x_j$, then 
\begin{align*}
    R_H(x) = \frac{\sum^n_{j=1}\lambda_j |c_j|^2}{\sum^n_{j=1} |c_j|^2}.
\end{align*}
Let $S_j = {\rm span} \{x_1,\cdots,x_j\}$, then 
\begin{align*}
    \max_{x\in S_j, x\neq 0} R_H(x) = \max \frac{\sum^j_{k=1}\lambda_k |c_k|^2}{\sum^j_{k=1} |c_k|^2} = \lambda_j,
\end{align*}
which implies 
\begin{align*}
    \min_{\dim S=j} \max_{x\in S, x\neq 0} R_H(x) \leq \lambda_j.
\end{align*}

Next, given any $S$ with $\dim S = j$, we need to show $\max_{x\in S, x\neq 0} R_H(x) \geq \lambda_j$. It suffices to show that there exists $x\in S$, such that the projection of $x$ on $S_{j-1}$ is zero. Denote the projection by $P:S_j\to S_{j-1}$, with Rank-Nullity theorem, there exists $x^* = \sum_{k\geq j}c_k x_k \in S, x^*\neq 0$ such that $Px^* = 0$. Hence, we have
\begin{align*}
    \max_{x\in S, x\neq 0} \frac{(x,Hx)}{(x,x)} \geq \frac{(x^*, Hx^*)}{(x^*,x^*)} = \frac{\sum_{k\geq j}\lambda_k |c_k|^2}{\sum_{k\geq j} |c_k|^2} \geq \lambda_j.
\end{align*}
since $S$ is arbitrary subspace of dimension $j$, then we have
\begin{align*}
    \min_{\dim S=j} \max_{x\in S, x\neq 0} R_H(x) \geq \lambda_j.
\end{align*}
Thus we complete the theorem.
\end{proof}

\medskip

\section{Generalized Rayleigh Quotient}

\begin{definition}
A self-adjoint map $M$ is called positive if for all nonzero $x\in X$, $(x, Mx) > 0$.
\end{definition}

\begin{remark}
With Minimax principle, we can know that $M$ being positive is equivalent to that all eigenvalues of $M$ are positive.
\end{remark}

Now we consider a generalization of the Rayleigh quotient: for $H^* = H, M^* = M$ and $M > 0$, 
\begin{align*}
    R_{H,M}(x) = \frac{(x,Hx)}{(x,Mx)}.
\end{align*}
Let $M = \left(\sqrt{M}\right)^2$, $\sqrt{M} > 0$ and $y = \sqrt{M} x$, then we have
\begin{align*}
    R_{H,M}(x) = \frac{(x,Hx)}{(x,Mx)} & = \frac{\left(\sqrt{M}^{-1}y, H\sqrt{M}^{-1}y\right)}{\left(\sqrt{M}^{-1}y, M\sqrt{M}^{-1}y\right)} \\
    & = \frac{\left(y,\left(\sqrt{M}^*\right)^{-1}H \sqrt{M}^{-1} y \right)}{(y,y)} = R_{\widetilde{H}}(y).
\end{align*}
Now we consider
\begin{align*}
    \dv{}{t}\bigg|_{t = 0} \frac{\left(x+ty,H(x+ty)\right)}{\left(x+ty,M(x+ty)\right)} = \frac{2\Re(x,Hy)}{(x,Mx)} - \frac{2(x,Hx)\Re(x,My)}{(x,Mx)^2} = 0,
\end{align*}
which implies 
\begin{align*}
    \Re(x,Mx)(x,Hy) & = \Re(x,Hx)(x,My) \\
    \Rightarrow \Re(x,Hy) & = \Re R_{H,M}(x) (Mx,y).
\end{align*}
If $x$ is a critical point of $R_{H,M}(x)$, then $\Re(x,Hy)  = \Re R_{H,M}(x) (Mx,y)$ for all $y\in X$. Thus, we have
\begin{align*}
    (Hx, y) = R_{H,M}(x) (Mx,y) \iff Hx = R_{H,M}(x) Mx.
\end{align*}

\medskip

\begin{theorem}
There exists a basis $\{x_1,\cdots,x_n\}$ of $X$ such that 
\begin{align*}
    Hx_j = \lambda_j M x_j, 
\end{align*}
where $\lambda_j$ is real and $\lambda_j = R_{H,M}(x_j)$. Moreover, $\left(x_i, Mx_j\right) = 0$ for $i\neq j$.
\end{theorem}

\medskip

\begin{corollary}
If $M,H$ are self-adjoint, all the eigenvalues of $M^{-1}H$ are real and $M^{-1}H$ is diagonalizable. If $H > 0$, then all the eigenvalues of $M^{-1}H$ are positive.
\end{corollary}

\medskip

\section{Norm and eigenvalues}

We recall that $\|A\| = \max \|Ax\|, \|x\|=1$ or $\|A\| = \sup\frac{\|Ax\|}{\|x\|}$. Then we have the following theorem.

\begin{theorem}
Suppose $N$ is a normal mapping of an Euclidean space $X$ into itself, then $\|N\| \leq r(A)$.
\end{theorem}
\begin{proof}
~\begin{enumerate}[label=(\arabic*)]
    \item $r(N)\leq \|N\|$. Indeed, for any eigenvalue $\lambda_j$ and its corresponding eigenvector $x_j$, we have \begin{align*}
        \|Nx_j\| = |\lambda_j|\cdot \|x_j\|\leq \|N\|\cdot \|x_j\|
    \end{align*}
    which implies $\max \lambda_j = r(N) \leq \|N\|$.
    \item Now we prove the other direction. If $N$ is normal, then there exists orthogonal basis $\{x_1,\cdots, x_n\}$ of $X$ consisting of eigenvectors of $N$, such that $Nx_j = \lambda_j x_j, 1\leq j\leq n$. For any $x\in X$, $x = \sum^n_{j=1}c_j x_j$, then we have
    \begin{align*}
        Nx = \sum^n_{j=1}c_j\lambda_j x_j.
    \end{align*}
    Thus, we have
    \begin{align*}
        \frac{\|Nx\|}{\|x\|} = \left(\frac{\sum^n_{j=1} |c_j|^2 |\lambda_j|^2}{\sum^n_{j=1} |c_j|^2}\right)^{\frac{1}{2}} \leq \max \lambda_j = r(N).
    \end{align*}
\end{enumerate}
\end{proof}

\medskip

\begin{theorem}
Let $A:X\to Y$, then $\|A\| = \sqrt{r(A^*A)}$.
\end{theorem}
\begin{proof}
~\begin{enumerate}[label=(\arabic*)]
    \item $A^*A$ is normal and we have
    \begin{align*}
        \|Ax\|^2 & = (Ax, Ax) \\
        & = (x, A^*Ax) \\
        & \leq \|A^*A\|\cdot \|x\|^2 = r(A^*A) \|x\|^2
    \end{align*}
    which implies $\|A\| \leq \sqrt{r(A^*A)}$.
    \item Let $\lambda$ be an eigenvalue of $A^*A$ such that $\lambda = r(A^*A) \geq 0$ and $A^*Ax = \lambda x, x\neq 0$. Then, for eigenvector $x$, we have
    \begin{align*}
        \|Ax\|^2 & = (Ax,Ax) \\
        & = (x,A^*Ax) \\
        & = \lambda \|x\|^2 \\
        & = r(A^*A) \|x\|^2 
    \end{align*}
    which implies $\|A\| \geq r(A^*A)$. Indeed, 
    \begin{align*}
        \sqrt{r(A^*A)} = \frac{\|Ax\|}{\|x\|} \leq \|A\| = \max \frac{\|Ay\|}{\|y\|}.
    \end{align*}
\end{enumerate}
\end{proof}

\medskip

\section{Schur Decomposition}

\begin{theorem}
Let $A$ be an $n\times n$ matrix. There exists unitary matrix $U$ such that
$$A = UTU^*$$
for some upper triangular matrix $T$.
\end{theorem}
\begin{proof}
First we pick $\lambda_1\in\sigma(A)$ and let $N_1 = N_{(A - \lambda_1 I)}$. Assume $k_1 = \dim N_1 \geq 1$, then there exists orthogonal basis $\{x_1,\cdots,x_{k_1}\}$ for $N_1$. We can complete this basis to an orthogonal basis $\{x_1,\cdots,x_{k_1},x_{k_1+1},\cdots,x_n\}$ of $\mathbb{C}^n$. Let $U_1 = (x_1,\cdots,x_n)$, where $x_j$ is $j$th column of the matrix $U_1$, and 
\begin{align*}
    Ax_j & = \lambda_1 x_j, 1\leq j\leq k_1 \\
    Ax_j & = \sum^n_{k=1}b_{kj} \lambda_k, k_1+1\leq j\leq n. 
\end{align*}
Then we have 
\begin{align*}
    A(x_1,\cdots,x_n) & = (x_1,\cdots,x_n) \begin{pmatrix}
    \lambda_1 I_{k_1} & B_{12} \\
     & B_{22}
    \end{pmatrix},
\end{align*}
where
\begin{align*}
    \begin{pmatrix}
     B_{12} \\
     B_{22}
    \end{pmatrix} = \begin{pmatrix}
    b_{1,k_1+1} & \cdots & b_{1,n} \\
    \vdots & \ddots & \vdots \\
    b_{n,k_1+1} & \cdots & b_{n,n}
    \end{pmatrix}.
\end{align*}
Then we have 
\begin{align*}
    A = U_1 
    \begin{pmatrix}
    \lambda_1 I_{k_1} & B_{12} \\
     & B_{22}
    \end{pmatrix} U_1^*.
\end{align*}

Second, let $\lambda_2\in \sigma(B_{22})$, and $k_2 = \dim N_2$, where $N_2 = N_{(B_{22}-\lambda_2 I)}$. Then there exists unitary matrix $U_2$ such that 
\begin{align*}
    B_{22} = U_2 \begin{pmatrix}
    \lambda_2 I_{k_2} & C_{12} \\
     & C_{22}
    \end{pmatrix} U_2^*.
\end{align*}
Continue this process and we can obtain an upper triangular matrix.
\end{proof}

\medskip

\begin{theorem}
Let $Ab = BA$, then $A$ and $B$ can be simultaneously upper diagonalized by a unitary matrix.
\end{theorem}
\begin{proof}
Let $\lambda_1\in\sigma(A), N_1 = N_{(A - \lambda_1 I)}$, then we can know $B$ is invariant under $N_1$, i.e., $B:N_1\to N_1$. Then we have 
\begin{align*}
    A & = U_1 \begin{pmatrix}
    \lambda_1 I_{k_1} & A_{12} \\
     & A_{22}
    \end{pmatrix} U_1^* \\
    B & = U_1 \begin{pmatrix}
    \mu_1 I_{k_1} & B_{12} \\
     & B_{22}
    \end{pmatrix} U_1^*,
\end{align*}
and we claim $A_{22}B_{22} = B_{22}A_{22}$. Then this process can continue.
\end{proof}

\medskip

\begin{theorem}
If $AB = BA$, then $r(A+B)\leq r(A)+r(B)$.
\end{theorem}
\begin{proof}
With previous theorem, $A,B$ can be simultaneously upper diagonalized by a unitary matrix, then $A = UT_1U^*, B = UT_2U^*$, where $U_1, U_2$ are upper triangular matrix where the diagonal components are eigenvalues of $A$ and $B$. Then we have
$A+b = U(T_1+T_2)U^*$, and 
\begin{align*}
    T_1+T_2 = \begin{pmatrix}
    \lambda_1 + \mu_1 &  &  \\
     & \ddots &  \\
     &  & \lambda_n + \mu_n
    \end{pmatrix}
\end{align*}
where $\lambda_j\in\sigma(A), \mu_j\in\sigma(B)$. Thus we have
\begin{align*}
    r(A+B) = \max (\lambda_j + \mu_j) \leq \max \lambda_j + \max \mu_j = r(A)+r(B).
\end{align*}
\end{proof}




\chapter{Calculus of Vector and Matrix valued Functions}

\section{Convergence in Norm}

Let $A(t)$ be a matrix valued function, $t\in\mathbb{R}$ and $A(t)$ is an $m\times n$ matrix. 

\begin{definition}
We say $A(t)$ is continuous at $t_0\in I$, where $I$ is an open interval, if 
$$\lim_{t\to t_0}\left\|A(t) - A(t_0)\right\| = 0.$$
We say $A(t)$ is differentiable at $t_0\in I$, with derivative $\dot{A}(t_0) = \dv{A(t)}{t}\bigg|_{t=t_0}$, if 
$$\lim_{h\to 0}\left\|\frac{A(t_0+h) - A(t_0)}{h} - \dot{A}(t_0)\right\| = 0.$$
\end{definition}

\begin{remark}
Different norms for finite-dimensional spaces are equivalent. So
the above definition will not depend on the norm we use.
\end{remark}

\begin{remark}
Continuity and differentiability is equivalent to those of every element of $A(t)$. If $A(t) = \left(a_{ij}(t) \right)$, then $\dot{A}(t) = \left(\dv{}{t}a_{ij}(t)\right)$.
\end{remark}

















\addcontentsline{toc}{chapter}{Bibliography}
\bibliographystyle{plain}
\bibliography{bibliography}

\end{document}
