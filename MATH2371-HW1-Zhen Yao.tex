\documentclass[12pt,leqno]{amsart}
\pagestyle{plain}
\usepackage{latexsym,amsmath,amssymb}
%\usepackage[notref,notcite]{showkeys}
\usepackage{amsfonts}
\usepackage{geometry}
\usepackage{graphicx}
\graphicspath{ {images/} }

\setlength{\oddsidemargin}{1pt}
\setlength{\evensidemargin}{1pt}
\setlength{\marginparwidth}{30pt} % these gain 53pt width
\setlength{\topmargin}{1pt}       % gains 26pt height
\setlength{\headheight}{1pt}      % gains 11pt height
\setlength{\headsep}{1pt}         % gains 24pt height
%\setlength{\footheight}{12 pt} 	  % cannot be changed as number must fit
\setlength{\footskip}{24pt}       % gains 6pt height
\setlength{\textheight}{650pt}    % 528 + 26 + 11 + 24 + 6 + 55 for luck
\setlength{\textwidth}{460pt}     % 360 + 53 + 47 for luck

\newtheorem{theorem}{Theorem}



\def\dsp{\def\baselinestretch{1.35}\large
\normalsize}
%%%%This makes a double spacing. Use this with 11pt style. If you
%%%%want to use this just insert \dsp after the \begin{document}
%%%%The correct baselinestretch for double spacing is 1.37. However
%%%%you can use different parameter.


\def\U{{\mathcal U}}

\begin{document}

\centerline{\bf Homework 1 for Math 2371}
\centerline{Zhen Yao}



\noindent{\bf Problem 1.} Let $A > 0$. Show that 
$$A + A^{-1} \geq 2I.$$
\begin{proof}
Since $A > 0$, then there exists a unique matrix $U$ such that $A = U \Lambda U^*$, where 
\begin{align*}
    \Lambda = \begin{pmatrix}
    \lambda_1 & 0 & 0  \\
    0 & \ddots & 0  \\
    0 & 0 & \lambda_n 
    \end{pmatrix}.
\end{align*}
and $\lambda_j > 0, 1\leq j \leq n$. Then, we have
\begin{align*}
    A + A^{-1} - 2I & = U \Lambda U^* + U \Lambda^{-1} U^* - 2 U I U^* \\
    & = U\left(\Lambda + \Lambda^{-1} - 2I \right)U^* \\
    & = U \begin{pmatrix}
    \lambda_1+\lambda_1^{-1}-2 & 0 & 0  \\
    0 & \ddots & 0  \\
    0 & 0 & \lambda_n+\lambda_n^{-1}-2 
    \end{pmatrix} U^* \\
    & = U \overline{\Lambda} U^*.
\end{align*}
and we can use the function $f(x) = x + \frac{1}{x} - 2$ to obtain that if $x > 0$, then $f(x) \geq 0$. Thus, we can know every diagonal entry of $\overline{\Lambda}$ is greater or equal to $0$, which implies $A + A^{-1} - 2I = U \overline{\Lambda} U^* \geq 0$.
\end{proof}

\medskip

\noindent{\bf Problem 2.} Suppose $A,B > 0$. Show that $$\det (A+B) \geq 2\sqrt{\det A \det B}.$$
\begin{proof}
With inequality $\det (tA + (1-t)B) \geq (\det A)^t (\det B)^{1-t}$, we can substitute $A, B$ by $2A, 2B$, which also satisfy $2A, 2B > 0$ and choose $t = 1/2$, then we have
\begin{align*}
    \det (A+B) & = \det \left(\frac{1}{2} (2A) + \frac{1}{2} (2B)\right) \\
    & \geq (\det 2A)^{\frac{1}{2}} (\det 2B)^{\frac{1}{2}} \\
    & = \sqrt{2^{n+1} \det A \det B} \\
    & = 2^n \sqrt{\det A \det B} \\
    & \geq 2 \sqrt{\det A \det B}, \,\,{\rm for}\,\, n \geq 1.
\end{align*}
where in the third step we assume $A, B$ are $n\times n$ matrices. Thus the proof is complelte.
\end{proof}

\medskip

\noindent{\bf Problem 3.} Let $A,B$ be two real positive matrices. Suppose that $AB = BA$, show that $AB > 0$. 
\begin{proof}
Since $Ab = BA$, we can know that $A,B$ have the same eigensystem, i.e., if $x_j$ is an eigenvector of $A$ corresponding to eigenvalue $\lambda_j$, then $Bx_j$ is also an eigenvector of $A$ and $x_j$ is also an eigenvector of $B$. 

Let $x_1, \cdots, x_n$ be a basis consisting of eigenvectors of $B$ corresponding to eigenvalues $\mu_1,\cdots,\mu_n$. Then we have
\begin{align*}
    \left(x_j, ABx_j\right) = \left(x_j, A\mu_j x_j\right) = \left(x_j, \mu_j Ax_j\right)
\end{align*}
also, since $A,B > 0$, then all $\mu_j > 0$, which imlpies $\left(x_j, \mu_j Ax_j\right) > 0$ for all $x_j$. Thus, $AB > 0$.
\end{proof}

\medskip

\noindent{\bf Problem 4.} Given $m$ positive numbers $r_j, 1\leq j \leq m$. Show that 
\begin{align*}
    G = \left(\frac{1}{r_i + r_j + 1}\right)_{m\times m}
\end{align*}
is positive definite.
\begin{proof}
Let $f_j = x^{r_j}, 1\leq j \leq m$, and we define $(f,g) = \int^1_0 f(x)g(x)dx$, then we have 
\begin{align*}
    (f_i, f_j) = \int^1_0 x^{r_i}x^{r_j}dx = \frac{1}{r_i + r_j + 1}.
\end{align*}
Thus, $G$ is a Gram matrix, then $G$ is positive definite.
\end{proof}



\end{document}

