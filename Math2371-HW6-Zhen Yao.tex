\documentclass[12pt]{article}
\pagestyle{plain}
\usepackage[utf8]{inputenc}
\usepackage[english]{babel}

\usepackage{latexsym,amsmath,amssymb}
\usepackage{amsthm}
%\usepackage[notref,notcite]{showkeys}
\usepackage{amsfonts}
\usepackage{geometry}
\usepackage{graphicx}
\usepackage{lmodern}
\usepackage{pifont}
\usepackage{tikz}
\usepackage{pgfplots}
\usepackage{thmtools}
\usepackage{wrapfig}
\usepackage{extarrows}
\usepackage{breqn}
\usepackage{physics}
\usepackage{afterpage}
\usepackage{enumitem}
\usepackage[utf8]{inputenc}
\usepackage{mathrsfs}
\usepackage{scalerel}
\usepackage{stackengine,wasysym}
\usepackage{aligned-overset}
\usepackage{stackengine}
\usepackage{mathtools}
\usepackage{nccmath}
\graphicspath{ {images/} }

\setlength{\oddsidemargin}{1pt}
\setlength{\evensidemargin}{1pt}
\setlength{\marginparwidth}{30pt} % these gain 53pt width
\setlength{\topmargin}{1pt}       % gains 26pt height
\setlength{\headheight}{1pt}      % gains 11pt height
\setlength{\headsep}{1pt}         % gains 24pt height
%\setlength{\footheight}{12 pt} 	  % cannot be changed as number must fit
\setlength{\footskip}{24pt}       % gains 6pt height
\setlength{\textheight}{650pt}    % 528 + 26 + 11 + 24 + 6 + 55 for luck
\setlength{\textwidth}{460pt}     % 360 + 53 + 47 for luck

\newtheorem{theorem}{Theorem}



\def\dsp{\def\baselinestretch{1.35}\large
\normalsize}
%%%%This makes a double spacing. Use this with 11pt style. If you
%%%%want to use this just insert \dsp after the \begin{document}
%%%%The correct baselinestretch for double spacing is 1.37. However
%%%%you can use different parameter.


\def\U{{\mathcal U}}

\begin{document}

\centerline{\bf Homework 6 for Math 2371}
\centerline{Zhen Yao}

\medskip

\noindent{\bf Problem 1.}
Let $A$ be a $2\times 2$ complex matrix. Show that $\|A\| = r(A)$ if and only if $A$ is normal. Show that the statement is not true if $A$ is an $n\times n$ matrix with $n \geq 3$.
\begin{proof}
~\begin{enumerate}[label=(\arabic*)]
    \item Suppose $A_{2\times 2}$ is complex matrix and $AA^* = A^* A$.
    \begin{enumerate}
        \item[a)] $r(A) \leq \|A\|$. Indeed, for eigenvalue $\lambda_j$ and the corresponding eigenvector $x_j$, we have $Ax_j = \lambda_j x_j$ and 
        \begin{align*}
            \|Ax_j\| = \left|\lambda_j\right| \|x_j\| \leq \|A\| \|x_j\|.
        \end{align*}
        Hence, $\max|\lambda_j| = r(A) \leq \|A\|$.
        
        \item[b)]$A$ is normal, then there exists orthogonal basis of $X$ consisting of eigenvectors of $A$. For any $x\in X$, $x = \sum^n_{i=1} c_j x_j$, applying $A$ to $x$ gives
        \begin{align*}
            Ax = \sum^n_{i=1} c_j \lambda_j x_j.
        \end{align*}
        Then we have
        \begin{align*}
            \frac{\|Ax\|}{\|x\|} = \left(\frac{\sum^n_{i=1} c_j^2 \lambda_j^2 }{\sum^n_{i=1} c_j^2}\right)^{\frac{1}{2}} \leq \max \lambda_j = r(A).
        \end{align*}
        Hence, $\|A\| = \frac{\|Ax\|}{\|x\|} \leq r(A)$.
    \end{enumerate}
    Thus, $\|A\| = r(A)$.
    
    \item If $\|A\| = r(A)$, then we set $A = \begin{pmatrix}  
        a & b \\
        c & d
    \end{pmatrix}$. Also, we have $\|A\| = \sqrt{A^*A}$. Then after computing, we have $b = c$, which implies $A$ is normal.
    
    \item Now take $A = \begin{pmatrix} 
        1 & 1 & 0 \\
        0 & 1 & 0 \\
        0 & 0 & 2
    \end{pmatrix}$, then $\|A\| = 2 = \sqrt{r(A^* A)}$. However, $AA^* - A^* A \neq 0$, which implies $A$ is not normal. 
\end{enumerate}
\end{proof}

\medskip

\noindent{\bf Problem 2.}
Construct a continuous matrix function $A(t)$ where $A(t)$ is an anti-symmetric real $3\times 3$ matrix for each $t$ such that the unique solution to
\begin{align*}
    \left\{ 
    \begin{aligned}
        & M_t = A(t)M,\\
        & M(0) = I,
    \end{aligned}
    \right.
\end{align*}
is not give by $\exp\left(\int^t_0 A(s)\, ds \right)$.
\begin{proof}
We can set $A$ as 
\begin{align*}
    A = \begin{pmatrix}
        0 & t & 0 \\
        -t & 0 & 0 \\
        0 & 0 & 0 
    \end{pmatrix},
\end{align*}
and we can solve for $M$, which is
\begin{align*}
    M = \begin{pmatrix}
        \sin \frac{t^2}{2} + \cos \frac{t^2}{2} &  \sin \frac{t^2}{2} & 0 \\
        \cos \frac{t^2}{2} - \sin \frac{t^2}{2} &  \cos \frac{t^2}{2} & 0 \\
        0 & 0 & 1 
    \end{pmatrix}.
\end{align*}
Also, we have 
\begin{align*}
    e^{\int^t_0 A(s)\, ds} = \begin{pmatrix}
        1 & \frac{1}{2} t^2 & 0 \\
        -\frac{1}{2} t^2 & 1 & 0 \\
        0 & 0 & 1 
    \end{pmatrix}.
\end{align*}
Then when $A$ depends on $t$ and $A(t)$ does not commute with $\int^t_0 A(s)\, ds$, the solution cannot be represented by $e^{\int^t_0 A(s)\, ds}$.
\end{proof}

\medskip

\noindent{\bf Problem 3.}
Let 
\begin{align*}
    K = \left\{(x,y)\in\mathbb{R}^2: |x| < 1, |y| < 1 \right\}.
\end{align*}
Show that $K$ is convex and find its gauge function $p_K$.
\begin{proof}
~\begin{enumerate}[label=(\arabic*)]
    \item For any $x = (x_1, x_2), y = (y_1, y_2) \in K$ and $t\in (0,1)$, we define $z = (tx_1 + (1-t)y_1,tx_2 + (1-t)y_2)$. And we have
    \begin{align*}
        |tx_1 + (1-t)y_1| \leq t|x_1| + (1-t)|x_2| \leq \max \{x_1, x_2\} < 1,
    \end{align*}
    similarly, $|tx_2 + (1-t)y_2| < 1$. Thus we have $z\in K$, and hence $K$ is convex.
    
    \item $p_K = \max \{x,y\}$.
\end{enumerate}

\end{proof}


\end{document}