\documentclass[12pt]{article}
\pagestyle{plain}
\usepackage[utf8]{inputenc}
\usepackage[english]{babel}

\usepackage{latexsym,amsmath,amssymb}
\usepackage{amsthm}
%\usepackage[notref,notcite]{showkeys}
\usepackage{amsfonts}
\usepackage{geometry}
\usepackage{graphicx}
\usepackage{lmodern}
\usepackage{pifont}
\usepackage{tikz}
\usepackage{pgfplots}
\usepackage{thmtools}
\usepackage{wrapfig}
\usepackage{extarrows}
\usepackage{breqn}
\usepackage{physics}
\usepackage{afterpage}
\usepackage{enumitem}
\usepackage[utf8]{inputenc}
\usepackage{mathrsfs}
\usepackage{scalerel}
\usepackage{stackengine,wasysym}
\usepackage{aligned-overset}
\usepackage{stackengine}
\usepackage{mathtools}
\usepackage{nccmath}
\graphicspath{ {images/} }

\setlength{\oddsidemargin}{1pt}
\setlength{\evensidemargin}{1pt}
\setlength{\marginparwidth}{30pt} % these gain 53pt width
\setlength{\topmargin}{1pt}       % gains 26pt height
\setlength{\headheight}{1pt}      % gains 11pt height
\setlength{\headsep}{1pt}         % gains 24pt height
%\setlength{\footheight}{12 pt} 	  % cannot be changed as number must fit
\setlength{\footskip}{24pt}       % gains 6pt height
\setlength{\textheight}{650pt}    % 528 + 26 + 11 + 24 + 6 + 55 for luck
\setlength{\textwidth}{460pt}     % 360 + 53 + 47 for luck

\newtheorem{theorem}{Theorem}



\def\dsp{\def\baselinestretch{1.35}\large
\normalsize}
%%%%This makes a double spacing. Use this with 11pt style. If you
%%%%want to use this just insert \dsp after the \begin{document}
%%%%The correct baselinestretch for double spacing is 1.37. However
%%%%you can use different parameter.


\def\U{{\mathcal U}}

\begin{document}

\centerline{\bf Homework 7 for Math 2371}
\centerline{Zhen Yao}

\medskip

\noindent{\bf Problem 1.}
Let 
\begin{align*}
    K = \left\{(x,y)\in \mathbb{R}^2: |x| < 1, |y| < 2\right\}.
\end{align*}
Find its support function $q_K$. Here you can identify any $l\in \left(\mathbb{R}^2\right)'$ as $(l_1,l_2)\in \mathbb{R}^2$ such that
\begin{align*}
    l(x+y) = l_1 x + l_2 y.
\end{align*}
\begin{proof}
First we consider the region $x > 0, y > 0$ and split it into two regions $R_1 = \{x > 0, y < 2x\}$ and $R_2 = \{x > 0, y \geq 2x\}$. In $R_1$, for any point $(x,y) \in \mathbb{R}^2$, we have \begin{align*}
    (x,y)\cdot\left(1, \frac{y}{x}\right) & = x + \frac{y^2}{x} \\
    & \leq x + \frac{2y^2}{y} \\
    & = x + 2y = (x,y)\cdot(1,2),
\end{align*}
which implies $q_K(l) = x + 2y$, where $l_1 = 1$ and $l_2 = 2$. Similarly, we can have for region $R_2$, $q_K(l) = x + 2y$. With the same argument, we can have
\begin{align*}
    q_K(l) = \left\{
    \begin{aligned}
        & x + 2y, & x > 0, y > 0 \\
        & -x - 2y, & x \leq 0, y < 0 \\
        & -a, & x < 0, b = 0 \\
        & a, & x > 0, b = 0 \\
        & x - 2y, & x > 0, b < 0 \\
        & -x + 2y, & x \leq 0, y > 0.
    \end{aligned}
    \right.
\end{align*}
\end{proof}

\medskip

\noindent{\bf Problem 2.}
Let $K$ be a bounded nonempty closed convex set of a finite dimensional linear space $X$. Show there exists at least one extreme point of $K$.
\begin{proof}
Since $K$ is a bounded and closed convex set in finite dimensional space $X$, then we can find $x\in K$ such that $\|x\|$ is maximized. We prove by contradiction and suppose that there is a $y\neq 0$ such that $x+y \in K$, then 
\begin{align*}
    2\|x\|^2 \geq \|x+y\|^2 + \|x-y\|^2 = 2\|x\|^2 + 2\|y\|^2 > 2\|x\|^2,
\end{align*}
which is a contradiction. Thus, $K$ has at least one extreme point.
\end{proof}

\medskip

\noindent{\bf Problem 3.}
Let $K$ be a bounded nonempty closed convex set of a linear space $X$ with $\dim X = n$. Suppose in addition that $K$ contains at least one interior point. Show that $K$ has at least $n+1$ extreme points. 
\begin{proof}
With Caratheodory theorem, we can know that every point in $K$ can be represented by at most $n+1$ extreme points, then we can know that $K$ at least $n+1$ extreme points.
\end{proof}





\end{document}