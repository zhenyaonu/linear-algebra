\documentclass[12pt]{article}
\pagestyle{plain}
\usepackage[utf8]{inputenc}
\usepackage[english]{babel}

\usepackage{latexsym,amsmath,amssymb}
\usepackage{amsthm}
%\usepackage[notref,notcite]{showkeys}
\usepackage{amsfonts}
\usepackage{geometry}
\usepackage{graphicx}
\usepackage{lmodern}
\usepackage{pifont}
\usepackage{tikz}
\usepackage{pgfplots}
\usepackage{thmtools}
\usepackage{wrapfig}
\usepackage{extarrows}
\usepackage{breqn}
\usepackage{physics}
\usepackage{afterpage}
\usepackage{enumitem}
\usepackage[utf8]{inputenc}
\usepackage{mathrsfs}
\usepackage{scalerel}
\usepackage{stackengine,wasysym}
\usepackage{aligned-overset}
\usepackage{stackengine}
\usepackage{mathtools}
\usepackage{nccmath}
\graphicspath{ {images/} }

\setlength{\oddsidemargin}{1pt}
\setlength{\evensidemargin}{1pt}
\setlength{\marginparwidth}{30pt} % these gain 53pt width
\setlength{\topmargin}{1pt}       % gains 26pt height
\setlength{\headheight}{1pt}      % gains 11pt height
\setlength{\headsep}{1pt}         % gains 24pt height
%\setlength{\footheight}{12 pt} 	  % cannot be changed as number must fit
\setlength{\footskip}{24pt}       % gains 6pt height
\setlength{\textheight}{650pt}    % 528 + 26 + 11 + 24 + 6 + 55 for luck
\setlength{\textwidth}{460pt}     % 360 + 53 + 47 for luck

\newtheorem{theorem}{Theorem}



\def\dsp{\def\baselinestretch{1.35}\large
\normalsize}
%%%%This makes a double spacing. Use this with 11pt style. If you
%%%%want to use this just insert \dsp after the \begin{document}
%%%%The correct baselinestretch for double spacing is 1.37. However
%%%%you can use different parameter.


\def\U{{\mathcal U}}

\begin{document}

\centerline{\bf Homework 4 for Math 2371}
\centerline{Zhen Yao}

\medskip

\noindent{\bf Problem 1.}
Suppose $A$ and $B$ are $2 \times 2$ nilpotent matrices. Prove that $AB$ is diagonalizable.
\begin{proof}
~\begin{enumerate}[label=(\arabic*)]
    \item If $A$ or $B$ is zero, then it is trivial.
    \item If $A,B \neq 0$, since $A, B$ are nilpotent, then $\text{tr}(A) = \text{tr}(B) = 0$ and the characteristic polynomials of $A, B$ are $\lambda^2$. Then let $A = \begin{pmatrix} 
        a & b  \\
        c & -a 
    \end{pmatrix}$ and $B = \begin{pmatrix} 
        d & e  \\
        f & -d 
    \end{pmatrix}$, where $-a^2 = bc, -d^2 = ef$ and $a,b,c,d,e,f \neq 0$. Then we have $AB = \begin{pmatrix} 
        ad+bf & ae-bd  \\
        cd-afc & ce+ad 
    \end{pmatrix}$, and we can have its characteristic polynomial
    \begin{align*}
        P_{AB}(\lambda) = \lambda^2  - \left(\sqrt{bf} + \frac{ad}{\sqrt{bf}} \right)\lambda.
    \end{align*}
    Hence, $AB$ has two different eigenvalues. Thus, $AB$ is diagonalizable.
\end{enumerate}
\end{proof}

\medskip

\noindent{\bf Problem 2.}
Let $Q$ be a reflection clockwise rotation with angel $\theta$ about the origin in $\mathbb{R}^2$. Find the matrix representation of $Q$ in the standard basis and express $Q$ as a composition of two reflections.
\begin{proof}
For any $u\in\mathbb{R}^2$, it can be represented as $u = \begin{pmatrix} 
    r \cos \alpha \\
    r \sin\alpha  
\end{pmatrix}$, and we can have $Qr = \begin{pmatrix} 
    r \cos (\alpha+\theta) \\
    r \sin (\alpha+\theta)  
\end{pmatrix}$, then we have 
$$Q = \begin{pmatrix} 
    \cos \theta & - \sin \theta \\
    \sin \theta & \cos \theta  
\end{pmatrix}.$$

Now consider the reflection about the line $y = \tan \frac{\theta}{2} x$ and it can be represented as 
$$R_{\frac{\theta}{2}} = \begin{pmatrix} 
    \cos \theta & \sin \theta \\
    \sin \theta & - \cos \theta  
\end{pmatrix}.$$
then we have 
\begin{align*}
    R_{\frac{\alpha}{2}} R_{\frac{\beta}{2}} = \begin{pmatrix} 
    \cos (\alpha - \beta) & \sin (\alpha - \beta) \\
    \sin (\alpha - \beta) & - \cos (\alpha - \beta)  
\end{pmatrix}.
\end{align*}
Let $\alpha = \theta - \beta$, then we can have $Q = R_{\frac{\alpha}{2}} R_{\frac{\beta}{2}}$.
\end{proof}

\medskip

\noindent{\bf Problem 3.}
Let $R$ be a reflection and $Q$ be an orthogonal matrix in $\mathbb{R}^n$, show that $QRQ^T$ is also a reflection. Explain the connection of two reflections geometrically.
\begin{proof}
Since $Q$ is orthogonal matrix, then $Q$ can be written as a composition of at most $n$ reflections. Then the composition $QRQ^T$ is also reflection.
\end{proof}

\medskip

\noindent{\bf Problem 4.}
Show that every orthogonal map $Q$ is the composition of at most $k$ hyperplane reflections, where
\begin{align*}
    k = n - \dim (\ker(Q - I)).
\end{align*}

















\end{document}

