\documentclass[12pt]{article} 
%\documentclass[12pt,leqno]{amsart}
\pagestyle{plain}
\usepackage[utf8]{inputenc}
\usepackage[english]{babel}

\usepackage{latexsym,amsmath,amssymb}
\usepackage{amsthm}
%\usepackage[notref,notcite]{showkeys}
\usepackage{amsfonts}
\usepackage{geometry}
\usepackage{graphicx}
\usepackage{lmodern}
\usepackage{pifont}
\usepackage{tikz}
\usepackage{pgfplots}
\usepackage{thmtools}
\usepackage{wrapfig}
\usepackage{extarrows}
\usepackage{breqn}
\usepackage{physics}
\usepackage{afterpage}
\usepackage{enumitem}
\usepackage[utf8]{inputenc}
\usepackage{mathrsfs}
\usepackage{scalerel}
\usepackage{stackengine,wasysym}
\usepackage{aligned-overset}
\usepackage{stackengine}
\usepackage{mathtools}
\usepackage{nccmath}
\graphicspath{ {images/} }

\setlength{\oddsidemargin}{1pt}
\setlength{\evensidemargin}{1pt}
\setlength{\marginparwidth}{30pt} % these gain 53pt width
\setlength{\topmargin}{1pt}       % gains 26pt height
\setlength{\headheight}{1pt}      % gains 11pt height
\setlength{\headsep}{1pt}         % gains 24pt height
%\setlength{\footheight}{12 pt} 	  % cannot be changed as number must fit
\setlength{\footskip}{24pt}       % gains 6pt height
\setlength{\textheight}{650pt}    % 528 + 26 + 11 + 24 + 6 + 55 for luck
\setlength{\textwidth}{460pt}     % 360 + 53 + 47 for luck

\newtheorem{theorem}{Theorem}


\def\dsp{\def\baselinestretch{1.35}\large
\normalsize}
%%%%This makes a double spacing. Use this with 11pt style. If you
%%%%want to use this just insert \dsp after the \begin{document}
%%%%The correct baselinestretch for double spacing is 1.37. However
%%%%you can use different parameter.


\def\U{{\mathcal U}}

\begin{document}

\centerline{\bf Homework 10 for Math 2371}
\centerline{Zhen Yao}

\medskip

\noindent{\bf Problem 1.}
Let $V$ be a finite dimensional vector space over $\mathbb{R}$ and $T: V \to V$ be linear. Show that for any subspace $W$ of $V$, 
\begin{align*}
    \dim T^{-1}(W) \leq \dim N_T + \dim W.
\end{align*}
\begin{proof}
Suppose $U \subset V$ such that $T(U) = W$, then for $T|_U:U \to W$, we have
\begin{align*}
    \dim T(U) + \dim N_{T|_U} = \dim U = \dim T^{-1}(W).
\end{align*}
Also, with $\dim N_{T|_U} \leq \dim N_T$, we have 
\begin{align*}
    \dim T^{-1}(W) \leq \dim N_T + \dim T(U) = \dim N_T + \dim W.
\end{align*}
\end{proof}

\medskip

\noindent{\bf Problem 2.}
Suppose $A$ and $B$ are $n \times n$ matrices, and $A + B$ is invertible. Prove that 
\begin{align*}
    \rank A + \rank B \geq n.
\end{align*}
Also, show that 
\begin{align*}
    \rank A + \rank B = n
\end{align*}
if and only if
\begin{align*}
    R_A \cap R_B = \{0\}.
\end{align*}
\begin{proof}
~\begin{enumerate}[label=(\alph*)]
    \item Since $A + B$ is invertible, then $A + B$ is full rank, which implies $\rank (A+B) = n$ and $N_{A+B} = \{0\}$. Then, $\dim N_{A+B} = 0$, and we have 
    \begin{align*}
        \dim (N_A + N_B) = \dim N_A + \dim N_B - \dim (N_A \cap N_B).
    \end{align*}
    Also, for $x\in N_A \cap N_B$, then $(A + B)x = 0$, hence $N_A \cap N_B \subset N_{A+B}$. Then we have $\dim (N_A \cap N_B) = 0$, which yields
    \begin{align*}
        \dim N_A + \dim N_B = \dim (N_A + N_B) \leq n.
    \end{align*}
    With rank-nullity theorem, we have
    \begin{align*}
        \rank A + \rank B = n - \dim N_A + n - \dim N_B \geq n.
    \end{align*}
    
    \item \begin{enumerate}[label=\arabic*)]
        \item If $\rank A + \rank B = n$, with the fact that $R_{A+B} \subset R_A + R_B$, then, $\dim (R_A + R_B) = n$,
        \begin{align*}
            n = \dim (R_A + R_B) = \rank A + \rank B - \dim (R_A \cap R_B),
        \end{align*}
        which implies $\dim (R_A \cap R_B)$. Hence, $R_A \cap R_B = \{0\}$.
        
        \item If $R_A \cap R_B = \{0\}$, then $\dim (R_A \cap R_B)$. Thus,
        \begin{align*}
            \rank A + \rank B = \dim (R_A + R_B) - \dim (R_A \cap R_B) = n - 0 = n.
        \end{align*}
    \end{enumerate}
\end{enumerate}
\end{proof}

\medskip

\noindent{\bf Problem 3.}
Suppose $A, B, C, D$ are $n \times n$ matrices satisfying
\begin{align*}
    AB = DB, AC = 2 DC.
\end{align*}
Show that 
\begin{align*}
    \rank A + \rank B + \rank C \leq 2n.
\end{align*}
\begin{proof}
Since $AB = DB$, then we have $(A - D)B = 0$ and thus $R_B \subset N_{A-D}$. Similarly, we have $R_C \subset N_{A-2D}$. Then, 
\begin{align*}
    \rank B & \leq \dim N_{A - D} = n -  \rank (A - D), \\
    \rank C & \leq \dim N_{A - 2D} = n - \rank (A - 2D).
\end{align*}
Then, we want to prove that $\rank A \leq \rank (A - D) + \rank (A - 2D)$.
\end{proof}

\medskip

\noindent{\bf Problem 4.}
Suppose that $A_{n\times n}, B_{n\times m}, C_{m\times n}$ and $D_{m\times m}$ are matrices such that $\det A \neq 0$. Show that
\begin{align*}
    \det \begin{pmatrix}
        A & B \\
        C & D
    \end{pmatrix} = \det A \det \left( D - C A^{-1} B\right).
\end{align*}
\begin{proof}
With elementary row operation, multiplying $- C A^{-1}$ with the first row and adding it to the second row yields
\begin{align*}
    \begin{pmatrix}
        A & B \\
        C & D
    \end{pmatrix} = \begin{pmatrix}
        A & B \\
        0 & D - C A^{-1} B,
    \end{pmatrix}
\end{align*}
since $\det A \neq 0$, and hence $A^{-1}$ exists. And it is obviously that 
\begin{align*}
    \det \begin{pmatrix}
        A & B \\
        C & D
    \end{pmatrix} = \det A \det \left( D - C A^{-1} B\right).
\end{align*}
\end{proof}

\medskip

\noindent{\bf Problem 5.}
Let $A, B, C, D$ be $n \times n$ matrices and 
\begin{align*}
    E = \begin{pmatrix}
        A & B \\
        C & D
    \end{pmatrix}.
\end{align*}
\begin{enumerate}[label=(\alph*)]
    \item Prove that 
    \begin{align*}
        \det E = \det (AD - BC)
    \end{align*}
    when all matrices $A, B, C, D$ are diagonal.
    
    \item Prove that 
    \begin{align*}
        \det E = \det (AD - BC)
    \end{align*}
    when all matrices $A, B, C, D$ are upper triangular. 
    
    \item Prove that
    \begin{align*}
        \det E = \det (AD - BC)
    \end{align*}
    when all matrices $A, B, C, D$ commute. 
\end{enumerate}
\begin{proof}
~\begin{enumerate}[label=(\alph*)]
    \item
    \begin{enumerate}[label=\arabic*)]
        \item If $A$ is invertible, then, with Problem 4, we have 
        \begin{align*}
            \det E & = det A \det \left( D - C A^{-1} B\right) \\
            & = \det \left(AD - A C A^{-1} B\right) \\
            & = \det \left(AD - C A A^{-1} B\right) \\
            & = \det \left(AD - CB\right)\\
            & = \det \left(AD - BC\right),
        \end{align*}
        where in the last two step we used the fact that $AC = CA$ and $BC = CB$ since $A,C,B$ are diagonal.
        \item If $A$ is not invertible, then there exist $\varepsilon_k \to 0$ such that 
        \begin{align*}
            \det A_k = \det (A + \varepsilon_k I) \neq 0.
        \end{align*}
        Then, we have $A_k C = CA_k$. Thus, with similar argument in $1)$, 
        \begin{align*}
            \det E = \lim_{k\to \infty} \begin{pmatrix}
            A_k & B \\
            C & D
        \end{pmatrix} = \lim_{k\to\infty} \det \left(A_k D - BC\right) = \det \left(AD - BC\right).
        \end{align*}
    \end{enumerate}
    
    \item If $D$ is invertible, then similar to Problem 4, we have $\det E = \det (A-BD^{-1}C)\det(D)$. Since $D$ is upper triangular, then so is $D^{-1}$. Then, 
    \begin{align*}
        \det (A-BD^{-1}C) & = \prod \left(A_{ii} - B_{ii} D^{-1}_{ii} C_{ii}\right) \\
        & = \prod \left(A_{ii} - B_{ii} C_{ii} D^{-1}_{ii} \right) \\
        & = \det (A-BCD^{-1}).
    \end{align*}
    It follows that
    \begin{align*}
        \det E = \det (A-BCD^{-1})\det(D) = \det (AD - BC).
    \end{align*}
    If $D$ is not invertible, with the similar argument in $(a)\,\,2)$, the result follows easily.
    
    \item If $D$ is invertible, and $A, B, C, D$ commute, then 
    \begin{align*}
        \det E & = \det (A-BD^{-1}C) \det(D) \\
        & = \det (A-BD^{-1}CD) \\
        & = \det (A-BD^{-1}DC) \\
        & = \det (A-BC).
    \end{align*}
    If $D$ is not invertible, with the similar argument in $(a)\,\,2)$, the result follows easily.
\end{enumerate}
\end{proof}




\end{document}