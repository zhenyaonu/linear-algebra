\documentclass[12pt,leqno]{amsart}
\pagestyle{plain}
%\documentclass{article}
%\usepackage[utf8]{inputenc}
%\usepackage[english]{babel}

\usepackage{latexsym,amsmath,amssymb}
\usepackage{amsthm}
%\usepackage[notref,notcite]{showkeys}
\usepackage{amsfonts}
\usepackage{geometry}
\usepackage{graphicx}
\usepackage{lmodern}
\usepackage{pifont}
\usepackage{tikz}
\usepackage{pgfplots}
\usepackage{thmtools}
\usepackage{wrapfig}
\usepackage{extarrows}
\graphicspath{ {images/} }


\setlength{\oddsidemargin}{1pt}
\setlength{\evensidemargin}{1pt}
\setlength{\marginparwidth}{30pt} % these gain 53pt width
\setlength{\topmargin}{1pt}       % gains 26pt height
\setlength{\headheight}{1pt}      % gains 11pt height
\setlength{\headsep}{1pt}         % gains 24pt height
%\setlength{\footheight}{12 pt} 	  % cannot be changed as number must fit
\setlength{\footskip}{24pt}       % gains 6pt height
\setlength{\textheight}{650pt}    % 528 + 26 + 11 + 24 + 6 + 55 for luck
\setlength{\textwidth}{460pt}     % 360 + 53 + 47 for luck

\title{Sections and Chapters}

\newtheorem{definition}{Definition}[section]
\newtheorem{theorem}{Theorem}[section]
\newtheorem{corollary}{Corollary}[theorem]
\newtheorem{lemma}[theorem]{Lemma}
\newtheorem{proposition}{Proposition}[section]
\newtheorem{exercise}{Exercise}[section]
\newtheorem{remark}{Remark}[section]

\theoremstyle{definition}
\newtheorem{example}{Example}[section]

\def\dsp{\def\baselinestretch{1.35}\large
\normalsize}
%%%%This makes a double spacing. Use this with 11pt style. If you
%%%%want to use this just insert \dsp after the \begin{document}
%%%%The correct baselinestretch for double spacing is 1.37. However
%%%%you can use different parameter.


\def\U{{\mathcal U}}

\begin{document}

\centerline{\bf Homework 9 for Math 2370}
\centerline{Zhen Yao}

\bigskip

\medskip

\noindent \textbf{Problem 1.} 
Let%
$$
q\left(  x\right)  =2x_{1}x_{2}-6x_{2}x_{3}+2x_{1}x_{3}.
$$
Find an invertible matrix $L$, such that%
$$
q\left(  L^{-1}y\right)  =d_{1}y_{1}^{2}+d_{2}y_{2}^{2}+d_{3}y_{3}^{2}%
$$
where $d_{i}=0$ or $\pm1$.
\begin{proof}
We have $q(x) = (x,Hx)$, where 
$$
H = \left(
\begin{matrix}
0 & 1 & 1\\
1 & 0 & -3\\
1 & -3 & 0
\end{matrix}
\right)
$$
Now we need to normalize the matrix $H$, and we can compute for its eigenvalues, which are $\lambda = 3, \frac{3-\sqrt{17}}{2}, \frac{3+\sqrt{17}}{2}$, with eigenvectors 
\begin{align*}
    \left(\begin{matrix} 0 \\ - 1\\ 1 
    \end{matrix}\right), 
    \left(\begin{matrix} \frac{3-\sqrt{17}}{2} \\ 1\\ 1 
    \end{matrix}\right), 
    \left(\begin{matrix} \frac{3+\sqrt{17}}{2} \\ 1\\ 1 
    \end{matrix}\right), 
\end{align*}
Now we can normalize these vectors and we get 
\begin{align*}
    \left(\begin{matrix} 0 \\ - \frac{1}{\sqrt{2}}\\ \frac{1}{\sqrt{2}} 
    \end{matrix}\right), 
    \left(\begin{matrix} -\sqrt{\frac{17-3\sqrt{17}}{34}} \\ \frac{2}{\sqrt{17-3\sqrt{17}}}\\ \frac{2}{\sqrt{17-3\sqrt{17}}} 
    \end{matrix}\right), 
    \left(\begin{matrix} \sqrt{\frac{17+3\sqrt{17}}{34}} \\ \frac{2}{\sqrt{17+3\sqrt{17}}} \\ \frac{2}{\sqrt{17+3\sqrt{17}}} 
    \end{matrix}\right), 
\end{align*}
And we arrange eigenvectors into a matrix, denoting it by 
\begin{align*}
    C = \left(\begin{matrix}
    0 & -\sqrt{\frac{17-3\sqrt{17}}{34}} & \sqrt{\frac{17+3\sqrt{17}}{34}} \\
    - \frac{1}{\sqrt{2}} & \frac{2}{\sqrt{17-3\sqrt{17}}} & \frac{2}{\sqrt{17+3\sqrt{17}}} \\
    \frac{1}{\sqrt{2}} & \frac{2}{\sqrt{17-3\sqrt{17}}} & \frac{2}{\sqrt{17+3\sqrt{17}}}
    \end{matrix}\right)
\end{align*}
We can verify that $C^*HC = \left(
\begin{matrix}
3 & 0 & 0\\
0 & \frac{3-\sqrt{17}}{2} & 0\\
0 & 0 & \frac{3+\sqrt{17}}{2}
\end{matrix}
\right)$. Now we denote $z = Cx = (z_1, z_2, z_3)$, where 
\begin{align*}
    z_1 & = -\sqrt{\frac{17-3\sqrt{17}}{34}}x_2 + \sqrt{\frac{17+3\sqrt{17}}{34}}x_3\\
    z_2 & = - \frac{1}{\sqrt{2}}x_1 + \frac{2}{\sqrt{17-3\sqrt{17}}}x_2 + \frac{2}{\sqrt{17+3\sqrt{17}}}x_3 \\
    z_3 & = \frac{1}{\sqrt{2}}x_1 + \frac{2}{\sqrt{17-3\sqrt{17}}}x_2 + \frac{2}{\sqrt{17+3\sqrt{17}}}x_3
\end{align*}
and we need to change variable to get the quadratic form $q\left(  L^{-1}y\right)  =d_{1}y_{1}^{2}+d_{2}y_{2}^{2}+d_{3}y_{3}^{2}$. We make the change of variable
\begin{align*}
    y_1 & = \frac{1}{\sqrt{3}}z_1 \\
    y_2 & = \sqrt{\frac{2}{3-\sqrt{17}}} z_2\\
    y_3 & = \sqrt{\frac{2}{3+\sqrt{17}}} z_3
\end{align*}
and we can denote this transform by matrix $E$, where 
\begin{align*}
    E = \left(
    \begin{matrix}
    \frac{1}{\sqrt{3}} & 0 & 0\\
    0 & \sqrt{\frac{2}{3-\sqrt{17}}} & 0\\
    0 & 0 & \sqrt{\frac{2}{3+\sqrt{17}}}
    \end{matrix}
    \right)
\end{align*}
then we can know that $L^{-1} = CE$, which are defined above. And finally, $L = (CE)^{-1}$.
\end{proof}

\medskip

\noindent \textbf{Problem 2.}
Show that the congruence is an equivalence relation for symmetric
matrices. Find the total number of equivalence classes for $n\times n$
symmetric matrices.
\begin{proof}We denote the relation of congruence by $\sim$. \\
\hspace*{1em}\, (1) For $A$ is a symmetric matrix, then we have $A\sim A$, since $A = I^TAI$, where $I$ is identity matrix. \\
\hspace*{1em}\, For $A,B$ are symmetric matrices, we have if $A\sim B$ , then $B\sim A$. Since if $A = S^TBS$, where $S$ is invertible, then we have $B = (S^T)^{-1}AS^{-1}$, which means $B\sim A$.\\
\hspace*{1em}\, For $A,B$ and $C$ are symmetric matrices, we have if $A\sim B,B\sim C$, then $A\sim C$. Since if we have $A = S^TBS$ and $B = P^TCP$, then we have $A = S^TP^TCPS = (PS)^TCPS$, which implies $A\sim C$. Then we proved the congruence is an equivalence relation. \\
\hspace*{1em}\, (2) Suppose $A = S^TBS$, and $S$ is invetible. Also, we have $R_{BS}\subseteq R_{B}$ with equality when $S$ is invertible, since $S$ is full rank. Then we have, in this case, $\dim B = \dim BS$. Then we have $S^T$ is also full rank and $\dim A = \dim S^TBS = \dim B$. So we can know that for symmetric matrices $A$ and $B$, if they are congruent then they have the same rank, which means there are $n+1$ equivalebce classes, since there are matrix with rank $0,1,2,\cdots,n$, which is $n+1$ possibilities.
\end{proof}

\medskip

\noindent \textbf{Problem 3.} 
Let $A,B$ be two $n\times n$ real orthogonal matrices satisfying
$$
\det A+\det B=0.
$$
Show there exists a unit vector $x$ such that%
$$
Ax=-Bx.
$$

\begin{proof}
Since $A$ and $B$ are orthogonal matrices, then we have $\det A = \det B = \pm 1$ and $A^TA = B^TB = I$. We prove by contradiction, and suppose that there does not exist $x\in \mathbb{R}^n$ and $\|x\| = 1$ such that $Ax = -Bx$.  \\
\hspace*{1em}\, Then for $\forall x$, $(A+B)x\neq 0$, then $(A^T+B^T)(A+B)x\neq 0$. Then we have, for $\forall x\neq 0$
\begin{align*}
    & (A^TA + A^TB - B^TA - B^TB)x \neq 0 \\
    \Rightarrow & (A^TB - B^TA)x \neq 0 \\
    \Rightarrow & A^TB - B^TA \neq 0 \\
    \Rightarrow & \det A^TB \neq \det B^TA \\
    \Rightarrow & \frac{1}{\det A}\det B \neq \frac{1}{\det B}\det A \\
    \Rightarrow & (\det A)^2 \neq (\det B)^2
\end{align*}
Since $\det A + \det B = 0$, and $\det A = \det B = \pm 1$, without losing generality, we can assume $\det A = 1$ and $\det B = -1$. Then this contradicts $(\det A)^2 \neq (\det B)^2$. The proof is complete.
\end{proof}





\end{document}


