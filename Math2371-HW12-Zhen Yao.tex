\documentclass[12pt]{article} 
%\documentclass[12pt,leqno]{amsart}
\pagestyle{plain}
\usepackage[utf8]{inputenc}
\usepackage[english]{babel}

\usepackage{latexsym,amsmath,amssymb}
\usepackage{amsthm}
%\usepackage[notref,notcite]{showkeys}
\usepackage{amsfonts}
\usepackage{geometry}
\usepackage{graphicx}
\usepackage{lmodern}
\usepackage{pifont}
\usepackage{tikz}
\usepackage{pgfplots}
\usepackage{thmtools}
\usepackage{wrapfig}
\usepackage{extarrows}
\usepackage{breqn}
\usepackage{physics}
\usepackage{afterpage}
\usepackage{enumitem}
\usepackage[utf8]{inputenc}
\usepackage{mathrsfs}
\usepackage{scalerel}
\usepackage{stackengine,wasysym}
\usepackage{aligned-overset}
\usepackage{stackengine}
\usepackage{mathtools}
\usepackage{nccmath}
\graphicspath{ {images/} }

\setlength{\oddsidemargin}{1pt}
\setlength{\evensidemargin}{1pt}
\setlength{\marginparwidth}{30pt} % these gain 53pt width
\setlength{\topmargin}{1pt}       % gains 26pt height
\setlength{\headheight}{1pt}      % gains 11pt height
\setlength{\headsep}{1pt}         % gains 24pt height
%\setlength{\footheight}{12 pt} 	  % cannot be changed as number must fit
\setlength{\footskip}{24pt}       % gains 6pt height
\setlength{\textheight}{650pt}    % 528 + 26 + 11 + 24 + 6 + 55 for luck
\setlength{\textwidth}{460pt}     % 360 + 53 + 47 for luck

\newtheorem{theorem}{Theorem}


\def\dsp{\def\baselinestretch{1.35}\large
\normalsize}
%%%%This makes a double spacing. Use this with 11pt style. If you
%%%%want to use this just insert \dsp after the \begin{document}
%%%%The correct baselinestretch for double spacing is 1.37. However
%%%%you can use different parameter.


\def\U{{\mathcal U}}

\begin{document}

\centerline{\bf Homework 12 for Math 2371}
\centerline{Zhen Yao}

\medskip

\noindent{\bf Problem 1.}
Let $\Pi$ be the plane in $\mathbb{R}^3$ defined by
\begin{align*}
    x + 2y + 3z = 0.
\end{align*}
Let $T$ be the reflection in $\mathbb{R}^3$ about the plane $\Pi$. Find the matrix representation of $T$.
\begin{proof}
The normal vector, which perpendicular to the plane $\Pi$, is $n = (1,2,3)$. For any vector $r = (x,y,z)$ and its reflection $r' = (x',y',z')$, then the normal component of $r_n'$ with respect to the plane is
\begin{align*}
    r_n' = \frac{r \cdot n}{n \cdot n} n,
\end{align*}
and then the reflected vector is
\begin{align*}
    r' & = r - 2 \frac{r \cdot n}{n \cdot n} n \\
    & = \left(x - \frac{x + 2y + 3z}{7}, y - 2\frac{x + 2y + 3z}{7}, z - 3 \frac{x + 2y + 3z}{7} \right).
\end{align*}
Thus the matrix reperesentation is 
\begin{align*}
    T = \begin{pmatrix}
        \frac{6}{7}  & -\frac{2}{7} & -\frac{3}{7} \\
        -\frac{2}{7} & \frac{3}{7}  & -\frac{6}{7} \\
        -\frac{3}{7} & -\frac{6}{7} & -\frac{2}{7}
    \end{pmatrix}.
\end{align*}
\end{proof}

\medskip

\noindent{\bf Problem 2.}
Let $V$ be the complex linear space consists of all $2 \times 2$ complex matrices, hence $\dim V = 4$. Let $E = \begin{pmatrix} 0 & 1 \\ 0 & 0 \end{pmatrix}$. We define a linear map $T: V \to V$ such that 
\begin{align*}
    T(V) = VE - EV.
\end{align*}
\begin{enumerate}[label=(\alph*)]
    \item Find all eigenvalues of the linear map $T$.
    
     \item Find the Jordan canonical form the linear map $T$.
\end{enumerate}
\begin{proof}
~\begin{enumerate}[label=(\alph*)]
    \item For any $A = \begin{pmatrix} a & b \\ c & d \end{pmatrix} \in V$, we have $T(A) = \begin{pmatrix} -c & a-d \\ 0 & c \end{pmatrix}$. We can write $A$ as $A = \begin{pmatrix} a & b & c & d \end{pmatrix}^T$, and then $T$ can be expressed as
    \begin{align*}
        T = \begin{pmatrix}
            0 & 0 & -1 & 0 \\
            1 & 0 & 0 & -1 \\
            0 & 0 & 0 & 0  \\
            0 & 0 & 1 & 0
        \end{pmatrix}.
    \end{align*}
    Then the eigenvalue of $T$ is $0$ with multiplicity $4$.
    
    \item Also, $\dim N_{T-0I} = 3$, then there are three Jordan blocks in its Jordan canonical form, which is
    \begin{align*}
        J_T = \begin{pmatrix}
            0 & 1 & 0 & 0 \\
            0 & 0 & 0 & 0 \\
            0 & 0 & 0 & 0  \\
            0 & 0 & 0 & 0
        \end{pmatrix}.
    \end{align*}
\end{enumerate}
\end{proof}

\medskip

\noindent{\bf Problem 3.}
Let
\begin{align*}
    A = \begin{pmatrix} 
        3 & 1 \\ 
        -2 & 0 
    \end{pmatrix}.
\end{align*}
Calculate 
\begin{align*}
    \sin A = \sum^\infty_{k=0} \frac{(-1)^k}{(2k+1)!}A^{2k+1}.
\end{align*}
\begin{proof}
The characteristic polynomial of $A$ is $\lambda^2 - 3\lambda + 3 = 0$. Then the eigenvalues are $\lambda = 1$, and $\lambda = 2$. Also, the eigenvectors are $\begin{pmatrix} -1 \\ 2 \end{pmatrix}$ and $\begin{pmatrix} -1 \\ 1 \end{pmatrix}$ respectively. Denote by 
\begin{align*}
    S = \begin{pmatrix} 
        -1 & -1 \\ 
        2 & 1 
    \end{pmatrix}, \quad \Lambda = \begin{pmatrix} 
        1 & 0 \\ 
        0 & 2 
    \end{pmatrix},
\end{align*}
then $A = S \Lambda S^{-1}$. Then,
\begin{align*}
    \sin A & = S \left(\sum^\infty_{k=0}  \frac{(-1)^k}{(2k+1)!} \Lambda^{2k+1} \right) S^{-1} \\
    & = S \begin{pmatrix} 
        \sin 1 & 0 \\ 
        0 & \sin 2 
    \end{pmatrix} S^{-1} \\
    & = \begin{pmatrix} 
        -\sin 1 - 2\sin 2 & -\sin 1 - \sin 2 \\ 
        -2\sin 1 - 2\sin 2 & -2\sin 1 - \sin 2
    \end{pmatrix}.
\end{align*}
\end{proof}




\end{document}