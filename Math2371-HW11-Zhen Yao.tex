\documentclass[12pt]{article} 
%\documentclass[12pt,leqno]{amsart}
\pagestyle{plain}
\usepackage[utf8]{inputenc}
\usepackage[english]{babel}

\usepackage{latexsym,amsmath,amssymb}
\usepackage{amsthm}
%\usepackage[notref,notcite]{showkeys}
\usepackage{amsfonts}
\usepackage{geometry}
\usepackage{graphicx}
\usepackage{lmodern}
\usepackage{pifont}
\usepackage{tikz}
\usepackage{pgfplots}
\usepackage{thmtools}
\usepackage{wrapfig}
\usepackage{extarrows}
\usepackage{breqn}
\usepackage{physics}
\usepackage{afterpage}
\usepackage{enumitem}
\usepackage[utf8]{inputenc}
\usepackage{mathrsfs}
\usepackage{scalerel}
\usepackage{stackengine,wasysym}
\usepackage{aligned-overset}
\usepackage{stackengine}
\usepackage{mathtools}
\usepackage{nccmath}
\graphicspath{ {images/} }

\setlength{\oddsidemargin}{1pt}
\setlength{\evensidemargin}{1pt}
\setlength{\marginparwidth}{30pt} % these gain 53pt width
\setlength{\topmargin}{1pt}       % gains 26pt height
\setlength{\headheight}{1pt}      % gains 11pt height
\setlength{\headsep}{1pt}         % gains 24pt height
%\setlength{\footheight}{12 pt} 	  % cannot be changed as number must fit
\setlength{\footskip}{24pt}       % gains 6pt height
\setlength{\textheight}{650pt}    % 528 + 26 + 11 + 24 + 6 + 55 for luck
\setlength{\textwidth}{460pt}     % 360 + 53 + 47 for luck

\newtheorem{theorem}{Theorem}


\def\dsp{\def\baselinestretch{1.35}\large
\normalsize}
%%%%This makes a double spacing. Use this with 11pt style. If you
%%%%want to use this just insert \dsp after the \begin{document}
%%%%The correct baselinestretch for double spacing is 1.37. However
%%%%you can use different parameter.


\def\U{{\mathcal U}}

\begin{document}

\centerline{\bf Homework 11 for Math 2371}
\centerline{Zhen Yao}

\medskip

\noindent{\bf Problem 1.}
Suppose $V$ is the space of complex polynomials in $x$ and $y$ of total degree at most $3$, i.e.
\begin{align*}
    V = \left\{\sum_{i,j \geq 0,\,i+j \leq 3} a_{ij} x^i y^j \Bigg| a_{ij} \in \mathbb{C} \right\}.
\end{align*}
Consider $T: V \to V$ defined by 
\begin{align*}
    T: p \mapsto y \frac{\partial p}{\partial y}.
\end{align*}
Find the Jordan canonical form of $T$.
\begin{proof}
For any $p \in V$, we have
\begin{align*}
    p = a_{00} + a_{10}x + a_{01}y + a_{11}xy + a_{21}x^2y + a_{12}xy^2,
\end{align*}
and 
\begin{align}
    T(p) = a_{01}y + a_{11}xy + a_{21}x^2y + 2 a_{12}xy^2.
\end{align}
Thus, $T$ has the form 
\begin{align*}
    T = \begin{pmatrix}
        0 & 0 & 0 & 0 & 0 & 0 \\
        0 & 0 & 0 & 0 & 0 & 0 \\
        0 & 0 & 1 & 0 & 0 & 0 \\
        0 & 0 & 0 & 1 & 0 & 0 \\
        0 & 0 & 0 & 0 & 1 & 0 \\
        0 & 0 & 0 & 0 & 0 & 2
    \end{pmatrix}.
\end{align*}
And the characteristic polynomial of $T$ is 
\begin{align*}
    (\lambda - 1)^3 (\lambda - 2) = 0,
\end{align*}
then the eigenvalues are $\lambda = 0, 1, 2$. For $\lambda = 0$, we know that $\dim N_{A} = 2$, then there are two Jordan blocks corresponding to $\lambda = 0$. For $\lambda = 1$, $\dim N_{A - I} = 3$, then there are three Jordan blocks corresponding to $\lambda = 1$. Thus, the Jordan canonical form of $T$ is 
\begin{align*}
    J_T = \begin{pmatrix}
        0 & 0 & 0 & 0 & 0 & 0 \\
        0 & 0 & 0 & 0 & 0 & 0 \\
        0 & 0 & 1 & 0 & 0 & 0 \\
        0 & 0 & 0 & 1 & 0 & 0 \\
        0 & 0 & 0 & 0 & 1 & 0 \\
        0 & 0 & 0 & 0 & 0 & 2
    \end{pmatrix}.
\end{align*}
\end{proof}

\medskip

\noindent{\bf Problem 2.}
Let $A$ be an $n \times n$ complex matrix with $\det A \neq 0$. Suppose $A^3$ is diagonalizable. Prove that $A^2$ is diagonalizable. Provide a counter example if $\det A = 0$.
\begin{proof}
~\begin{enumerate}[label=(\alph*)]
    \item If $\det A \neq 0$, then $\det A^3 \neq 0$. Also, $A^3$ is diagonalizable, then $A^3$ has $n$ nonzero distinct eigenvalues. Then the Jordan canonical form $J_{A^3}$ of $A^3$ has $n$ size one blocks, each corresponding to its eigenvalues $\lambda_j, 1\leq j \leq n$. 
    
    Now consider the Jordan canonical form $J_{A}$ of $A$, then there exists $P$ such that $A = P J_{A} P^{-1}$ and 
    \begin{align*}
        P J_A^3 P^{-1} = P J_{A^3} P^{-1} = A^3,
    \end{align*}
    which implies each block in $J_{A^3}$ is power to the block in $J_A$, making each block in $J_A$ a size one block. Indeed, for each $\lambda_j$, if the multiplicity of $\lambda_j$ is greater than one, then its corresponding Jordan block satisfies
    \begin{align*}
        \begin{pmatrix}
            \lambda_j & 1         & \cdots & \cdots & 0 \\
                      & \lambda_j & \cdots & \cdots & 0 \\
                      &           & \ddots &        & \vdots \\
                      &           &        & \ddots & 1 \\
                      &           &        &        & \lambda_j
        \end{pmatrix}^n = 
        \begin{pmatrix}
            \begin{array}{cc}\lambda_j & \\ & \ddots \end{array} & * \\
            0 & \lambda_j
        \end{pmatrix}.
    \end{align*}
    This shows that each Jordan block of $A$ is size one. And so is $A^2$, then $A^2$ also has $n$ nonzero eigenvalues. Hence $A^2$ is also diagonalizable. 
    
    \item Take $A = \begin{pmatrix} 1 & 1 \\ 1 & -1  \end{pmatrix}$, and then $\det A = 0$. Then, $A^2 = \begin{pmatrix} 2 & 0 \\ 0 & 2  \end{pmatrix}$, and hence $A^2$ is not diagonalizable since it has repeated eigenvalues $\lambda = 2$. Also, $A^3 = \begin{pmatrix} 2 & 2 \\ 2 & -2  \end{pmatrix}$, and its eigenvalues are $\lambda = 2\sqrt{2}, - 2 \sqrt{2}$, which implies $A^3$ is diagonalizable.
\end{enumerate}
\end{proof}

\medskip

\noindent{\bf Problem 3.}
Let $A,B$ be $n \times n$ matrices. Suppose $A$ is a Hermitian matrix.
\begin{enumerate}[label=(\alph*)]
    \item Show that $A^{2015}B = B A^{2015}$ implies that $A^{2014}B = B A^{2014}$.
    
    \item Is it true that $A^{2014}B = B A^{2014}$ implies that $A^{2015}B = B A^{2015}$?
\end{enumerate}
\begin{proof}
~\begin{enumerate}[label=(\alph*)]
    \item Since $A$ is a Hermitian matrix, then there exists invertible matrix $P$ such that $A = P \Lambda P^{-1}$, where
    $\Lambda = \left[\lambda_1, \cdots, \lambda_n\right]$ is a diagonal matrix. Let $\widetilde{B} = P^{-1}\Lambda P$, then $$\Lambda^{2015} \widetilde{B} = \widetilde{B} \Lambda^{2015}.$$
    Then for $ij$-th entry, we have $\lambda_i^{2015} b_{ij} = b_{ij} \lambda_j^{2015}$. Then $\lambda_i = \lambda_j$, and hence $\lambda_i^{2014} b_{ij} = b_{ij} \lambda_j^{2014}$. Hence, $A^{2014}B = B A^{2014}$.
    
    \item It is not true. Since similarly to argument in $(a)$, we have $\lambda_i^{2014} b_{ij} = b_{ij} \lambda_j^{2014}$, and then $\lambda_i = \lambda_j$ or $\lambda_i = - \lambda_j$. In the case $\lambda_i = - \lambda_j$, we have $\lambda_i^{2015} b_{ij} = - b_{ij} \lambda_j^{2015}$. Thus, $A^{2015}B \neq B A^{2015}$.
\end{enumerate}
\end{proof}

\medskip

\noindent{\bf Problem 4.}
Let $A$ be a complex $n \times n$ matrix which commutes with all reflection matrices of the form
\begin{align*}
    R = I - 2 v v^T,
\end{align*}
where $v \in \mathbb{R}^n$ is a unit vector. Is it true that $A = kI$ for some scalar $k$?
\begin{proof}
For any $R$, we have $A \left(I - 2 v v^T\right) = \left(I - 2 v v^T\right) A$, then we have $A v v^T = v v^T A$. Then, $A$ maps $N_{v v^T - \lambda I} \to N_{v v^T - \lambda I}$, for some eigenvalue $\lambda$ of $v v^T$. Indeed, for any $x \in N_{v v^T - \lambda I}$, we have $A v v^T x - \lambda Ax = 0$, with $A v v^T = v v^T A$, we have $v v^T(Ax) = \lambda (Ax)$.

Then for all $x\in N_{v v^T - \lambda I}$, there exists $\lambda \in \mathbb{C}$, we have $Ax = \lambda x$. Then, $A = kI$, otherwise, it cannot have above property.
\end{proof}














\end{document}