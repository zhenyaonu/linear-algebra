\documentclass[12pt,leqno]{amsart}
\pagestyle{plain}
\usepackage{latexsym,amsmath,amssymb}
%\usepackage[notref,notcite]{showkeys}
\usepackage{amsfonts}
\usepackage{geometry}
\usepackage{graphicx}
\graphicspath{ {images/} }

\setlength{\oddsidemargin}{1pt}
\setlength{\evensidemargin}{1pt}
\setlength{\marginparwidth}{30pt} % these gain 53pt width
\setlength{\topmargin}{1pt}       % gains 26pt height
\setlength{\headheight}{1pt}      % gains 11pt height
\setlength{\headsep}{1pt}         % gains 24pt height
%\setlength{\footheight}{12 pt} 	  % cannot be changed as number must fit
\setlength{\footskip}{24pt}       % gains 6pt height
\setlength{\textheight}{650pt}    % 528 + 26 + 11 + 24 + 6 + 55 for luck
\setlength{\textwidth}{460pt}     % 360 + 53 + 47 for luck

\newtheorem{theorem}{Theorem}



\def\dsp{\def\baselinestretch{1.35}\large
\normalsize}
%%%%This makes a double spacing. Use this with 11pt style. If you
%%%%want to use this just insert \dsp after the \begin{document}
%%%%The correct baselinestretch for double spacing is 1.37. However
%%%%you can use different parameter.


\def\U{{\mathcal U}}

\begin{document}

\centerline{\bf Homework 2 for Math 2371}
\centerline{Zhen Yao}

\medskip

\noindent{\bf Problem 1.} Let $Y$ be a subspace of a finite dimensional linear space $X$ such that 
\begin{align*}
    \dim Y = \dim X - k.
\end{align*}
Let $Z$ be a subspace of $X$. Show that 
\begin{align*}
    \dim (Z\cap Y) \geq \dim Z - k.
\end{align*}
\begin{proof}
With $\dim (Y+Z) = \dim Y + \dim Z - \dim (Y\cap Z)$ and $\dim(Y+Z) \leq \dim X$, then we have
\begin{align*}
    \dim (Y\cap Z) & = \dim Y + \dim Z - \dim (Y+Z) \\
    & \geq \dim Z + \dim X - k - \dim X \\
    & \geq \dim Z - k.
\end{align*}
\end{proof}

\medskip

\noindent{\bf Problem 2.} Let $v_1,v_2,\cdots, v_k, k\geq 2$, be vectors in $\mathbf{R}^n$ and $1\leq s < k$. Show that
\begin{align*}
    \det G(v_1,\cdots, v_k) \leq \det G(v_1,\cdots, v_s) \det G(v_{s+1},\cdots, v_k).
\end{align*}
where $G(v_1,\cdots, v_k)$ is the Gram matrix of vectors $v_1,\cdots, v_k$ with the standard inner product.
\begin{proof}
We denote 
$G(v_1,\cdots, v_k) = \begin{pmatrix}
    G(v_1,\cdots, v_s) & B \\
    B^* & G(v_{s+1},\cdots, v_k)
    \end{pmatrix} = \begin{pmatrix}
    A & B \\
    B^* & D
    \end{pmatrix}$. We assume $D > 0$, and
with Schur complement, we have
\begin{align*}
    \frac{G(v_1,\cdots, v_k)}{D} = A - B D^{-1} B^*.
\end{align*}
Then we have $\det G(v_1,\cdots, v_k) = \det D \det (A - B D^{-1} B^*)$.
\end{proof}

\medskip

\noindent{\bf Problem 3.} Find the polar decomposition of 
\begin{align*}
    A = \begin{pmatrix}
    1 & 2 \\
    2 & 3
    \end{pmatrix}.
\end{align*}
\begin{proof}
We have $A^T A = \begin{pmatrix}
    10 & 8 \\
    8 & 8
    \end{pmatrix}$, and the eigenvalues are $9+\sqrt{65}, 9-\sqrt{65}$. In polar decomposition, $A = QS$, and we have $S = \begin{pmatrix}
    \sqrt{9+\sqrt{65}} & 0 \\
    0 & \sqrt{9+\sqrt{65}}
    \end{pmatrix}$. Thus, we have $Q = AS^{-1} = \begin{pmatrix}
    \frac{1}{\sqrt{9+\sqrt{65}}} & \frac{2}{\sqrt{9-\sqrt{65}}} \\
    \frac{2}{\sqrt{9+\sqrt{65}}} & \frac{3}{\sqrt{9-\sqrt{65}}}
    \end{pmatrix}$.
\end{proof}

\medskip

\noindent{\bf Problem 4.} Let $A$ be self-adjoint. Show that the singular values of $A$ are absolute values of eigenvalues of $A$.
\begin{proof}
In singular value decomposition, we have $A = WDV$, where $W,V$ are unitary, and $D\geq 0$ is diagonal. Then we have
\begin{align*}
    AA^* = W D V V^* D W^* = W D^2 W^*,
\end{align*}
which implies the singular values are eigenvalues of $AA^*$, i.e., $\sigma\left(D^2\right) = \sigma(AA^*)$. 

Also, $A$ is self-adjoint, and suppose $\lambda$ is an eigenvalue of $A$ with corresponding eigenvector $v$. Then $\overline{\lambda}$ is an eigenvalue of $A^*$ with the same eigenvector $v$. Suppose $\lambda_1,\cdots,\lambda_n$ are eigenvalues of $A$ with eigenvectors $v_1,\cdots,v_n$, then we have
\begin{align*}
    AA^*v_j = A \overline{\lambda_j}v_j = \lambda_j \overline{\lambda_j}v_j = |\lambda_j|^2 v_j,
\end{align*}
which implies that the eigenvalues of $AA^*$ are $|\lambda_1|^2,\cdots,|\lambda_n|^2$, then we have $\sigma(D) = |\lambda_j|, 1\leq j \leq n$.
\end{proof}












\end{document}
